\lstset{language=C++,
    keywordstyle={\bfseries\color{RoyalBlue}},
    basicstyle=\small\ttfamily,
    commentstyle=\color{ForestGreen}\ttfamily,
    rulecolor=\color{black},
    upquote=true,
    numberstyle=\tiny\color{gray},
    stepnumber=1,
    numbersep=8pt,
    showstringspaces=false,
    breaklines=true,
    frame=single,
%belowcaptionskip=3mm,
    tabsize=4,
    captionpos=b
%escapechar=\@
}

\lstdefinestyle{sidebyside}
{
    basicstyle=\small\ttfamily,
    framexleftmargin=3pt,
    xleftmargin=8pt,
    numbersep=-4pt,
}


\definecolor{set1red}{RGB}{228, 26, 28}
\definecolor{set1green}{RGB}{77, 175, 74}
\definecolor{set1violet}{RGB}{152, 78, 163}
\definecolor{set1orange}{RGB}{255, 127, 0}

\lstdefinestyle{asmcode}{
    language={[x86masm]Assembler},
    escapechar=|,
    frame=none,
    keywordstyle={\bfseries\color{RoyalBlue}},
    basicstyle=\small\ttfamily,
    commentstyle=\color{ForestGreen}\ttfamily,
    rulecolor=\color{black},
    upquote=true,
    numberstyle=\tiny\color{gray},
    stepnumber=1,
    numbersep=8pt,
    showstringspaces=false,
    breaklines=true,
%belowcaptionskip=3mm,
    tabsize=4,
    captionpos=b,
    morekeywords={vmovdqa,vmovdqu,vzeroupper},
}

\lstdefinestyle{coloredrads}{
    keywords=[2]{BTRA1},
    keywords=[3]{BTRA2},
    keywords=[4]{BTRA3},
    keywords=[5]{BTRA3},
    keywordstyle=[2]{\color{set1red}},
    keywordstyle=[3]{\color{set1green}},
    keywordstyle=[4]{\color{set1violet}},
    keywordstyle=[5]{\color{set1orange}},
}

\AtBeginDocument{\DeclareCaptionSubType{lstlisting}}

\newcommand{\makename}[3][s]{%
    \expandafter\newcommand\csname #2\endcsname{#3\xspace}%
    \expandafter\newcommand\csname #2s\endcsname{#3#1\xspace}%
}

\newcommand{\sbrtodo}[1]{\todo[inline,color=NavyBlue!80]{\textcolor{white}{sbr: #1}}}
\newcommand{\fbetodo}[1]{\todo[inline,color=ForestGreen!80]{\textcolor{white}{fbe: #1}}}


\newcommand{\prj}[1][]{%
    \ifthenelse{\isempty{#1}}%
    {\textsc{IRIdium}}%
    {IRIdium}%
    \xspace%
}

\makename{cra}{code-reuse attack}
\makename{btra}{BTRA}
\makename{heapbt}{BTDP}

\newcommand{\specPerfAvg}{1.006$\times$}
\newcommand{\specQuantRetAddr}{70\%}
\newcommand{\wox}{W$\oplus$X\xspace}
\newcommand{\plusplus}{\nolinebreak\hspace{-.02em}\raisebox{.1ex}{\small +}\nolinebreak\hspace{-.10em}\raisebox{.1ex}{\small +}\xspace}
\newcommand{\cpp}{C\plusplus}
\newcommand{\maxoverheadcfs}{3.61\%\xspace}
\newcommand{\geomeancfs}{0.79\%\xspace}
%\newcommand{\geomeanonlylib}{1.02\%\xspace}
\newcommand{\geomeannobt}{2.44\%\xspace} % geometric mean overhead when moving only the stack pointer
\newcommand{\avxomnetppimprov}{14\%\xspace} % improvvement in absolute percent points of omnetpp with AVX2
\newcommand{\hardenedheapbtsoverhead}{2\%\xspace} % overhead in absolute percent points of hardened heap booby traps compared to regular heap booby traps
\newcommand{\meanpercentcodepointers}{97\%\xspace}
\newcommand{\protectedcodepointers}{93\%\xspace}
\newcommand{\heapbtgeomean}{1.59\%\xspace}
\newcommand{\allprotectionsgeomean}{7.48\%\xspace}
\newcommand{\allprotectionspushgeomean}{12.47\%\xspace}
\newcommand{\allprotectionshardenedgeomean}{9.89\%\xspace}
%\newcommand{\adaptivepronegeomean}{4.19\%\xspace}
%\newcommand{\adaptiveprtwogeomean}{7.19\%\xspace}
\newcommand{\eg}{e.g.,~}
\newcommand{\ie}{i.e.,~}
\newcommand{\etal}{et al\@.\xspace}
\newcommand{\cfs}{offset-invariant addressing\xspace}
\newcommand{\Cfs}{Offset-invariant addressing\xspace}
\newcommand{\code}{\texttt}
\newcommand{\propername}{\textsf}
\newcommand{\krx}{kR\^{}X\xspace}

\newcommand{\musllib}{\propername{musl}/\propername{libc\plusplus}\xspace}
\newcommand{\glibclib}{\propername{glibc}/\propername{libstdc\plusplus}\xspace}
\newcommand*{\figuretitle}[1]{%
        {\centering%   <--------  will only affect the title because of the grouping (by the
    \sffamily{\textbf{#1}}%              braces before \centering and behind \medskip). If you remove
    \par\medskip}%            these braces the whole body of a {figure} env will be centered.
}

\newacronym{CPH}{CPH}{Code-Pointer Hiding}
\newacronym{XOM}{XOM}{Execute-Only Memory}
\newacronym{AOCR}{AOCR}{Address-Oblivious Code Reuse}
\newacronym[plural=TLBs,firstplural=Translation Lookaside Buffers (TLBs)]{TLB}{TLB}{Translation Lookaside Buffer}
\newacronym[plural=EPTs,firstplural=Extended Page Tables (EPTs)]{EPT}{EPT}{Extended Page Table}
\newacronym{PIC}{PIC}{Position Independent Code}
\newacronym{GOT}{GOT}{Global Offset Table}
\newacronym{CFI}{CFI}{Control-Flow Integrity}
\newacronym{SFI}{SFI}{Software-based Fault Isolation}
\newacronym{CFG}{CFG}{Control-Flow Graph}
\newacronym{RSB}{RSB}{Return Stack Buffer}
\newacronym{PGO}{PGO}{Profile-Guided Optimization}
\newacronym{LTO}{LTO}{Link-Time Optimization}
\newacronym{BTB}{BTB}{Branch Target Buffer}
\newacronym{PC}{PC}{Program Counter}
\newacronym{CMQ}{CMQ}{Cross-Module Quickening}

\ifxetex
% WORKAROUND for xelatex vs tikzmark:
% Definition copied from /usr/share/texlive/texmf-dist/tex/generic/pgf/systemlayer/pgfsys-common-pdf-via-dvi.def
% Compare https://tex.stackexchange.com/q/229500 and comments!
\makeatletter
\def\pgfsys@hboxsynced#1{%
        {%
        \pgfsys@beginscope%
        \setbox\pgf@hbox=\hbox{%
            \hskip\pgf@pt@x%
            \raise\pgf@pt@y\hbox{%
                \pgf@pt@x=0pt%
                \pgf@pt@y=0pt%
                \special{pdf: content q}%
                \pgflowlevelsynccm%
                \pgfsys@invoke{q -1 0 0 -1 0 0 cm}%
                \special{pdf: content -1 0 0 -1 0 0 cm q}% translate to original coordinate system
                \pgfsys@invoke{0 J [] 0 d}% reset line cap and dash
                \wd#1=0pt%
                \ht#1=0pt%
                \dp#1=0pt%
                \box#1%
                \pgfsys@invoke{n Q Q Q}%
            }%
            \hss%
        }%
        \wd\pgf@hbox=0pt%
        \ht\pgf@hbox=0pt%
        \dp\pgf@hbox=0pt%
        \pgfsys@hbox\pgf@hbox%
        \pgfsys@endscope%
    }%
}
\makeatother

\newfontfamily\firasans[Path=fonts/,
    UprightFont = *-Regular,
    BoldFont = *-Bold,
    ItalicFont = *-Italic,
    BoldItalicFont = *-BoldItalic,
    FontFace={xl}{n}{*-ExtraLight},
    FontFace={xl}{it}{*-ExtraLightItalic},
    FontFace={l}{n}{*-Light},
    FontFace={l}{it}{*-LightItalic},
    FontFace={mb}{n}{*-Medium},
    FontFace={mb}{it}{*-MediumItalic},
    FontFace={k}{n}{*-Black},
    FontFace={k}{it}{*-BlackItalic},
    Extension = .ttf,
]{FiraSans}

\newfontfamily\firacode[Path=fonts/,
    UprightFont = *-Regular,
    BoldFont = *-Bold,
    FontFace={l}{n}{*-Light},
    FontFace={m}{n}{*-Medium},
    FontFace={sb}{n}{*-SemiBold},
    Extension = .ttf,
]{FiraCode}

\newcommand{\figurefont}[1]{{\sf{\firasans #1}}}%
\else
\newcommand{\figurefont}[1]{#1}%
\fi

\newcommand*\circledtikz[1]{\tikz[baseline=(char.base)]{
    \node[shape=circle,draw,inner sep=1pt] (char) {#1};}}
\newcommand*\circleddashedtikz[1]{\tikz[baseline=(char.base)]{
    \node[shape=circle,dashed,draw,inner sep=1pt] (char) {#1};}}
