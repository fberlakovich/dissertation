% Font setup for LuaLaTeX - Palatino with Noto Mono for code
\usepackage{fontspec}
\usepackage{unicode-math}
\usepackage{tabularray}
\usepackage{fp}
\usepackage{mathtools}
\usepackage{subcaption}
\usepackage[usenames,dvipsnames,table]{xcolor} % Allows the definition and use of colors. This package has to be included before tikz.
\usepackage{nag}       % Issues warnings when best practices in writing LaTeX documents are violated.
\usepackage{todonotes} % Provides tooltip-like todo notes.
\usepackage[acronym,toc]{glossaries} % Enables acronyms
\usepackage{xspace} % automatic space after user defined commands
\usepackage[normalem]{ulem}
\usepackage{listings} % source code listings
\usepackage{mwe} % placeholder images
\usepackage{tikz}
\usepackage{circledsteps}
\usepackage[latex=false]{svg}
\usepackage{xifthen}% provides \isempty test
\usepackage{placeins}
\usepackage{iftex}
\usepackage[framemethod=tikz]{mdframed}
\usepackage{siunitx}
\usepackage{pgfplots}
\usepackage{textgreek}
\usepackage{xparse} % <-- Add this package
\usepackage{luacode}

% Check Python version compatibility for minted/latexminted
% Python 3.14+ has breaking changes in argparse that break latexminted
\begin{luacode}
  local handle = io.popen("python3 --version 2>&1")
  if handle then
    local result = handle:read("*a")
    handle:close()
    local major, minor = result:match("Python ([0-9]+)%.([0-9]+)")
    if major and minor then
      major, minor = tonumber(major), tonumber(minor)
      if major >= 3 and minor >= 14 then
        tex.error("Python " .. major .. "." .. minor .. " detected but latexminted requires Python < 3.14. Fix: brew unlink python@3.14")
      end
    end
  end
\end{luacode}

\usepackage[newfloat,cachedir=minted-cache]{minted}
\usepackage{booktabs}
\usepackage{csquotes}
\usepackage[most]{tcolorbox}
\usepackage{pifont}
\usepackage[pdfpagelabels,hidelinks,pdfusetitle]{hyperref}
\usepackage{bookmark}
\usepackage{multirow}
\sisetup{round-mode=places,round-precision=3,round-pad=false,group-minimum-digits=4,group-separator={,},scientific-notation=fixed}
\usepackage{array}
\UseTblrLibrary{booktabs,functional}
\usepackage{tabularx}
\usepackage[htt]{hyphenat}
\usepackage{collcell}   % To capture cell content
\usepackage{makecell}

% --- FLOAT TUNING START ---
% Encourage floats to be placed on the page where they are defined
\renewcommand{\topfraction}{0.85}
\renewcommand{\bottomfraction}{0.85}
\renewcommand{\textfraction}{0.1}
\renewcommand{\floatpagefraction}{0.75}
\setcounter{topnumber}{5}
\setcounter{bottomnumber}{5}
\setcounter{totalnumber}{10}
% --- FLOAT TUNING END ---

% Cleveref - must be loaded after hyperref
\usepackage[capitalize,noabbrev,nameinlink]{cleveref}

% Enumitem for inline lists
\usepackage[inline]{enumitem}

\setmainfont{TeX Gyre Pagella}
\setmathfont{TeX Gyre Pagella Math}
\setmonofont{Noto Sans Mono}[Scale=0.9]


\sisetup{round-mode=places,round-precision=3,round-pad=false,group-minimum-digits=4,group-separator={,},scientific-notation=fixed}
\UseTblrLibrary{booktabs,functional}
\usetikzlibrary{arrows.meta,arrows,automata,positioning,tikzmark,decorations.pathreplacing,backgrounds,calc}
%\usetikzlibrary{external}
%\tikzexternalize[prefix=figures/tikz-ext/]
%\DeclareUnicodeCharacter{2212}{−}
\usepgfplotslibrary{groupplots,dateplot}
\usetikzlibrary{patterns,shapes.arrows, arrows.meta, matrix}
\pgfplotsset{compat=newest}

\pgfplotsset{every axis/.append style={
    scaled y ticks = false,
    scaled x ticks = false,
    y tick label style = {/pgf/number format/fixed},
    x tick label style = {/pgf/number format/fixed},
    legend style={font=\small}
}}

\lstset{language=C++,
    keywordstyle={\bfseries\color{RoyalBlue}},
    basicstyle=\small\ttfamily,
    commentstyle=\color{ForestGreen}\ttfamily,
    rulecolor=\color{black},
    upquote=true,
    numberstyle=\tiny\color{gray},
    stepnumber=1,
    numbersep=8pt,
    showstringspaces=false,
    breaklines=true,
    frame=single,
%belowcaptionskip=3mm,
    tabsize=4,
    captionpos=b
%escapechar=\@
}

\lstdefinestyle{sidebyside}
{
    basicstyle=\small\ttfamily,
    framexleftmargin=3pt,
    xleftmargin=8pt,
    numbersep=-4pt,
}


\definecolor{set1red}{RGB}{228, 26, 28}
\definecolor{set1green}{RGB}{77, 175, 74}
\definecolor{set1violet}{RGB}{152, 78, 163}
\definecolor{set1orange}{RGB}{255, 127, 0}

\lstdefinestyle{asmcode}{
    language={[x86masm]Assembler},
    escapechar=|
,
    frame=none,
    keywordstyle={\bfseries\color{RoyalBlue}},
    basicstyle=\small\ttfamily,
    commentstyle=\color{ForestGreen}\ttfamily,
    rulecolor=\color{black},
    upquote=true,
    numberstyle=\tiny\color{gray},
    stepnumber=1,
    numbersep=8pt,
    showstringspaces=false,
    breaklines=true,
%belowcaptionskip=3mm,
    tabsize=4,
    captionpos=b,
    morekeywords={vmovdqa,vmovdqu,vzeroupper},
}

\lstdefinestyle{coloredrads}{
    keywords=[2]{BTRA1},
    keywords=[3]{BTRA2},
    keywords=[4]{BTRA3},
    keywords=[5]{BTRA3},
    keywordstyle=[2]{\color{set1red}},
    keywordstyle=[3]{\color{set1green}},
    keywordstyle=[4]{\color{set1violet}},
    keywordstyle=[5]{\color{set1orange}},
}

\AtBeginDocument{\DeclareCaptionSubType{lstlisting}}

\makeatletter
\newcommand{\closepart}{
    \begingroup
    % Prevent \part* from forcing a page break
    \let\clearpage\relax
    \let\cleardoublepage\relax
    % Start an unnumbered, empty part:
    % - no page
    % - no TOC entry
    % - but it resets the “current part” state / marks
    \part*{}
    \endgroup
}
\makeatother

\newcommand{\sbrtodo}[1]{\todo[inline,color=NavyBlue!80]{\textcolor{white}{sbr: #1}}}
\newcommand{\fbetodo}[1]{\todo[inline,color=ForestGreen!80]{\textcolor{white}{fbe: #1}}}



\newcommand{\includetimed}[1]{
    \timerstart{#1}%
    \include{#1}%
    \timerprint{#1}%
}


% --- helpers you likely already have ---
\newsavebox{\drawioMacroBox}
\newdimen\drawioTargetWidth

% --- main macro (Lua-driven) ---
\makeatletter
\DeclareDocumentCommand{\includeDrawioFigure}{ O{\textwidth} m O{} +m }{
% #1 target width (TeX dimen, default \textwidth)
% #2 base name (expects #2.pdf, #2-defs.tex, #2-coords.tex)
% #3 tikz options
% #4 overlay tikz code

% typeset the PDF into a box so we can measure it
    \savebox{\drawioMacroBox}{\includegraphics[width=#1]{#2.pdf}}%
    \input{#2-defs.tex}% defines \drawionativewidthpx, \drawionativeheightpx

    % capture target width as a dimen
    \drawioTargetWidth=#1\relax

    % the formatter gets confused with the braces in the lua code
    % @formatter:off
    \directlua{
        local function sp_to_pt(sp) return sp/65536.0 end
        local function pt_to_bp(pt) return pt * 72.0 / 72.27 end
        local w_bp = pt_to_bp(sp_to_pt(\number\wd\drawioMacroBox))
        local h_bp = pt_to_bp(sp_to_pt(\number\ht\drawioMacroBox))
        local nat_w = \drawionativewidthpx
        local nat_h = \drawionativeheightpx
        local xbp = w_bp / nat_w
        local ybp = h_bp / nat_h
        tex.sprint("\\def\\drawioXbp{" .. xbp .. "}\\def\\drawioYbp{" .. ybp .. "}")
    }%

    \begin{tikzpicture}[x=\drawioXbp bp, y=\drawioYbp bp, #3]
        \node[anchor=south west, inner sep=0] at (0,0) {\usebox{\drawioMacroBox}};
        \input{#2-coords.tex}
        #4
    \end{tikzpicture}%
    % @formatter:on
}
\makeatother

\newcommand{\elbow}[3][0.5]{($ (#2)!#1!(#3) $)}
\newcommand{\specPerfAvg}{1.006$\times$}
\newcommand{\specQuantRetAddr}{70\%}
\newcommand{\wox}{W$\oplus$X\xspace}
\newcommand{\plusplus}{\nolinebreak\hspace{-.02em}\raisebox{.1ex}{\small +}\nolinebreak\hspace{-.10em}\raisebox{.1ex}{\small +}\xspace}
\newcommand{\cpp}{C\plusplus}
\newcommand{\maxoverheadcfs}{3.61\%\xspace}
\newcommand{\geomeancfs}{0.79\%\xspace}
%\newcommand{\geomeanonlylib}{1.02\%\xspace}
\newcommand{\geomeannobt}{2.44\%\xspace} % geometric mean overhead when moving only the stack pointer
\newcommand{\avxomnetppimprov}{14\%\xspace} % improvvement in absolute percent points of omnetpp with AVX2
\newcommand{\hardenedheapbtsoverhead}{2\%\xspace} % overhead in absolute percent points of hardened heap booby traps compared to regular heap booby traps
\newcommand{\meanpercentcodepointers}{97\%}
\newcommand{\protectedcodepointers}{93\%}
\newcommand{\heapbtgeomean}{1.59\%\xspace}
\newcommand{\allprotectionsgeomean}{7.48\%\xspace}
\newcommand{\allprotectionspushgeomean}{12.47\%\xspace}
\newcommand{\allprotectionshardenedgeomean}{9.89\%\xspace}
%\newcommand{\adaptivepronegeomean}{4.19\%\xspace}
%\newcommand{\adaptiveprtwogeomean}{7.19\%\xspace}
\newcommand{\eg}{e.g.,~}
\newcommand{\ie}{i.e.,~}
\newcommand{\etal}{et al\@.\xspace}
\newcommand{\cfs}{offset-invariant addressing\xspace}
\newcommand{\Cfs}{Offset-invariant addressing\xspace}
\newcommand{\code}{\texttt}
\newcommand{\krx}{kR\^{}X\xspace}
\newcommand{\jacoco}{JaCoCo\xspace}
\newcommand{\musllib}{\propername{musl}/\propername{libc\plusplus}\xspace}
\newcommand{\glibclib}{\propername{glibc}/\propername{libstdc\plusplus}\xspace}
\newcommand*{\figuretitle}[1]{
{\centering%   <--------  will only affect the title because of the grouping (by the
\sffamily{\textbf{#1}}%              braces before \centering and behind \medskip). If you remove
\par\medskip}%            these braces the whole body of a {figure} env will be centered.
}


\newcommand{\propernamedecl}{\ttfamily\hyphenchar\font=`-\relax}
\newcommand{\propername}[1]{{\propernamedecl#1}}
\newcommand{\afl}{AFL}
\newcommand{\aflpp}{AFL\plusplus}


% @formatter:off
\newcounter{rowcounter}
\setcounter{rowcounter}{0}
\newcommand{\nextrow}{\stepcounter{rowcounter}}
\AtBeginEnvironment{tabular}{\setcounter{rowcounter}{0}}
\AtBeginEnvironment{tabularx}{\setcounter{rowcounter}{0}} % if you use tabularx
\AtBeginEnvironment{longtable}{\setcounter{rowcounter}{0}} % if you use longtable
\newcolumntype{C}{S[table-format=5.0]}
\newcolumntype{O}{S[table-format=6.1, round-precision=1]}
\newcolumntype{T}{S[table-format=3.1, round-precision=1]}
\newcolumntype{E}{>{\ifnum
                        \value{rowcounter}>0\propernamedecl
\fi\arraybackslash}l}

\definecolor{rowgray}{gray}{0.95} % A very light gray
\newcommand{\conditionalcell}[1]{
    \ifnum
        \value{rowcounter}=1
        % Row 1: Apply rotation
        \rotatebox[origin=bc]{45}{
        % Use [b] for bottom alignment
        % Use the new \rotHeadWidth
            \parbox[b]{3cm}{
                \propernamedecl
                \smaller
                \centering % Add this for clean line breaks
                #1%
            }
        }
    \else
    % Other rows: Just output the content
        #1%
    \fi
}

\newcolumntype{Y}{%
        >{\ifnum
              \value{rowcounter}=1
              \relax % Row 1: Don't apply \raggedright
    \else
              \raggedright % Other rows: Apply \raggedright
    \fi
    \arraybackslash     % Fix \raggedright
    \collectcell\conditionalcell % Collect the cell content
    }%
    X% Base column type
        <{\endcollectcell}%
}
% @formatter:on


\newacronym{CPH}{CPH}{Code-Pointer Hiding}
\newacronym{XOM}{XOM}{Execute-Only Memory}
\newacronym{AOCR}{AOCR}{Address-Oblivious Code Reuse}
\newacronym[plural=TLBs,firstplural=Translation Lookaside Buffers (TLBs)]{TLB}{TLB}{Translation Lookaside Buffer}
\newacronym[plural=EPTs,firstplural=Extended Page Tables (EPTs)]{EPT}{EPT}{Extended Page Table}
\newacronym{PIC}{PIC}{Position Independent Code}
\newacronym{GOT}{GOT}{Global Offset Table}
\newacronym{ROP}{ROP}{Return-Oriented Programming}
\newacronym{CFI}{CFI}{Control-Flow Integrity}
\newacronym{JITROP}{JIT ROP}{Just-In-Time Return-Oriented Programming}
\newacronym{SFI}{SFI}{Software-based Fault Isolation}
\newacronym{CFG}{CFG}{Control-Flow Graph}
\newacronym{RSB}{RSB}{Return Stack Buffer}
\newacronym{PGO}{PGO}{Profile-Guided Optimization}
\newacronym{LTO}{LTO}{Link-Time Optimization}
\newacronym{BTB}{BTB}{Branch Target Buffer}
\newacronym{PC}{PC}{Program Counter}
\newacronym{CMQ}{CMQ}{Cross-Module Quickening}
\newacronym{MVEE}{MVEE}{Multi-Variant Execution Engine}
\newacronym{PIROP}{PIROP}{Position-Independent Code Reuse}
\newacronym{CPI}{CPI}{Control-Pointer Integrity}
\newacronym{CCFI}{CCFI}{Cryptographically Enforced Control-Flow Integrity}
\newacronym{PAC}{PAC}{Pointer Authentication Code}
\newacronym{HMAC}{HMAC}{Hash-based Message Authentication Code}
\newacronym{BROP}{BROP}{Blind ROP}
\newacronym{DEP}{DEP}{Data Execution Prevention}
\newacronym{CHOP}{CHOP}{Catch Handler Oriented Programming}
\newacronym{ASLR}{ASLR}{Address Space Layout Randomization}
\newacronym{BTRA}{BTRA}{Booby-trapped return address}
\newacronym{heapbt}{BTDP}{Booby-trapped data pointer}
\newacronym{CPU}{CPU}{Central Processing Unit}
\newacronym{JITcomp}{JIT compiler}{Just-In-Time compiler}
\newacronym{JIT}{JIT}{Just-In-Time}
\newacronym{AOT}{AOT}{Ahead-Of-Time}
\newacronym{JVM}{JVM}{Java Virtual Machine}
\newacronym{SMT}{SMT}{Satisfiability Modulo Theory}
\newacronym{PSC}{PSC}{Program State Convolution}
\newacronym{MLTA}{MLTA}{Multi-Level Type Analysis}
\newacronym{AIR}{AIR}{Average Indirect-target Reduction}
\newacronym{ISR}{ISR}{Instruction Set Randomization}
\newacronym{LBR}{LBR}{Last Branch Record}
\newacronym{COOP}{COOP}{Counterfeit Object-Oriented Programming}
\newacronym{BOP}{BOP}{Block-Oriented Programming}
\newacronym{TEA}{TEA}{Transient-Execution Attack}
\newacronym{COP}{COP}{Call-Oriented Programming}
\newacronym{ILR}{ILR}{Instruction-Layout Randomization}
\newacronym{IPC}{IPC}{Instructions Per Cycle}
\newacronym{VRP}{VRP}{Value-Range Partitioning}

\newglossaryentry{CRA}{
name={code-reuse attack},
plural={code-reuse attacks},
description={An attack that reuses code already present in the target process}
}

\ifxetex
% WORKAROUND for xelatex vs tikzmark:
% Definition copied from /usr/share/texlive/texmf-dist/tex/generic/pgf/systemlayer/pgfsys-common-pdf-via-dvi.def
% Compare https://tex.stackexchange.com/q/229500 and comments!
\makeatletter
\def\pgfsys@hboxsynced#1+{
{% 
\pgfsys@beginscope%
\setbox\pgf@hbox=\hbox+{
\hskip\pgf@pt@x%
\raise\pgf@pt@y\hbox+{
\pgf@pt@x=0pt%
\pgf@pt@y=0pt%
\special{pdf: content q}%
\pgflowlevelsynccm%
\pgfsys@invoke{q -1 0 0 -1 0 0 cm}%
\special{pdf: content -1 0 0 -1 0 0 cm q}% translate to original coordinate system
\pgfsys@invoke{0 J [] 0 d}% reset line cap and dash
\wd#1=0pt%
\ht#1=0pt%
\dp#1=0pt%
\box#1%
\pgfsys@invoke{n Q Q Q}%
}
\hss%
}
\wd\pgf@hbox=0pt%
\ht\pgf@hbox=0pt%
\dp\pgf@hbox=0pt%
\pgfsys@hbox\pgf@hbox%
\pgfsys@endscope%
}
}
\makeatother

\newfontfamily\firasans[Path=fonts/,
UprightFont = *-Regular,
BoldFont = *-Bold,
ItalicFont = *-Italic,
BoldItalicFont = *-BoldItalic,
FontFace={xl}{n}{*-ExtraLight},
FontFace={xl}{it}{*-ExtraLightItalic},
FontFace={l}{n}{*-Light},
FontFace={l}{it}{*-LightItalic},
FontFace={mb}{n}{*-Medium},
FontFace={mb}{it}{*-MediumItalic},
FontFace={k}{n}{*-Black},
FontFace={k}{it}{*-BlackItalic},
Extension = .ttf,
]{FiraSans}

\newfontfamily\firacode[Path=fonts/,
UprightFont = *-Regular,
BoldFont = *-Bold,
FontFace={l}{n}{*-Light},
FontFace={m}{n}{*-Medium},
FontFace={sb}{n}{*-SemiBold},
Extension = .ttf,
]{FiraCode}

\newcommand{\figurefont}[1]{{\sf{\firasans #1}}}%
\else
\newcommand{\figurefont}[1]{#1}%
\fi

\newcommand*\circledtikz[1]{\tikz[baseline=(char.base)]{
\node[shape=circle,draw,inner sep=1pt] (char) {#1};}}
\newcommand*\circleddashedtikz[1]{\tikz[baseline=(char.base)]{
\node[shape=circle,dashed,draw,inner sep=1pt] (char) {#1};}}

\newcommand{\cnp}{\prj{}-\np{}\xspace}
\newcommand{\oapilong}{Optimization Interface\xspace}
\newcommand{\oapi}{OINT\xspace}

\newcommand{\makepropername}[2]{
\expandafter\newcommand\csname #1\endcsname{\propername{#2}\xspace}%
\expandafter\newcommand\csname #1noformat\endcsname{#2\xspace}%
}

\makepropername{np}{NumPy}
\makepropername{cp}{CPython}
\makepropername{npbench}{NPBench}
\makepropername{ocache}{occurence cache}
\makepropername{Ocache}{Occurence cache}

\newcommand{\cext}{C extension\xspace}
\newcommand{\cexts}{C extensions\xspace}

\newcommand{\XXX}{\fbetodo{Skipped during writing}<placeholder to be filled>}

\newcommand{\perinstrcache}{per-instruction caches\xspace}
\newcommand{\Perinstrcache}{Per-instruction caches\xspace}



\DeclareRobustCommand{\rtwoc}{
\ifdefined\pdfstringdefPreHook
% This branch is executed for PDF bookmarks and the Table of Contents.
\texorpdfstring{R\textsuperscript{2}C}{R²C}%
\else
% This branch is executed for the main document text.
\textsc{R\textsuperscript{2}C}%
\fi
\xspace
}


\DeclareRobustCommand{\lool}{
\ifdefined\pdfstringdefPreHook
% This branch is executed for PDF bookmarks and the Table of Contents.
\texorpdfstring{LOOL}{LOOL}%
\else
% This branch is executed for the main document text.
\textsc{LOOL}%
\fi
\xspace
}


\newlist{problems}{enumerate}{1} % also creates a counter called 'problemsi'
\setlist[problems,1]{label=\Alph*}
\crefname{problemsi}{Problem}{Problems}

\newlist{configurations}{enumerate}{1}
\setlist[configurations,1]{
label=\textbf{\Alph*)}      % Bold + uppercase letter + closing parenthesis
ref=\Alph*,                  % For cleveref referencing (gives "A", "B", etc.)
}
\newcommand{\configuration}[1]{
\item \textbf{#1}
}
\crefname{configurationsi}{Configuration}{Configurations}
% see https://github.com/matplotlib/matplotlib/issues/27907
\providecommand\mathdefault[1]{#1}


\newcounter{hypothesiscounter}
\crefname{hypothesiscounter}{Hypothesis}{Hypotheses}
\Crefname{hypothesiscounter}{Hypothesis}{Hypotheses}
\makeatletter
\crefformat{hypothesiscounter}{#2#1#3}
\makeatother


% 1. Define a new list environment based on 'description'
\newlist{hypotheses}{description}{1}

% 2. Configure the new list
\setlist[hypotheses]{
font=\bfseries,   % Makes the label text bold
leftmargin=1.5em, % Adjust indentation as needed
labelwidth=!,    % Let the label width be determined by the longest label
itemsep=2pt,       % Space between items
style=nextline
}

\makeatletter
\newcommand{\hypothesis}[2][]{%
% Step the counter. This sets \@currentlabel to the number.
\refstepcounter{hypothesiscounter}%
% Check if an optional label key was provided.
\if
\relax\detokenize{#1}\relax\else
% Save the original \@currentlabel (the number).
\let\original@currentlabel\@currentlabel
% Redefine \@currentlabel to be your custom text for the label.
\protected@edef\@currentlabel{\unexpanded{#2}}%
% Write the label using the new, custom \@currentlabel.
\label{#1}%
% IMPORTANT: Restore the original \@currentlabel.
\let\@currentlabel\original@currentlabel
\fi
% Now, display the item text in the list.
\item[\textbf{Hypothesis #2}]%
}
\makeatother




\newcommand*{\threeapprox}{\approx}

\newcommand{\speccpu}{SPEC CPU 2017\xspace}



\crefname{codelisting}{Listing}{Listings}
\Crefname{codelisting}{Listing}{Listings}

\crefname{enumfigure}{List}{Lists}
\Crefname{enumfigure}{List}{Lists}
\crefname{step}{Step}{Step}
\Crefname{step}{Steps}{Steps}

\DeclareFloatingEnvironment[placement={ht},name=Listing]{codelisting}
\DeclareFloatingEnvironment[placement={!ht},name=List]{enumfigure}

\newlist{inline}{enumerate*}{1}
\setlist[inline]{label=(\roman*)}
\crefname{inlinei}{item}{items}

\newlist{inlinenum}{enumerate*}{1}
\setlist[inlinenum]{label=(\arabic*)}
\crefname{inlinenumi}{item}{items}

\newlist{inlinealpha}{enumerate*}{1}
\setlist[inlinealpha]{label=(\alph*), itemjoin={,\ }, itemjoin*={ and\ }}
\crefname{inlinealphai}{item}{items}

\newlist{inlinecircle}{enumerate*}{1}
\setlist[inlinecircle]{label=\protect\Circled{\arabic*}, itemjoin={,\ }, itemjoin*={ and\ }}
\crefname{inlinecirclei}{item}{items}

\newtcolorbox{infobox}{
colback=black!5,      % Background color (5% black, 95% white)
colframe=black,       % Frame color
boxrule=1.2pt,        % Frame thickness
%arc=3mm,              % Radius of the rounded corners
boxsep=3pt,           % Padding between text and box
left=8pt,            % Specific padding on the left
right=8pt,           % Specific padding on the right
top=8pt,              % Specific padding on the top
bottom=8pt,           % Specific padding on the bottom
%    breakable,            % Allows the box to break across pages
}

\numberwithin{figure}{section}
\numberwithin{listing}{section}

% Inline helper: draw a tiny boxed number and mark it with a global node name.
\newcommand{\numtag}[2]{
\tikz[remember picture,baseline=(#1.base)]%
\node[inner sep=1pt,outer sep=0pt,draw,rounded corners=2pt,%
font=\footnotesize] (#1) {#2};
}

\setminted{fontsize=\footnotesize}

\usepackage{pdflscape}

\usepackage{rotating}
