\section{Discussion}\label{s:discussion}

\newcommand{\rpairvsmaxcorrelation}{\ensuremath{0.9505016978980805}}
\newcommand{\regressionr}{\ensuremath{0.8841179762377677}}
\newcommand{\pmimean}{\ensuremath{0.11721308111143029}}
\newcommand{\experimentrankcorrelation}{\ensuremath{0.9824175824175824}}


\subsection{Coverage vs. No Coverage}
The main advantage of having no coverage at all is the higher execution speed, as evidenced by the data in \cref{tbl:coverage-overhead}.
On the other hand, coverage feedback enables guidance through feedback-directed input mutation.
\cref{fig:cumulative-coverage} shows that \propername{random-params}, \ie without feedback but with random generation parameters, achieves the highest cumulative code coverage of all configurations.
This advantage in code coverage is not, however, reflected in the achieved rare optimization scores (see \cref{fig:rare-opts,fig:rare-pairs}).
\propername{Random-params} ranks below all of the feedback-driven configurations based on the genetic algorithm.
One possible explanation is that even if \propername{random-params} discovers new code coverage, the lack of feedback means the configuration cannot build on these gains systematically.
Unlike rare optimization scores, cumulative code coverage does not reflect how consistently a configuration triggers edges.
A single lucky test can boost the cumulative coverage of an entire run.
Although \propername{random-params} is more stable in terms of triggering rare optimizations, its best run is only as good as the weakest quartile of the feedback-driven runs.
\propername{Static-params} on the other hand achieves less code coverage than the feedback-driven configurations, but reaches high rare optimization scores.
The high scores are likely the result of the manual tuning that went into the \propername{static-params} configuration over the last years, with the explicit goal of triggering interesting optimizations.

Despite the lack of feedback, both \propername{random-params} and \propername{static-params} achieve better results than \aflpp-based configurations.
We attribute the improvement of the unguided configurations over \aflpp to the more structural issues with \aflpp, which we discuss in \cref{ss:aflpp-vs-genetic}.

\begin{infobox}
    \textbf{\ref{hpt:feedback-vs-nofeedback}}\hspace{0.8em}\ding{51}
    While coverage feedback does not lead to higher cumulative code coverage compared to \propername{random-params}, the feedback enables higher optimization rareness scores.
    Configurations based on the genetic algorithm achieve higher rareness scores than \propername{random-params} and also slightly higher scores than \propername{static-params}.

\end{infobox}

\subsection{\aflpp vs. Genetic Parameter Mutation}\label{ss:aflpp-vs-genetic}
\begin{table}[!t]
    \small
    \caption{The number of identical test runs and how many tests were successfully generated on the first try or had to be retried with reduced generation parameters.}
    \label{tbl:afl-overview}
    \begin{tabular}{>{\nextrow}E C C C C C C C C C}
        \toprule
        \multicolumn{4}{c}{} & \multicolumn{6}{c}{Attempts} \\
        \cmidrule(lr){5-10}
        {Experiment} & {Total} & {Skipped} & {Identical} & {First} & {Once} & {Second} & {Third} & {Fourth} & {Fifth} \\
        \midrule
        \fileInput{generated/lool/afl-test-overview}
        \bottomrule
    \end{tabular}

\end{table}


The cumulative coverage in \cref{fig:cumulative-coverage} indicates that our custom genetic algorithm outperforms the \aflpp and \propername{static-params} configurations in terms of code coverage.
The only exception is \propername{genetic-code}, which achieves less code coverage than the \propername{static-params} and the \propername{random-params} configuration.
We attribute this difference to the significantly reduced test throughput of \propername{genetic-code}, caused by the code coverage instrumentation.
\cref{fig:rare-opts,fig:rare-pairs} show that \propername{genetic-code} nonetheless triggers a similar amount of rare optimizations and pairs.

Except for the \propername{afl-contextblind-only-params} configuration, all \aflpp-based configurations achieve less coverage than even the \propername{static-params} configuration.
One reason for \aflpp's poor performance lies in the translation of \aflpp generated input files to generated test cases.
We found that over stretches of dozens of generated tests the \aflpp configurations generate either \emph{identical} tests or \emph{small tests}, which are the result of re-generation attempts.
The number of identical tests and re-generation attempts are shown in \cref{tbl:afl-overview}.

Identical tests happen when \aflpp mutates bits in the input file that do not change the code generator's behavior.
Although \aflpp eventually learns that mutating these bits is useless, the feedback loop in the GraalVM context is slow compared to traditional \aflpp targets such as image libraries.
On average, a generated test case takes 9--10s, which means that \aflpp often wastes several minutes trying to mutate meaningless bits.
In the variants that mutate only generation parameters (see \cref{sss:afl-hyperonly}), the generator reads a short, fixed byte sequence from the \aflpp generated input, making it easier for \aflpp to detect which bits have an effect.
Nonetheless, even though we started with an ideally sized seed file, \aflpp eventually tries different sizes as well and wastes time with meaningless mutations.
Another source of identical tests are \aflpp calibration runs, which become particularly expensive with long-running tests.
The problem is compounded by the fact that neither the coverage feedback nor the execution times are fully stable, due to the JVM's profiling and class-loading heuristics.
\cref{tbl:afl-overview} shows the number of identical tests summed over all runs of an experiment for each of the \aflpp configurations.

Small tests happen when \aflpp generates input files that are too small for the generator to generate valid test cases.
Again, this is more likely to happen in the fully parametric case because there, earlier bits in the file can change how many more random bits are needed later on.
This variable length requirement makes it difficult for \aflpp to infer useful file sizes.
In cases where there is too little randomness for the generator's random decisions, the generator tries to size down the generated test case by adapting structural parameters (see \cref{sss:afl-fully-parametric}).
We found that in the fully parametric variant (see \cref{sss:afl-fully-parametric}) the generator reduced the parameters of almost all tests at least once.
\cref{tbl:afl-overview} shows the number of tests with reduced generation parameters for each configuration.
As expected, this effect is mitigated in the variant that reads only generation parameters from the seed (see \cref{sss:afl-hyperonly}).
In this variant, the generator fails only if the seed is even too short to read generation parameters, which happened in about 5\% of the generated tests.
The \enquote{Skipped} column in \cref{tbl:test-overview} shows the number of tests that had to be skipped due to such an unusable seed file.

The difference between \propername{afl-contextblind-only-params} and the other \aflpp based configurations can be explained by looking at \aflpp's statistics.
Apart from the slowdown caused by code coverage, the primary slowdown for \aflpp configurations is the calibration and trimming time.
The code coverage and context-aware \aflpp configurations spent only between 10\% and 20\% of the total runtime actually fuzzing, the rest of the time was calibration and trimming.
Paradoxically, \propername{afl-contextblind} and \propername{afl-contextblind-only-params} are less affected by trimming because \aflpp sees their coverage as \emph{less} stable.
The mapping of global optimization counters to \aflpp's coverage bitmap occupies only about 80 bytes, leading to large relative changes if even only one counter changes.
The code coverage and context-aware coverage types occupy larger parts of the bitmap and lead to higher stability.
Unstable seeds are not further calibrated, thus avoiding the enormous overhead caused by calibrating slow running seeds.
This difference is reflected in the low calibration times of 2\% and 4\% of the total runtime in \propername{afl-globalcounters} and \propername{afl-globalcounters-hyperonly}.
\propername{afl-globalcounter} loses this advantage again, however, by spending 90\% of the total runtime in trimming.

Configurations using the genetic algorithm consistently trigger more rare optimizations and optimization pairs than their \aflpp counterparts (see \cref{fig:rare-opts,fig:rare-pairs}).
For example, \propername{genetic-code} achieves a higher rareness score than \propername{afl-code-only-params} and \propername{afl-code}.
The domain-specific mutation has the advantage of knowing the types of the mutated parameters and is thus able to mutate them in small incremental steps.

Some of the issues described above could be mitigated with \aflpp options, but nonetheless \aflpp's learning process through rapid iteration is hampered by long running tests.
Overall, \aflpp's heuristics are tuned for fast iterations.

\begin{infobox}
    \textbf{\ref{hpt:domain-specific-mutation}}\hspace{0.8em}\ding{51}

    Domain-specific mutation achieves higher cumulative code coverage compared to general-purpose parametric mutation.
    In addition, domain-specific mutation achieves higher rareness scores for both optimizations and optimization pairs.
\end{infobox}

\subsection{Code Coverage vs. Optimization-Log Coverage}
\cref{tbl:coverage-overhead} shows that coverage feedback based on the optimization log (either global counters or per-method counters) has a significantly lower overhead than code coverage.
While our code coverage implementation is not fully optimized for speed, a considerable overhead, in particular for CPU-intensive code such as compilation, is likely to remain.
\begin{infobox}
    \textbf{\ref{hpt:coverage-overhead}}\hspace{0.8em}\ding{51}

    Coverage feedback based on optimization-log counters has a lower overhead than code coverage.
    Optimization-log counter feedback incurs a 7--12\% throughput decrease, whereas code coverage incurs a 71\% throughput decrease.
\end{infobox}
At the same time, \cref{fig:cumulative-coverage} shows that configurations based on code coverage consistently achieve less overall code coverage than configurations based on the optimization log, regardless of the mutation strategy.
For example, \propername{genetic-code} achieves less code coverage than all other configurations using the genetic algorithm and any form of optimization-log coverage.
\cref{fig:rare-opts,fig:rare-pairs} show that code coverage feedback triggers fewer rare optimizations and optimization pairs.
We suspect that two factors contribute to the worse performance of code coverage.
First, code coverage severely constrains the test throughput and fewer tests lead to less coverage overall.
Second, code coverage does not differentiate between \enquote{interesting} code paths and \enquote{uninteresting} code paths.
This loss of context can lead the fuzzer down a route where it tries to, \eg, increase the coverage of some library method called during compilation.

\subsection{Context-Aware Coverage vs. Context-Blind Coverage}
\begin{table}[!t]

    \caption{Rankings of experiments by optimization rareness score or optimization pair rareness score.}
    \label{tbl:contributor-crossover}

    \begin{tabular}{>{\nextrow}EOOC}
        \toprule
        {Experiment} & {\makecell{Optimization\\ Score}} & {\makecell{Optimization\\ Pair Score}} & {\makecell{Optimization\\ Pair Rank}} \\
        \midrule
        \fileInput{generated/lool/optimization-scores}
        \bottomrule
    \end{tabular}

\end{table}

Cumulative code coverage (\cref{fig:cumulative-coverage}) and rareness scores (\cref{fig:rare-opts,fig:rare-pairs}) suggest that context-aware coverage is less beneficial than context-blind coverage.
Surprisingly, the less fine-grained context-blind coverage performs slightly better in all three metrics than the more granular context-aware coverage.
We note, however, that the results of context-blind and context-aware coverage are within the variance range of one another.

The rankings of experiments based on their optimization uniqueness and optimization pair rareness scores mostly agree.
Differences in ranking are caused by relatively minor differences in rareness score.
For example, with both metrics the \propername{genetic-contextblind} variants occupy the first two places.
This similar ranking between the two coverage metrics is reflected by a Spearman rank correlation of $\experimentrankcorrelation$, calling into question the usability of the more complex, context-aware metric.

A possible explanation is that optimizations are largely independent of one another, meaning that $p\left((O_1, O_2)\right) \approx p(O_1) * p(O_2)$.
We can test this assumption by computing the Pointwise Mutual Information
$$
PMI(O_1, O_2) = \log\left(\frac{p((O_1, O_2))}{p(O_1) * p(O_2)} \right)
$$
based on the empirical probabilities for each pair.
Optimization pairs with a high/low $PMI$ are positively/negatively dependent, whereas a $PMI$ around 0 suggests that the two optimizations occur independently of each other.
In our optimization pair dataset, a mean $PMI$ of \pmimean{} suggests that most optimizations are independent, with a few select exceptions.
For example, a $PMI$ of 8.63 indicates a strong correlation between \propername{VectorMaterialization\_VectorMaterialization} and \propername{VectorSimplification\_NodeNarrowing}, both phases involved in vectorization.

We also performed a regression analysis on rarity scores of pairs with the corresponding rarity scores of the pair elements.
That is, can $R\left((O_1, O_2)\right)$ be approximated with $\alpha + \beta_1 * R(A) + \beta_2 * R(B)$?
We find a coefficient of determination $R^2 \approx \regressionr$, meaning that almost 90\% of the variation in optimization pair rareness can be explained with the individual optimization rareness scores.

A related issue contributing to the overlap of the coverage metrics is that rare optimizations also produce rare optimization pairs, even if they co-occur with relatively common optimizations.
For example, when ranked based on the optimization pair rareness, most experiments have pairs such as \texttt{(X,Canonicalizer\_CanonicalReplacement)} among the top contributing pairs.
In this case \texttt{X} is some rare optimization such as \texttt{LoopTransformations\_LoopPartialUnroll} and \texttt{Canonicalizer\_CanonicalReplacement} is a very common optimization.
The rarity of this pair is dominated by the rare element.
We can test for this effect by looking at the correlation of $R\left((A, B)\right)$ and $\max\left(R(A), R(B)\right)$.
For our dataset, we find a Spearman correlation coefficient of $\rpairvsmaxcorrelation$, indicating a strong correlation between a pair's rareness score and the bigger of its element rareness scores.

The above statistical metrics show that optimizations are mostly independent, with only a few exceptions.
Neither \aflpp nor our genetic algorithm filter for these exceptional, dependent optimizations precisely enough to take advantage of the more fine-grained context-aware coverage.
Similarly, optimizations and pairs follow a power distribution, where rare optimizations dominate both single optimization rareness and pair rareness.
Again, neither \aflpp nor our genetic algorithm manage to extract optimization pairs that are not dominated by one of their elements.

\begin{infobox}
    \textbf{\ref{hpt:contextaware-vs-contextblind}}\hspace{0.8em}\ding{55}

    Context-aware coverage shows no clear advantage over context-blind coverage.
    Both types of coverage achieve cumulative code coverage and rareness scores within the variance range of one another.
\end{infobox}

\subsection{Configuration Variants}
In addition to the configurations listed in \cref{s:evaluation}, we also evaluated variants of the different configurations.
For example, the context-blind configurations with \propername{-ffiif} suffix use our \cref{p:ffiif} algorithm instead of the initial optimization ranking.
Based on our evaluation, there seems to be little difference between these variants in terms of code coverage and triggering rare optimizations.
The advantage of our \cref{p:ffiif} algorithm is that it works for single optimizations and for optimization pairs alike.

Similarly, configurations using either the current generation (suffix \propername{-generation}) or the entire population history (suffix \propername{-memory}) perform similarly.
A possible explanation is that 24-hour runs are too short to explore the state space exhaustively enough for the population-history approach to make a difference.
Considering the population history helps avoid situations where the genetic algorithm circles between a network of distinct configurations.
If such a network is large, the generation-based approach cannot fully explore the entire network and eventually return to an already known configuration.
The population-history approach has the advantage, however, that knowledge about already explored optimizations could be transferred from one fuzzing campaign to the next.
