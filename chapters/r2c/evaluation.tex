\section{Evaluation}\label{sec:r2c:evaluation}
\begin{figure}[t!]
    \centering
    \begingroup
    \pgfplotsset{
        every axis/.append style={
            yticklabel style={font=\propernamedecl\footnotesize},
            execute at begin axis={
                \pgfplotsset{
                    width=\textwidth,
                    height=0.8\linewidth}
            },
        },
    }
    %% Creator: Matplotlib, PGF backend
%%
%% To include the figure in your LaTeX document, write
%%   \input{<filename>.pgf}
%%
%% Make sure the required packages are loaded in your preamble
%%   \usepackage{pgf}
%%
%% Also ensure that all the required font packages are loaded; for instance,
%% the lmodern package is sometimes necessary when using math font.
%%   \usepackage{lmodern}
%%
%% Figures using additional raster images can only be included by \input if
%% they are in the same directory as the main LaTeX file. For loading figures
%% from other directories you can use the `import` package
%%   \usepackage{import}
%%
%% and then include the figures with
%%   \import{<path to file>}{<filename>.pgf}
%%
%% Matplotlib used the following preamble
%%
\begingroup%
\makeatletter%
\begin{pgfpicture}%
\pgfpathrectangle{\pgfpointorigin}{\pgfqpoint{5.616630in}{6.294814in}}%
\pgfusepath{use as bounding box, clip}%
\begin{pgfscope}%
\pgfsetbuttcap%
\pgfsetmiterjoin%
\definecolor{currentfill}{rgb}{1.000000,1.000000,1.000000}%
\pgfsetfillcolor{currentfill}%
\pgfsetlinewidth{0.000000pt}%
\definecolor{currentstroke}{rgb}{1.000000,1.000000,1.000000}%
\pgfsetstrokecolor{currentstroke}%
\pgfsetdash{}{0pt}%
\pgfpathmoveto{\pgfqpoint{0.000000in}{0.000000in}}%
\pgfpathlineto{\pgfqpoint{5.616630in}{0.000000in}}%
\pgfpathlineto{\pgfqpoint{5.616630in}{6.294814in}}%
\pgfpathlineto{\pgfqpoint{0.000000in}{6.294814in}}%
\pgfpathlineto{\pgfqpoint{0.000000in}{0.000000in}}%
\pgfpathclose%
\pgfusepath{fill}%
\end{pgfscope}%
\begin{pgfscope}%
\pgfsetbuttcap%
\pgfsetmiterjoin%
\definecolor{currentfill}{rgb}{1.000000,1.000000,1.000000}%
\pgfsetfillcolor{currentfill}%
\pgfsetlinewidth{0.000000pt}%
\definecolor{currentstroke}{rgb}{0.000000,0.000000,0.000000}%
\pgfsetstrokecolor{currentstroke}%
\pgfsetstrokeopacity{0.000000}%
\pgfsetdash{}{0pt}%
\pgfpathmoveto{\pgfqpoint{1.249408in}{0.585648in}}%
\pgfpathlineto{\pgfqpoint{5.516630in}{0.585648in}}%
\pgfpathlineto{\pgfqpoint{5.516630in}{6.194814in}}%
\pgfpathlineto{\pgfqpoint{1.249408in}{6.194814in}}%
\pgfpathlineto{\pgfqpoint{1.249408in}{0.585648in}}%
\pgfpathclose%
\pgfusepath{fill}%
\end{pgfscope}%
\begin{pgfscope}%
\pgfpathrectangle{\pgfqpoint{1.249408in}{0.585648in}}{\pgfqpoint{4.267222in}{5.609167in}}%
\pgfusepath{clip}%
\pgfsetbuttcap%
\pgfsetroundjoin%
\pgfsetlinewidth{0.803000pt}%
\definecolor{currentstroke}{rgb}{0.862745,0.862745,0.862745}%
\pgfsetstrokecolor{currentstroke}%
\pgfsetdash{{2.960000pt}{1.280000pt}}{0.000000pt}%
\pgfpathmoveto{\pgfqpoint{1.249408in}{0.585648in}}%
\pgfpathlineto{\pgfqpoint{1.249408in}{6.194814in}}%
\pgfusepath{stroke}%
\end{pgfscope}%
\begin{pgfscope}%
\pgfsetbuttcap%
\pgfsetroundjoin%
\definecolor{currentfill}{rgb}{0.000000,0.000000,0.000000}%
\pgfsetfillcolor{currentfill}%
\pgfsetlinewidth{1.003750pt}%
\definecolor{currentstroke}{rgb}{0.000000,0.000000,0.000000}%
\pgfsetstrokecolor{currentstroke}%
\pgfsetdash{}{0pt}%
\pgfsys@defobject{currentmarker}{\pgfqpoint{0.000000in}{-0.066667in}}{\pgfqpoint{0.000000in}{0.000000in}}{%
\pgfpathmoveto{\pgfqpoint{0.000000in}{0.000000in}}%
\pgfpathlineto{\pgfqpoint{0.000000in}{-0.066667in}}%
\pgfusepath{stroke,fill}%
}%
\begin{pgfscope}%
\pgfsys@transformshift{1.249408in}{0.585648in}%
\pgfsys@useobject{currentmarker}{}%
\end{pgfscope}%
\end{pgfscope}%
\begin{pgfscope}%
\definecolor{textcolor}{rgb}{0.000000,0.000000,0.000000}%
\pgfsetstrokecolor{textcolor}%
\pgfsetfillcolor{textcolor}%
\pgftext[x=1.249408in,y=0.470370in,,top]{\color{textcolor}\sffamily\fontsize{12.000000}{14.400000}\selectfont 0}%
\end{pgfscope}%
\begin{pgfscope}%
\pgfpathrectangle{\pgfqpoint{1.249408in}{0.585648in}}{\pgfqpoint{4.267222in}{5.609167in}}%
\pgfusepath{clip}%
\pgfsetbuttcap%
\pgfsetroundjoin%
\pgfsetlinewidth{0.803000pt}%
\definecolor{currentstroke}{rgb}{0.862745,0.862745,0.862745}%
\pgfsetstrokecolor{currentstroke}%
\pgfsetdash{{2.960000pt}{1.280000pt}}{0.000000pt}%
\pgfpathmoveto{\pgfqpoint{2.115533in}{0.585648in}}%
\pgfpathlineto{\pgfqpoint{2.115533in}{6.194814in}}%
\pgfusepath{stroke}%
\end{pgfscope}%
\begin{pgfscope}%
\pgfsetbuttcap%
\pgfsetroundjoin%
\definecolor{currentfill}{rgb}{0.000000,0.000000,0.000000}%
\pgfsetfillcolor{currentfill}%
\pgfsetlinewidth{1.003750pt}%
\definecolor{currentstroke}{rgb}{0.000000,0.000000,0.000000}%
\pgfsetstrokecolor{currentstroke}%
\pgfsetdash{}{0pt}%
\pgfsys@defobject{currentmarker}{\pgfqpoint{0.000000in}{-0.066667in}}{\pgfqpoint{0.000000in}{0.000000in}}{%
\pgfpathmoveto{\pgfqpoint{0.000000in}{0.000000in}}%
\pgfpathlineto{\pgfqpoint{0.000000in}{-0.066667in}}%
\pgfusepath{stroke,fill}%
}%
\begin{pgfscope}%
\pgfsys@transformshift{2.115533in}{0.585648in}%
\pgfsys@useobject{currentmarker}{}%
\end{pgfscope}%
\end{pgfscope}%
\begin{pgfscope}%
\definecolor{textcolor}{rgb}{0.000000,0.000000,0.000000}%
\pgfsetstrokecolor{textcolor}%
\pgfsetfillcolor{textcolor}%
\pgftext[x=2.115533in,y=0.470370in,,top]{\color{textcolor}\sffamily\fontsize{12.000000}{14.400000}\selectfont 5}%
\end{pgfscope}%
\begin{pgfscope}%
\pgfpathrectangle{\pgfqpoint{1.249408in}{0.585648in}}{\pgfqpoint{4.267222in}{5.609167in}}%
\pgfusepath{clip}%
\pgfsetbuttcap%
\pgfsetroundjoin%
\pgfsetlinewidth{0.803000pt}%
\definecolor{currentstroke}{rgb}{0.862745,0.862745,0.862745}%
\pgfsetstrokecolor{currentstroke}%
\pgfsetdash{{2.960000pt}{1.280000pt}}{0.000000pt}%
\pgfpathmoveto{\pgfqpoint{2.981657in}{0.585648in}}%
\pgfpathlineto{\pgfqpoint{2.981657in}{6.194814in}}%
\pgfusepath{stroke}%
\end{pgfscope}%
\begin{pgfscope}%
\pgfsetbuttcap%
\pgfsetroundjoin%
\definecolor{currentfill}{rgb}{0.000000,0.000000,0.000000}%
\pgfsetfillcolor{currentfill}%
\pgfsetlinewidth{1.003750pt}%
\definecolor{currentstroke}{rgb}{0.000000,0.000000,0.000000}%
\pgfsetstrokecolor{currentstroke}%
\pgfsetdash{}{0pt}%
\pgfsys@defobject{currentmarker}{\pgfqpoint{0.000000in}{-0.066667in}}{\pgfqpoint{0.000000in}{0.000000in}}{%
\pgfpathmoveto{\pgfqpoint{0.000000in}{0.000000in}}%
\pgfpathlineto{\pgfqpoint{0.000000in}{-0.066667in}}%
\pgfusepath{stroke,fill}%
}%
\begin{pgfscope}%
\pgfsys@transformshift{2.981657in}{0.585648in}%
\pgfsys@useobject{currentmarker}{}%
\end{pgfscope}%
\end{pgfscope}%
\begin{pgfscope}%
\definecolor{textcolor}{rgb}{0.000000,0.000000,0.000000}%
\pgfsetstrokecolor{textcolor}%
\pgfsetfillcolor{textcolor}%
\pgftext[x=2.981657in,y=0.470370in,,top]{\color{textcolor}\sffamily\fontsize{12.000000}{14.400000}\selectfont 10}%
\end{pgfscope}%
\begin{pgfscope}%
\pgfpathrectangle{\pgfqpoint{1.249408in}{0.585648in}}{\pgfqpoint{4.267222in}{5.609167in}}%
\pgfusepath{clip}%
\pgfsetbuttcap%
\pgfsetroundjoin%
\pgfsetlinewidth{0.803000pt}%
\definecolor{currentstroke}{rgb}{0.862745,0.862745,0.862745}%
\pgfsetstrokecolor{currentstroke}%
\pgfsetdash{{2.960000pt}{1.280000pt}}{0.000000pt}%
\pgfpathmoveto{\pgfqpoint{3.847782in}{0.585648in}}%
\pgfpathlineto{\pgfqpoint{3.847782in}{6.194814in}}%
\pgfusepath{stroke}%
\end{pgfscope}%
\begin{pgfscope}%
\pgfsetbuttcap%
\pgfsetroundjoin%
\definecolor{currentfill}{rgb}{0.000000,0.000000,0.000000}%
\pgfsetfillcolor{currentfill}%
\pgfsetlinewidth{1.003750pt}%
\definecolor{currentstroke}{rgb}{0.000000,0.000000,0.000000}%
\pgfsetstrokecolor{currentstroke}%
\pgfsetdash{}{0pt}%
\pgfsys@defobject{currentmarker}{\pgfqpoint{0.000000in}{-0.066667in}}{\pgfqpoint{0.000000in}{0.000000in}}{%
\pgfpathmoveto{\pgfqpoint{0.000000in}{0.000000in}}%
\pgfpathlineto{\pgfqpoint{0.000000in}{-0.066667in}}%
\pgfusepath{stroke,fill}%
}%
\begin{pgfscope}%
\pgfsys@transformshift{3.847782in}{0.585648in}%
\pgfsys@useobject{currentmarker}{}%
\end{pgfscope}%
\end{pgfscope}%
\begin{pgfscope}%
\definecolor{textcolor}{rgb}{0.000000,0.000000,0.000000}%
\pgfsetstrokecolor{textcolor}%
\pgfsetfillcolor{textcolor}%
\pgftext[x=3.847782in,y=0.470370in,,top]{\color{textcolor}\sffamily\fontsize{12.000000}{14.400000}\selectfont 15}%
\end{pgfscope}%
\begin{pgfscope}%
\pgfpathrectangle{\pgfqpoint{1.249408in}{0.585648in}}{\pgfqpoint{4.267222in}{5.609167in}}%
\pgfusepath{clip}%
\pgfsetbuttcap%
\pgfsetroundjoin%
\pgfsetlinewidth{0.803000pt}%
\definecolor{currentstroke}{rgb}{0.862745,0.862745,0.862745}%
\pgfsetstrokecolor{currentstroke}%
\pgfsetdash{{2.960000pt}{1.280000pt}}{0.000000pt}%
\pgfpathmoveto{\pgfqpoint{4.713906in}{0.585648in}}%
\pgfpathlineto{\pgfqpoint{4.713906in}{6.194814in}}%
\pgfusepath{stroke}%
\end{pgfscope}%
\begin{pgfscope}%
\pgfsetbuttcap%
\pgfsetroundjoin%
\definecolor{currentfill}{rgb}{0.000000,0.000000,0.000000}%
\pgfsetfillcolor{currentfill}%
\pgfsetlinewidth{1.003750pt}%
\definecolor{currentstroke}{rgb}{0.000000,0.000000,0.000000}%
\pgfsetstrokecolor{currentstroke}%
\pgfsetdash{}{0pt}%
\pgfsys@defobject{currentmarker}{\pgfqpoint{0.000000in}{-0.066667in}}{\pgfqpoint{0.000000in}{0.000000in}}{%
\pgfpathmoveto{\pgfqpoint{0.000000in}{0.000000in}}%
\pgfpathlineto{\pgfqpoint{0.000000in}{-0.066667in}}%
\pgfusepath{stroke,fill}%
}%
\begin{pgfscope}%
\pgfsys@transformshift{4.713906in}{0.585648in}%
\pgfsys@useobject{currentmarker}{}%
\end{pgfscope}%
\end{pgfscope}%
\begin{pgfscope}%
\definecolor{textcolor}{rgb}{0.000000,0.000000,0.000000}%
\pgfsetstrokecolor{textcolor}%
\pgfsetfillcolor{textcolor}%
\pgftext[x=4.713906in,y=0.470370in,,top]{\color{textcolor}\sffamily\fontsize{12.000000}{14.400000}\selectfont 20}%
\end{pgfscope}%
\begin{pgfscope}%
\pgfpathrectangle{\pgfqpoint{1.249408in}{0.585648in}}{\pgfqpoint{4.267222in}{5.609167in}}%
\pgfusepath{clip}%
\pgfsetbuttcap%
\pgfsetroundjoin%
\pgfsetlinewidth{0.803000pt}%
\definecolor{currentstroke}{rgb}{0.862745,0.862745,0.862745}%
\pgfsetstrokecolor{currentstroke}%
\pgfsetdash{{0.800000pt}{1.320000pt}}{0.000000pt}%
\pgfpathmoveto{\pgfqpoint{1.422633in}{0.585648in}}%
\pgfpathlineto{\pgfqpoint{1.422633in}{6.194814in}}%
\pgfusepath{stroke}%
\end{pgfscope}%
\begin{pgfscope}%
\pgfsetbuttcap%
\pgfsetroundjoin%
\definecolor{currentfill}{rgb}{0.000000,0.000000,0.000000}%
\pgfsetfillcolor{currentfill}%
\pgfsetlinewidth{0.803000pt}%
\definecolor{currentstroke}{rgb}{0.000000,0.000000,0.000000}%
\pgfsetstrokecolor{currentstroke}%
\pgfsetdash{}{0pt}%
\pgfsys@defobject{currentmarker}{\pgfqpoint{0.000000in}{-0.044444in}}{\pgfqpoint{0.000000in}{0.000000in}}{%
\pgfpathmoveto{\pgfqpoint{0.000000in}{0.000000in}}%
\pgfpathlineto{\pgfqpoint{0.000000in}{-0.044444in}}%
\pgfusepath{stroke,fill}%
}%
\begin{pgfscope}%
\pgfsys@transformshift{1.422633in}{0.585648in}%
\pgfsys@useobject{currentmarker}{}%
\end{pgfscope}%
\end{pgfscope}%
\begin{pgfscope}%
\pgfpathrectangle{\pgfqpoint{1.249408in}{0.585648in}}{\pgfqpoint{4.267222in}{5.609167in}}%
\pgfusepath{clip}%
\pgfsetbuttcap%
\pgfsetroundjoin%
\pgfsetlinewidth{0.803000pt}%
\definecolor{currentstroke}{rgb}{0.862745,0.862745,0.862745}%
\pgfsetstrokecolor{currentstroke}%
\pgfsetdash{{0.800000pt}{1.320000pt}}{0.000000pt}%
\pgfpathmoveto{\pgfqpoint{1.595858in}{0.585648in}}%
\pgfpathlineto{\pgfqpoint{1.595858in}{6.194814in}}%
\pgfusepath{stroke}%
\end{pgfscope}%
\begin{pgfscope}%
\pgfsetbuttcap%
\pgfsetroundjoin%
\definecolor{currentfill}{rgb}{0.000000,0.000000,0.000000}%
\pgfsetfillcolor{currentfill}%
\pgfsetlinewidth{0.803000pt}%
\definecolor{currentstroke}{rgb}{0.000000,0.000000,0.000000}%
\pgfsetstrokecolor{currentstroke}%
\pgfsetdash{}{0pt}%
\pgfsys@defobject{currentmarker}{\pgfqpoint{0.000000in}{-0.044444in}}{\pgfqpoint{0.000000in}{0.000000in}}{%
\pgfpathmoveto{\pgfqpoint{0.000000in}{0.000000in}}%
\pgfpathlineto{\pgfqpoint{0.000000in}{-0.044444in}}%
\pgfusepath{stroke,fill}%
}%
\begin{pgfscope}%
\pgfsys@transformshift{1.595858in}{0.585648in}%
\pgfsys@useobject{currentmarker}{}%
\end{pgfscope}%
\end{pgfscope}%
\begin{pgfscope}%
\pgfpathrectangle{\pgfqpoint{1.249408in}{0.585648in}}{\pgfqpoint{4.267222in}{5.609167in}}%
\pgfusepath{clip}%
\pgfsetbuttcap%
\pgfsetroundjoin%
\pgfsetlinewidth{0.803000pt}%
\definecolor{currentstroke}{rgb}{0.862745,0.862745,0.862745}%
\pgfsetstrokecolor{currentstroke}%
\pgfsetdash{{0.800000pt}{1.320000pt}}{0.000000pt}%
\pgfpathmoveto{\pgfqpoint{1.769083in}{0.585648in}}%
\pgfpathlineto{\pgfqpoint{1.769083in}{6.194814in}}%
\pgfusepath{stroke}%
\end{pgfscope}%
\begin{pgfscope}%
\pgfsetbuttcap%
\pgfsetroundjoin%
\definecolor{currentfill}{rgb}{0.000000,0.000000,0.000000}%
\pgfsetfillcolor{currentfill}%
\pgfsetlinewidth{0.803000pt}%
\definecolor{currentstroke}{rgb}{0.000000,0.000000,0.000000}%
\pgfsetstrokecolor{currentstroke}%
\pgfsetdash{}{0pt}%
\pgfsys@defobject{currentmarker}{\pgfqpoint{0.000000in}{-0.044444in}}{\pgfqpoint{0.000000in}{0.000000in}}{%
\pgfpathmoveto{\pgfqpoint{0.000000in}{0.000000in}}%
\pgfpathlineto{\pgfqpoint{0.000000in}{-0.044444in}}%
\pgfusepath{stroke,fill}%
}%
\begin{pgfscope}%
\pgfsys@transformshift{1.769083in}{0.585648in}%
\pgfsys@useobject{currentmarker}{}%
\end{pgfscope}%
\end{pgfscope}%
\begin{pgfscope}%
\pgfpathrectangle{\pgfqpoint{1.249408in}{0.585648in}}{\pgfqpoint{4.267222in}{5.609167in}}%
\pgfusepath{clip}%
\pgfsetbuttcap%
\pgfsetroundjoin%
\pgfsetlinewidth{0.803000pt}%
\definecolor{currentstroke}{rgb}{0.862745,0.862745,0.862745}%
\pgfsetstrokecolor{currentstroke}%
\pgfsetdash{{0.800000pt}{1.320000pt}}{0.000000pt}%
\pgfpathmoveto{\pgfqpoint{1.942308in}{0.585648in}}%
\pgfpathlineto{\pgfqpoint{1.942308in}{6.194814in}}%
\pgfusepath{stroke}%
\end{pgfscope}%
\begin{pgfscope}%
\pgfsetbuttcap%
\pgfsetroundjoin%
\definecolor{currentfill}{rgb}{0.000000,0.000000,0.000000}%
\pgfsetfillcolor{currentfill}%
\pgfsetlinewidth{0.803000pt}%
\definecolor{currentstroke}{rgb}{0.000000,0.000000,0.000000}%
\pgfsetstrokecolor{currentstroke}%
\pgfsetdash{}{0pt}%
\pgfsys@defobject{currentmarker}{\pgfqpoint{0.000000in}{-0.044444in}}{\pgfqpoint{0.000000in}{0.000000in}}{%
\pgfpathmoveto{\pgfqpoint{0.000000in}{0.000000in}}%
\pgfpathlineto{\pgfqpoint{0.000000in}{-0.044444in}}%
\pgfusepath{stroke,fill}%
}%
\begin{pgfscope}%
\pgfsys@transformshift{1.942308in}{0.585648in}%
\pgfsys@useobject{currentmarker}{}%
\end{pgfscope}%
\end{pgfscope}%
\begin{pgfscope}%
\pgfpathrectangle{\pgfqpoint{1.249408in}{0.585648in}}{\pgfqpoint{4.267222in}{5.609167in}}%
\pgfusepath{clip}%
\pgfsetbuttcap%
\pgfsetroundjoin%
\pgfsetlinewidth{0.803000pt}%
\definecolor{currentstroke}{rgb}{0.862745,0.862745,0.862745}%
\pgfsetstrokecolor{currentstroke}%
\pgfsetdash{{0.800000pt}{1.320000pt}}{0.000000pt}%
\pgfpathmoveto{\pgfqpoint{2.288757in}{0.585648in}}%
\pgfpathlineto{\pgfqpoint{2.288757in}{6.194814in}}%
\pgfusepath{stroke}%
\end{pgfscope}%
\begin{pgfscope}%
\pgfsetbuttcap%
\pgfsetroundjoin%
\definecolor{currentfill}{rgb}{0.000000,0.000000,0.000000}%
\pgfsetfillcolor{currentfill}%
\pgfsetlinewidth{0.803000pt}%
\definecolor{currentstroke}{rgb}{0.000000,0.000000,0.000000}%
\pgfsetstrokecolor{currentstroke}%
\pgfsetdash{}{0pt}%
\pgfsys@defobject{currentmarker}{\pgfqpoint{0.000000in}{-0.044444in}}{\pgfqpoint{0.000000in}{0.000000in}}{%
\pgfpathmoveto{\pgfqpoint{0.000000in}{0.000000in}}%
\pgfpathlineto{\pgfqpoint{0.000000in}{-0.044444in}}%
\pgfusepath{stroke,fill}%
}%
\begin{pgfscope}%
\pgfsys@transformshift{2.288757in}{0.585648in}%
\pgfsys@useobject{currentmarker}{}%
\end{pgfscope}%
\end{pgfscope}%
\begin{pgfscope}%
\pgfpathrectangle{\pgfqpoint{1.249408in}{0.585648in}}{\pgfqpoint{4.267222in}{5.609167in}}%
\pgfusepath{clip}%
\pgfsetbuttcap%
\pgfsetroundjoin%
\pgfsetlinewidth{0.803000pt}%
\definecolor{currentstroke}{rgb}{0.862745,0.862745,0.862745}%
\pgfsetstrokecolor{currentstroke}%
\pgfsetdash{{0.800000pt}{1.320000pt}}{0.000000pt}%
\pgfpathmoveto{\pgfqpoint{2.461982in}{0.585648in}}%
\pgfpathlineto{\pgfqpoint{2.461982in}{6.194814in}}%
\pgfusepath{stroke}%
\end{pgfscope}%
\begin{pgfscope}%
\pgfsetbuttcap%
\pgfsetroundjoin%
\definecolor{currentfill}{rgb}{0.000000,0.000000,0.000000}%
\pgfsetfillcolor{currentfill}%
\pgfsetlinewidth{0.803000pt}%
\definecolor{currentstroke}{rgb}{0.000000,0.000000,0.000000}%
\pgfsetstrokecolor{currentstroke}%
\pgfsetdash{}{0pt}%
\pgfsys@defobject{currentmarker}{\pgfqpoint{0.000000in}{-0.044444in}}{\pgfqpoint{0.000000in}{0.000000in}}{%
\pgfpathmoveto{\pgfqpoint{0.000000in}{0.000000in}}%
\pgfpathlineto{\pgfqpoint{0.000000in}{-0.044444in}}%
\pgfusepath{stroke,fill}%
}%
\begin{pgfscope}%
\pgfsys@transformshift{2.461982in}{0.585648in}%
\pgfsys@useobject{currentmarker}{}%
\end{pgfscope}%
\end{pgfscope}%
\begin{pgfscope}%
\pgfpathrectangle{\pgfqpoint{1.249408in}{0.585648in}}{\pgfqpoint{4.267222in}{5.609167in}}%
\pgfusepath{clip}%
\pgfsetbuttcap%
\pgfsetroundjoin%
\pgfsetlinewidth{0.803000pt}%
\definecolor{currentstroke}{rgb}{0.862745,0.862745,0.862745}%
\pgfsetstrokecolor{currentstroke}%
\pgfsetdash{{0.800000pt}{1.320000pt}}{0.000000pt}%
\pgfpathmoveto{\pgfqpoint{2.635207in}{0.585648in}}%
\pgfpathlineto{\pgfqpoint{2.635207in}{6.194814in}}%
\pgfusepath{stroke}%
\end{pgfscope}%
\begin{pgfscope}%
\pgfsetbuttcap%
\pgfsetroundjoin%
\definecolor{currentfill}{rgb}{0.000000,0.000000,0.000000}%
\pgfsetfillcolor{currentfill}%
\pgfsetlinewidth{0.803000pt}%
\definecolor{currentstroke}{rgb}{0.000000,0.000000,0.000000}%
\pgfsetstrokecolor{currentstroke}%
\pgfsetdash{}{0pt}%
\pgfsys@defobject{currentmarker}{\pgfqpoint{0.000000in}{-0.044444in}}{\pgfqpoint{0.000000in}{0.000000in}}{%
\pgfpathmoveto{\pgfqpoint{0.000000in}{0.000000in}}%
\pgfpathlineto{\pgfqpoint{0.000000in}{-0.044444in}}%
\pgfusepath{stroke,fill}%
}%
\begin{pgfscope}%
\pgfsys@transformshift{2.635207in}{0.585648in}%
\pgfsys@useobject{currentmarker}{}%
\end{pgfscope}%
\end{pgfscope}%
\begin{pgfscope}%
\pgfpathrectangle{\pgfqpoint{1.249408in}{0.585648in}}{\pgfqpoint{4.267222in}{5.609167in}}%
\pgfusepath{clip}%
\pgfsetbuttcap%
\pgfsetroundjoin%
\pgfsetlinewidth{0.803000pt}%
\definecolor{currentstroke}{rgb}{0.862745,0.862745,0.862745}%
\pgfsetstrokecolor{currentstroke}%
\pgfsetdash{{0.800000pt}{1.320000pt}}{0.000000pt}%
\pgfpathmoveto{\pgfqpoint{2.808432in}{0.585648in}}%
\pgfpathlineto{\pgfqpoint{2.808432in}{6.194814in}}%
\pgfusepath{stroke}%
\end{pgfscope}%
\begin{pgfscope}%
\pgfsetbuttcap%
\pgfsetroundjoin%
\definecolor{currentfill}{rgb}{0.000000,0.000000,0.000000}%
\pgfsetfillcolor{currentfill}%
\pgfsetlinewidth{0.803000pt}%
\definecolor{currentstroke}{rgb}{0.000000,0.000000,0.000000}%
\pgfsetstrokecolor{currentstroke}%
\pgfsetdash{}{0pt}%
\pgfsys@defobject{currentmarker}{\pgfqpoint{0.000000in}{-0.044444in}}{\pgfqpoint{0.000000in}{0.000000in}}{%
\pgfpathmoveto{\pgfqpoint{0.000000in}{0.000000in}}%
\pgfpathlineto{\pgfqpoint{0.000000in}{-0.044444in}}%
\pgfusepath{stroke,fill}%
}%
\begin{pgfscope}%
\pgfsys@transformshift{2.808432in}{0.585648in}%
\pgfsys@useobject{currentmarker}{}%
\end{pgfscope}%
\end{pgfscope}%
\begin{pgfscope}%
\pgfpathrectangle{\pgfqpoint{1.249408in}{0.585648in}}{\pgfqpoint{4.267222in}{5.609167in}}%
\pgfusepath{clip}%
\pgfsetbuttcap%
\pgfsetroundjoin%
\pgfsetlinewidth{0.803000pt}%
\definecolor{currentstroke}{rgb}{0.862745,0.862745,0.862745}%
\pgfsetstrokecolor{currentstroke}%
\pgfsetdash{{0.800000pt}{1.320000pt}}{0.000000pt}%
\pgfpathmoveto{\pgfqpoint{3.154882in}{0.585648in}}%
\pgfpathlineto{\pgfqpoint{3.154882in}{6.194814in}}%
\pgfusepath{stroke}%
\end{pgfscope}%
\begin{pgfscope}%
\pgfsetbuttcap%
\pgfsetroundjoin%
\definecolor{currentfill}{rgb}{0.000000,0.000000,0.000000}%
\pgfsetfillcolor{currentfill}%
\pgfsetlinewidth{0.803000pt}%
\definecolor{currentstroke}{rgb}{0.000000,0.000000,0.000000}%
\pgfsetstrokecolor{currentstroke}%
\pgfsetdash{}{0pt}%
\pgfsys@defobject{currentmarker}{\pgfqpoint{0.000000in}{-0.044444in}}{\pgfqpoint{0.000000in}{0.000000in}}{%
\pgfpathmoveto{\pgfqpoint{0.000000in}{0.000000in}}%
\pgfpathlineto{\pgfqpoint{0.000000in}{-0.044444in}}%
\pgfusepath{stroke,fill}%
}%
\begin{pgfscope}%
\pgfsys@transformshift{3.154882in}{0.585648in}%
\pgfsys@useobject{currentmarker}{}%
\end{pgfscope}%
\end{pgfscope}%
\begin{pgfscope}%
\pgfpathrectangle{\pgfqpoint{1.249408in}{0.585648in}}{\pgfqpoint{4.267222in}{5.609167in}}%
\pgfusepath{clip}%
\pgfsetbuttcap%
\pgfsetroundjoin%
\pgfsetlinewidth{0.803000pt}%
\definecolor{currentstroke}{rgb}{0.862745,0.862745,0.862745}%
\pgfsetstrokecolor{currentstroke}%
\pgfsetdash{{0.800000pt}{1.320000pt}}{0.000000pt}%
\pgfpathmoveto{\pgfqpoint{3.328107in}{0.585648in}}%
\pgfpathlineto{\pgfqpoint{3.328107in}{6.194814in}}%
\pgfusepath{stroke}%
\end{pgfscope}%
\begin{pgfscope}%
\pgfsetbuttcap%
\pgfsetroundjoin%
\definecolor{currentfill}{rgb}{0.000000,0.000000,0.000000}%
\pgfsetfillcolor{currentfill}%
\pgfsetlinewidth{0.803000pt}%
\definecolor{currentstroke}{rgb}{0.000000,0.000000,0.000000}%
\pgfsetstrokecolor{currentstroke}%
\pgfsetdash{}{0pt}%
\pgfsys@defobject{currentmarker}{\pgfqpoint{0.000000in}{-0.044444in}}{\pgfqpoint{0.000000in}{0.000000in}}{%
\pgfpathmoveto{\pgfqpoint{0.000000in}{0.000000in}}%
\pgfpathlineto{\pgfqpoint{0.000000in}{-0.044444in}}%
\pgfusepath{stroke,fill}%
}%
\begin{pgfscope}%
\pgfsys@transformshift{3.328107in}{0.585648in}%
\pgfsys@useobject{currentmarker}{}%
\end{pgfscope}%
\end{pgfscope}%
\begin{pgfscope}%
\pgfpathrectangle{\pgfqpoint{1.249408in}{0.585648in}}{\pgfqpoint{4.267222in}{5.609167in}}%
\pgfusepath{clip}%
\pgfsetbuttcap%
\pgfsetroundjoin%
\pgfsetlinewidth{0.803000pt}%
\definecolor{currentstroke}{rgb}{0.862745,0.862745,0.862745}%
\pgfsetstrokecolor{currentstroke}%
\pgfsetdash{{0.800000pt}{1.320000pt}}{0.000000pt}%
\pgfpathmoveto{\pgfqpoint{3.501332in}{0.585648in}}%
\pgfpathlineto{\pgfqpoint{3.501332in}{6.194814in}}%
\pgfusepath{stroke}%
\end{pgfscope}%
\begin{pgfscope}%
\pgfsetbuttcap%
\pgfsetroundjoin%
\definecolor{currentfill}{rgb}{0.000000,0.000000,0.000000}%
\pgfsetfillcolor{currentfill}%
\pgfsetlinewidth{0.803000pt}%
\definecolor{currentstroke}{rgb}{0.000000,0.000000,0.000000}%
\pgfsetstrokecolor{currentstroke}%
\pgfsetdash{}{0pt}%
\pgfsys@defobject{currentmarker}{\pgfqpoint{0.000000in}{-0.044444in}}{\pgfqpoint{0.000000in}{0.000000in}}{%
\pgfpathmoveto{\pgfqpoint{0.000000in}{0.000000in}}%
\pgfpathlineto{\pgfqpoint{0.000000in}{-0.044444in}}%
\pgfusepath{stroke,fill}%
}%
\begin{pgfscope}%
\pgfsys@transformshift{3.501332in}{0.585648in}%
\pgfsys@useobject{currentmarker}{}%
\end{pgfscope}%
\end{pgfscope}%
\begin{pgfscope}%
\pgfpathrectangle{\pgfqpoint{1.249408in}{0.585648in}}{\pgfqpoint{4.267222in}{5.609167in}}%
\pgfusepath{clip}%
\pgfsetbuttcap%
\pgfsetroundjoin%
\pgfsetlinewidth{0.803000pt}%
\definecolor{currentstroke}{rgb}{0.862745,0.862745,0.862745}%
\pgfsetstrokecolor{currentstroke}%
\pgfsetdash{{0.800000pt}{1.320000pt}}{0.000000pt}%
\pgfpathmoveto{\pgfqpoint{3.674557in}{0.585648in}}%
\pgfpathlineto{\pgfqpoint{3.674557in}{6.194814in}}%
\pgfusepath{stroke}%
\end{pgfscope}%
\begin{pgfscope}%
\pgfsetbuttcap%
\pgfsetroundjoin%
\definecolor{currentfill}{rgb}{0.000000,0.000000,0.000000}%
\pgfsetfillcolor{currentfill}%
\pgfsetlinewidth{0.803000pt}%
\definecolor{currentstroke}{rgb}{0.000000,0.000000,0.000000}%
\pgfsetstrokecolor{currentstroke}%
\pgfsetdash{}{0pt}%
\pgfsys@defobject{currentmarker}{\pgfqpoint{0.000000in}{-0.044444in}}{\pgfqpoint{0.000000in}{0.000000in}}{%
\pgfpathmoveto{\pgfqpoint{0.000000in}{0.000000in}}%
\pgfpathlineto{\pgfqpoint{0.000000in}{-0.044444in}}%
\pgfusepath{stroke,fill}%
}%
\begin{pgfscope}%
\pgfsys@transformshift{3.674557in}{0.585648in}%
\pgfsys@useobject{currentmarker}{}%
\end{pgfscope}%
\end{pgfscope}%
\begin{pgfscope}%
\pgfpathrectangle{\pgfqpoint{1.249408in}{0.585648in}}{\pgfqpoint{4.267222in}{5.609167in}}%
\pgfusepath{clip}%
\pgfsetbuttcap%
\pgfsetroundjoin%
\pgfsetlinewidth{0.803000pt}%
\definecolor{currentstroke}{rgb}{0.862745,0.862745,0.862745}%
\pgfsetstrokecolor{currentstroke}%
\pgfsetdash{{0.800000pt}{1.320000pt}}{0.000000pt}%
\pgfpathmoveto{\pgfqpoint{4.021007in}{0.585648in}}%
\pgfpathlineto{\pgfqpoint{4.021007in}{6.194814in}}%
\pgfusepath{stroke}%
\end{pgfscope}%
\begin{pgfscope}%
\pgfsetbuttcap%
\pgfsetroundjoin%
\definecolor{currentfill}{rgb}{0.000000,0.000000,0.000000}%
\pgfsetfillcolor{currentfill}%
\pgfsetlinewidth{0.803000pt}%
\definecolor{currentstroke}{rgb}{0.000000,0.000000,0.000000}%
\pgfsetstrokecolor{currentstroke}%
\pgfsetdash{}{0pt}%
\pgfsys@defobject{currentmarker}{\pgfqpoint{0.000000in}{-0.044444in}}{\pgfqpoint{0.000000in}{0.000000in}}{%
\pgfpathmoveto{\pgfqpoint{0.000000in}{0.000000in}}%
\pgfpathlineto{\pgfqpoint{0.000000in}{-0.044444in}}%
\pgfusepath{stroke,fill}%
}%
\begin{pgfscope}%
\pgfsys@transformshift{4.021007in}{0.585648in}%
\pgfsys@useobject{currentmarker}{}%
\end{pgfscope}%
\end{pgfscope}%
\begin{pgfscope}%
\pgfpathrectangle{\pgfqpoint{1.249408in}{0.585648in}}{\pgfqpoint{4.267222in}{5.609167in}}%
\pgfusepath{clip}%
\pgfsetbuttcap%
\pgfsetroundjoin%
\pgfsetlinewidth{0.803000pt}%
\definecolor{currentstroke}{rgb}{0.862745,0.862745,0.862745}%
\pgfsetstrokecolor{currentstroke}%
\pgfsetdash{{0.800000pt}{1.320000pt}}{0.000000pt}%
\pgfpathmoveto{\pgfqpoint{4.194232in}{0.585648in}}%
\pgfpathlineto{\pgfqpoint{4.194232in}{6.194814in}}%
\pgfusepath{stroke}%
\end{pgfscope}%
\begin{pgfscope}%
\pgfsetbuttcap%
\pgfsetroundjoin%
\definecolor{currentfill}{rgb}{0.000000,0.000000,0.000000}%
\pgfsetfillcolor{currentfill}%
\pgfsetlinewidth{0.803000pt}%
\definecolor{currentstroke}{rgb}{0.000000,0.000000,0.000000}%
\pgfsetstrokecolor{currentstroke}%
\pgfsetdash{}{0pt}%
\pgfsys@defobject{currentmarker}{\pgfqpoint{0.000000in}{-0.044444in}}{\pgfqpoint{0.000000in}{0.000000in}}{%
\pgfpathmoveto{\pgfqpoint{0.000000in}{0.000000in}}%
\pgfpathlineto{\pgfqpoint{0.000000in}{-0.044444in}}%
\pgfusepath{stroke,fill}%
}%
\begin{pgfscope}%
\pgfsys@transformshift{4.194232in}{0.585648in}%
\pgfsys@useobject{currentmarker}{}%
\end{pgfscope}%
\end{pgfscope}%
\begin{pgfscope}%
\pgfpathrectangle{\pgfqpoint{1.249408in}{0.585648in}}{\pgfqpoint{4.267222in}{5.609167in}}%
\pgfusepath{clip}%
\pgfsetbuttcap%
\pgfsetroundjoin%
\pgfsetlinewidth{0.803000pt}%
\definecolor{currentstroke}{rgb}{0.862745,0.862745,0.862745}%
\pgfsetstrokecolor{currentstroke}%
\pgfsetdash{{0.800000pt}{1.320000pt}}{0.000000pt}%
\pgfpathmoveto{\pgfqpoint{4.367457in}{0.585648in}}%
\pgfpathlineto{\pgfqpoint{4.367457in}{6.194814in}}%
\pgfusepath{stroke}%
\end{pgfscope}%
\begin{pgfscope}%
\pgfsetbuttcap%
\pgfsetroundjoin%
\definecolor{currentfill}{rgb}{0.000000,0.000000,0.000000}%
\pgfsetfillcolor{currentfill}%
\pgfsetlinewidth{0.803000pt}%
\definecolor{currentstroke}{rgb}{0.000000,0.000000,0.000000}%
\pgfsetstrokecolor{currentstroke}%
\pgfsetdash{}{0pt}%
\pgfsys@defobject{currentmarker}{\pgfqpoint{0.000000in}{-0.044444in}}{\pgfqpoint{0.000000in}{0.000000in}}{%
\pgfpathmoveto{\pgfqpoint{0.000000in}{0.000000in}}%
\pgfpathlineto{\pgfqpoint{0.000000in}{-0.044444in}}%
\pgfusepath{stroke,fill}%
}%
\begin{pgfscope}%
\pgfsys@transformshift{4.367457in}{0.585648in}%
\pgfsys@useobject{currentmarker}{}%
\end{pgfscope}%
\end{pgfscope}%
\begin{pgfscope}%
\pgfpathrectangle{\pgfqpoint{1.249408in}{0.585648in}}{\pgfqpoint{4.267222in}{5.609167in}}%
\pgfusepath{clip}%
\pgfsetbuttcap%
\pgfsetroundjoin%
\pgfsetlinewidth{0.803000pt}%
\definecolor{currentstroke}{rgb}{0.862745,0.862745,0.862745}%
\pgfsetstrokecolor{currentstroke}%
\pgfsetdash{{0.800000pt}{1.320000pt}}{0.000000pt}%
\pgfpathmoveto{\pgfqpoint{4.540682in}{0.585648in}}%
\pgfpathlineto{\pgfqpoint{4.540682in}{6.194814in}}%
\pgfusepath{stroke}%
\end{pgfscope}%
\begin{pgfscope}%
\pgfsetbuttcap%
\pgfsetroundjoin%
\definecolor{currentfill}{rgb}{0.000000,0.000000,0.000000}%
\pgfsetfillcolor{currentfill}%
\pgfsetlinewidth{0.803000pt}%
\definecolor{currentstroke}{rgb}{0.000000,0.000000,0.000000}%
\pgfsetstrokecolor{currentstroke}%
\pgfsetdash{}{0pt}%
\pgfsys@defobject{currentmarker}{\pgfqpoint{0.000000in}{-0.044444in}}{\pgfqpoint{0.000000in}{0.000000in}}{%
\pgfpathmoveto{\pgfqpoint{0.000000in}{0.000000in}}%
\pgfpathlineto{\pgfqpoint{0.000000in}{-0.044444in}}%
\pgfusepath{stroke,fill}%
}%
\begin{pgfscope}%
\pgfsys@transformshift{4.540682in}{0.585648in}%
\pgfsys@useobject{currentmarker}{}%
\end{pgfscope}%
\end{pgfscope}%
\begin{pgfscope}%
\pgfpathrectangle{\pgfqpoint{1.249408in}{0.585648in}}{\pgfqpoint{4.267222in}{5.609167in}}%
\pgfusepath{clip}%
\pgfsetbuttcap%
\pgfsetroundjoin%
\pgfsetlinewidth{0.803000pt}%
\definecolor{currentstroke}{rgb}{0.862745,0.862745,0.862745}%
\pgfsetstrokecolor{currentstroke}%
\pgfsetdash{{0.800000pt}{1.320000pt}}{0.000000pt}%
\pgfpathmoveto{\pgfqpoint{4.887131in}{0.585648in}}%
\pgfpathlineto{\pgfqpoint{4.887131in}{6.194814in}}%
\pgfusepath{stroke}%
\end{pgfscope}%
\begin{pgfscope}%
\pgfsetbuttcap%
\pgfsetroundjoin%
\definecolor{currentfill}{rgb}{0.000000,0.000000,0.000000}%
\pgfsetfillcolor{currentfill}%
\pgfsetlinewidth{0.803000pt}%
\definecolor{currentstroke}{rgb}{0.000000,0.000000,0.000000}%
\pgfsetstrokecolor{currentstroke}%
\pgfsetdash{}{0pt}%
\pgfsys@defobject{currentmarker}{\pgfqpoint{0.000000in}{-0.044444in}}{\pgfqpoint{0.000000in}{0.000000in}}{%
\pgfpathmoveto{\pgfqpoint{0.000000in}{0.000000in}}%
\pgfpathlineto{\pgfqpoint{0.000000in}{-0.044444in}}%
\pgfusepath{stroke,fill}%
}%
\begin{pgfscope}%
\pgfsys@transformshift{4.887131in}{0.585648in}%
\pgfsys@useobject{currentmarker}{}%
\end{pgfscope}%
\end{pgfscope}%
\begin{pgfscope}%
\pgfpathrectangle{\pgfqpoint{1.249408in}{0.585648in}}{\pgfqpoint{4.267222in}{5.609167in}}%
\pgfusepath{clip}%
\pgfsetbuttcap%
\pgfsetroundjoin%
\pgfsetlinewidth{0.803000pt}%
\definecolor{currentstroke}{rgb}{0.862745,0.862745,0.862745}%
\pgfsetstrokecolor{currentstroke}%
\pgfsetdash{{0.800000pt}{1.320000pt}}{0.000000pt}%
\pgfpathmoveto{\pgfqpoint{5.060356in}{0.585648in}}%
\pgfpathlineto{\pgfqpoint{5.060356in}{6.194814in}}%
\pgfusepath{stroke}%
\end{pgfscope}%
\begin{pgfscope}%
\pgfsetbuttcap%
\pgfsetroundjoin%
\definecolor{currentfill}{rgb}{0.000000,0.000000,0.000000}%
\pgfsetfillcolor{currentfill}%
\pgfsetlinewidth{0.803000pt}%
\definecolor{currentstroke}{rgb}{0.000000,0.000000,0.000000}%
\pgfsetstrokecolor{currentstroke}%
\pgfsetdash{}{0pt}%
\pgfsys@defobject{currentmarker}{\pgfqpoint{0.000000in}{-0.044444in}}{\pgfqpoint{0.000000in}{0.000000in}}{%
\pgfpathmoveto{\pgfqpoint{0.000000in}{0.000000in}}%
\pgfpathlineto{\pgfqpoint{0.000000in}{-0.044444in}}%
\pgfusepath{stroke,fill}%
}%
\begin{pgfscope}%
\pgfsys@transformshift{5.060356in}{0.585648in}%
\pgfsys@useobject{currentmarker}{}%
\end{pgfscope}%
\end{pgfscope}%
\begin{pgfscope}%
\pgfpathrectangle{\pgfqpoint{1.249408in}{0.585648in}}{\pgfqpoint{4.267222in}{5.609167in}}%
\pgfusepath{clip}%
\pgfsetbuttcap%
\pgfsetroundjoin%
\pgfsetlinewidth{0.803000pt}%
\definecolor{currentstroke}{rgb}{0.862745,0.862745,0.862745}%
\pgfsetstrokecolor{currentstroke}%
\pgfsetdash{{0.800000pt}{1.320000pt}}{0.000000pt}%
\pgfpathmoveto{\pgfqpoint{5.233581in}{0.585648in}}%
\pgfpathlineto{\pgfqpoint{5.233581in}{6.194814in}}%
\pgfusepath{stroke}%
\end{pgfscope}%
\begin{pgfscope}%
\pgfsetbuttcap%
\pgfsetroundjoin%
\definecolor{currentfill}{rgb}{0.000000,0.000000,0.000000}%
\pgfsetfillcolor{currentfill}%
\pgfsetlinewidth{0.803000pt}%
\definecolor{currentstroke}{rgb}{0.000000,0.000000,0.000000}%
\pgfsetstrokecolor{currentstroke}%
\pgfsetdash{}{0pt}%
\pgfsys@defobject{currentmarker}{\pgfqpoint{0.000000in}{-0.044444in}}{\pgfqpoint{0.000000in}{0.000000in}}{%
\pgfpathmoveto{\pgfqpoint{0.000000in}{0.000000in}}%
\pgfpathlineto{\pgfqpoint{0.000000in}{-0.044444in}}%
\pgfusepath{stroke,fill}%
}%
\begin{pgfscope}%
\pgfsys@transformshift{5.233581in}{0.585648in}%
\pgfsys@useobject{currentmarker}{}%
\end{pgfscope}%
\end{pgfscope}%
\begin{pgfscope}%
\pgfpathrectangle{\pgfqpoint{1.249408in}{0.585648in}}{\pgfqpoint{4.267222in}{5.609167in}}%
\pgfusepath{clip}%
\pgfsetbuttcap%
\pgfsetroundjoin%
\pgfsetlinewidth{0.803000pt}%
\definecolor{currentstroke}{rgb}{0.862745,0.862745,0.862745}%
\pgfsetstrokecolor{currentstroke}%
\pgfsetdash{{0.800000pt}{1.320000pt}}{0.000000pt}%
\pgfpathmoveto{\pgfqpoint{5.406806in}{0.585648in}}%
\pgfpathlineto{\pgfqpoint{5.406806in}{6.194814in}}%
\pgfusepath{stroke}%
\end{pgfscope}%
\begin{pgfscope}%
\pgfsetbuttcap%
\pgfsetroundjoin%
\definecolor{currentfill}{rgb}{0.000000,0.000000,0.000000}%
\pgfsetfillcolor{currentfill}%
\pgfsetlinewidth{0.803000pt}%
\definecolor{currentstroke}{rgb}{0.000000,0.000000,0.000000}%
\pgfsetstrokecolor{currentstroke}%
\pgfsetdash{}{0pt}%
\pgfsys@defobject{currentmarker}{\pgfqpoint{0.000000in}{-0.044444in}}{\pgfqpoint{0.000000in}{0.000000in}}{%
\pgfpathmoveto{\pgfqpoint{0.000000in}{0.000000in}}%
\pgfpathlineto{\pgfqpoint{0.000000in}{-0.044444in}}%
\pgfusepath{stroke,fill}%
}%
\begin{pgfscope}%
\pgfsys@transformshift{5.406806in}{0.585648in}%
\pgfsys@useobject{currentmarker}{}%
\end{pgfscope}%
\end{pgfscope}%
\begin{pgfscope}%
\definecolor{textcolor}{rgb}{0.000000,0.000000,0.000000}%
\pgfsetstrokecolor{textcolor}%
\pgfsetfillcolor{textcolor}%
\pgftext[x=3.383019in,y=0.266667in,,top]{\color{textcolor}\sffamily\fontsize{13.000000}{15.600000}\selectfont Performance Impact (\%)}%
\end{pgfscope}%
\begin{pgfscope}%
\definecolor{textcolor}{rgb}{0.000000,0.000000,0.000000}%
\pgfsetstrokecolor{textcolor}%
\pgfsetfillcolor{textcolor}%
\pgftext[x=1.134130in,y=5.994487in,right,]{\color{textcolor}\sffamily\fontsize{14.000000}{16.800000}\selectfont perlbench}%
\end{pgfscope}%
\begin{pgfscope}%
\definecolor{textcolor}{rgb}{0.000000,0.000000,0.000000}%
\pgfsetstrokecolor{textcolor}%
\pgfsetfillcolor{textcolor}%
\pgftext[x=1.134130in,y=5.593832in,right,]{\color{textcolor}\sffamily\fontsize{14.000000}{16.800000}\selectfont gcc}%
\end{pgfscope}%
\begin{pgfscope}%
\definecolor{textcolor}{rgb}{0.000000,0.000000,0.000000}%
\pgfsetstrokecolor{textcolor}%
\pgfsetfillcolor{textcolor}%
\pgftext[x=1.134130in,y=5.193177in,right,]{\color{textcolor}\sffamily\fontsize{14.000000}{16.800000}\selectfont mcf}%
\end{pgfscope}%
\begin{pgfscope}%
\definecolor{textcolor}{rgb}{0.000000,0.000000,0.000000}%
\pgfsetstrokecolor{textcolor}%
\pgfsetfillcolor{textcolor}%
\pgftext[x=1.134130in,y=4.792522in,right,]{\color{textcolor}\sffamily\fontsize{14.000000}{16.800000}\selectfont lbm}%
\end{pgfscope}%
\begin{pgfscope}%
\definecolor{textcolor}{rgb}{0.000000,0.000000,0.000000}%
\pgfsetstrokecolor{textcolor}%
\pgfsetfillcolor{textcolor}%
\pgftext[x=1.134130in,y=4.391868in,right,]{\color{textcolor}\sffamily\fontsize{14.000000}{16.800000}\selectfont omnetpp}%
\end{pgfscope}%
\begin{pgfscope}%
\definecolor{textcolor}{rgb}{0.000000,0.000000,0.000000}%
\pgfsetstrokecolor{textcolor}%
\pgfsetfillcolor{textcolor}%
\pgftext[x=1.134130in,y=3.991213in,right,]{\color{textcolor}\sffamily\fontsize{14.000000}{16.800000}\selectfont xalancbmk}%
\end{pgfscope}%
\begin{pgfscope}%
\definecolor{textcolor}{rgb}{0.000000,0.000000,0.000000}%
\pgfsetstrokecolor{textcolor}%
\pgfsetfillcolor{textcolor}%
\pgftext[x=1.134130in,y=3.590558in,right,]{\color{textcolor}\sffamily\fontsize{14.000000}{16.800000}\selectfont x264}%
\end{pgfscope}%
\begin{pgfscope}%
\definecolor{textcolor}{rgb}{0.000000,0.000000,0.000000}%
\pgfsetstrokecolor{textcolor}%
\pgfsetfillcolor{textcolor}%
\pgftext[x=1.134130in,y=3.189903in,right,]{\color{textcolor}\sffamily\fontsize{14.000000}{16.800000}\selectfont deepsjeng}%
\end{pgfscope}%
\begin{pgfscope}%
\definecolor{textcolor}{rgb}{0.000000,0.000000,0.000000}%
\pgfsetstrokecolor{textcolor}%
\pgfsetfillcolor{textcolor}%
\pgftext[x=1.134130in,y=2.789249in,right,]{\color{textcolor}\sffamily\fontsize{14.000000}{16.800000}\selectfont imagick}%
\end{pgfscope}%
\begin{pgfscope}%
\definecolor{textcolor}{rgb}{0.000000,0.000000,0.000000}%
\pgfsetstrokecolor{textcolor}%
\pgfsetfillcolor{textcolor}%
\pgftext[x=1.134130in,y=2.388594in,right,]{\color{textcolor}\sffamily\fontsize{14.000000}{16.800000}\selectfont leela}%
\end{pgfscope}%
\begin{pgfscope}%
\definecolor{textcolor}{rgb}{0.000000,0.000000,0.000000}%
\pgfsetstrokecolor{textcolor}%
\pgfsetfillcolor{textcolor}%
\pgftext[x=1.134130in,y=1.987939in,right,]{\color{textcolor}\sffamily\fontsize{14.000000}{16.800000}\selectfont nab}%
\end{pgfscope}%
\begin{pgfscope}%
\definecolor{textcolor}{rgb}{0.000000,0.000000,0.000000}%
\pgfsetstrokecolor{textcolor}%
\pgfsetfillcolor{textcolor}%
\pgftext[x=1.134130in,y=1.587284in,right,]{\color{textcolor}\sffamily\fontsize{14.000000}{16.800000}\selectfont xz}%
\end{pgfscope}%
\begin{pgfscope}%
\definecolor{textcolor}{rgb}{0.000000,0.000000,0.000000}%
\pgfsetstrokecolor{textcolor}%
\pgfsetfillcolor{textcolor}%
\pgftext[x=1.134130in,y=1.186630in,right,]{\color{textcolor}\sffamily\fontsize{14.000000}{16.800000}\selectfont Geomean int}%
\end{pgfscope}%
\begin{pgfscope}%
\definecolor{textcolor}{rgb}{0.000000,0.000000,0.000000}%
\pgfsetstrokecolor{textcolor}%
\pgfsetfillcolor{textcolor}%
\pgftext[x=1.134130in,y=0.785975in,right,]{\color{textcolor}\sffamily\fontsize{14.000000}{16.800000}\selectfont Geomean all}%
\end{pgfscope}%
\begin{pgfscope}%
\pgfpathrectangle{\pgfqpoint{1.249408in}{0.585648in}}{\pgfqpoint{4.267222in}{5.609167in}}%
\pgfusepath{clip}%
\pgfsetbuttcap%
\pgfsetmiterjoin%
\definecolor{currentfill}{rgb}{0.795098,0.200980,0.206863}%
\pgfsetfillcolor{currentfill}%
\pgfsetlinewidth{0.000000pt}%
\definecolor{currentstroke}{rgb}{0.000000,0.000000,0.000000}%
\pgfsetstrokecolor{currentstroke}%
\pgfsetstrokeopacity{0.000000}%
\pgfsetdash{}{0pt}%
\pgfpathmoveto{\pgfqpoint{1.249408in}{6.154749in}}%
\pgfpathlineto{\pgfqpoint{5.213985in}{6.154749in}}%
\pgfpathlineto{\pgfqpoint{5.213985in}{6.074618in}}%
\pgfpathlineto{\pgfqpoint{1.249408in}{6.074618in}}%
\pgfpathlineto{\pgfqpoint{1.249408in}{6.154749in}}%
\pgfpathclose%
\pgfusepath{fill}%
\end{pgfscope}%
\begin{pgfscope}%
\pgfpathrectangle{\pgfqpoint{1.249408in}{0.585648in}}{\pgfqpoint{4.267222in}{5.609167in}}%
\pgfusepath{clip}%
\pgfsetbuttcap%
\pgfsetmiterjoin%
\definecolor{currentfill}{rgb}{0.795098,0.200980,0.206863}%
\pgfsetfillcolor{currentfill}%
\pgfsetlinewidth{0.000000pt}%
\definecolor{currentstroke}{rgb}{0.000000,0.000000,0.000000}%
\pgfsetstrokecolor{currentstroke}%
\pgfsetstrokeopacity{0.000000}%
\pgfsetdash{}{0pt}%
\pgfpathmoveto{\pgfqpoint{1.249408in}{5.754094in}}%
\pgfpathlineto{\pgfqpoint{3.682361in}{5.754094in}}%
\pgfpathlineto{\pgfqpoint{3.682361in}{5.673963in}}%
\pgfpathlineto{\pgfqpoint{1.249408in}{5.673963in}}%
\pgfpathlineto{\pgfqpoint{1.249408in}{5.754094in}}%
\pgfpathclose%
\pgfusepath{fill}%
\end{pgfscope}%
\begin{pgfscope}%
\pgfpathrectangle{\pgfqpoint{1.249408in}{0.585648in}}{\pgfqpoint{4.267222in}{5.609167in}}%
\pgfusepath{clip}%
\pgfsetbuttcap%
\pgfsetmiterjoin%
\definecolor{currentfill}{rgb}{0.795098,0.200980,0.206863}%
\pgfsetfillcolor{currentfill}%
\pgfsetlinewidth{0.000000pt}%
\definecolor{currentstroke}{rgb}{0.000000,0.000000,0.000000}%
\pgfsetstrokecolor{currentstroke}%
\pgfsetstrokeopacity{0.000000}%
\pgfsetdash{}{0pt}%
\pgfpathmoveto{\pgfqpoint{1.249408in}{5.353439in}}%
\pgfpathlineto{\pgfqpoint{3.326731in}{5.353439in}}%
\pgfpathlineto{\pgfqpoint{3.326731in}{5.273308in}}%
\pgfpathlineto{\pgfqpoint{1.249408in}{5.273308in}}%
\pgfpathlineto{\pgfqpoint{1.249408in}{5.353439in}}%
\pgfpathclose%
\pgfusepath{fill}%
\end{pgfscope}%
\begin{pgfscope}%
\pgfpathrectangle{\pgfqpoint{1.249408in}{0.585648in}}{\pgfqpoint{4.267222in}{5.609167in}}%
\pgfusepath{clip}%
\pgfsetbuttcap%
\pgfsetmiterjoin%
\definecolor{currentfill}{rgb}{0.795098,0.200980,0.206863}%
\pgfsetfillcolor{currentfill}%
\pgfsetlinewidth{0.000000pt}%
\definecolor{currentstroke}{rgb}{0.000000,0.000000,0.000000}%
\pgfsetstrokecolor{currentstroke}%
\pgfsetstrokeopacity{0.000000}%
\pgfsetdash{}{0pt}%
\pgfpathmoveto{\pgfqpoint{1.249408in}{4.952784in}}%
\pgfpathlineto{\pgfqpoint{1.414436in}{4.952784in}}%
\pgfpathlineto{\pgfqpoint{1.414436in}{4.872653in}}%
\pgfpathlineto{\pgfqpoint{1.249408in}{4.872653in}}%
\pgfpathlineto{\pgfqpoint{1.249408in}{4.952784in}}%
\pgfpathclose%
\pgfusepath{fill}%
\end{pgfscope}%
\begin{pgfscope}%
\pgfpathrectangle{\pgfqpoint{1.249408in}{0.585648in}}{\pgfqpoint{4.267222in}{5.609167in}}%
\pgfusepath{clip}%
\pgfsetbuttcap%
\pgfsetmiterjoin%
\definecolor{currentfill}{rgb}{0.795098,0.200980,0.206863}%
\pgfsetfillcolor{currentfill}%
\pgfsetlinewidth{0.000000pt}%
\definecolor{currentstroke}{rgb}{0.000000,0.000000,0.000000}%
\pgfsetstrokecolor{currentstroke}%
\pgfsetstrokeopacity{0.000000}%
\pgfsetdash{}{0pt}%
\pgfpathmoveto{\pgfqpoint{1.249408in}{4.552130in}}%
\pgfpathlineto{\pgfqpoint{4.220186in}{4.552130in}}%
\pgfpathlineto{\pgfqpoint{4.220186in}{4.471999in}}%
\pgfpathlineto{\pgfqpoint{1.249408in}{4.471999in}}%
\pgfpathlineto{\pgfqpoint{1.249408in}{4.552130in}}%
\pgfpathclose%
\pgfusepath{fill}%
\end{pgfscope}%
\begin{pgfscope}%
\pgfpathrectangle{\pgfqpoint{1.249408in}{0.585648in}}{\pgfqpoint{4.267222in}{5.609167in}}%
\pgfusepath{clip}%
\pgfsetbuttcap%
\pgfsetmiterjoin%
\definecolor{currentfill}{rgb}{0.795098,0.200980,0.206863}%
\pgfsetfillcolor{currentfill}%
\pgfsetlinewidth{0.000000pt}%
\definecolor{currentstroke}{rgb}{0.000000,0.000000,0.000000}%
\pgfsetstrokecolor{currentstroke}%
\pgfsetstrokeopacity{0.000000}%
\pgfsetdash{}{0pt}%
\pgfpathmoveto{\pgfqpoint{1.249408in}{4.151475in}}%
\pgfpathlineto{\pgfqpoint{3.537820in}{4.151475in}}%
\pgfpathlineto{\pgfqpoint{3.537820in}{4.071344in}}%
\pgfpathlineto{\pgfqpoint{1.249408in}{4.071344in}}%
\pgfpathlineto{\pgfqpoint{1.249408in}{4.151475in}}%
\pgfpathclose%
\pgfusepath{fill}%
\end{pgfscope}%
\begin{pgfscope}%
\pgfpathrectangle{\pgfqpoint{1.249408in}{0.585648in}}{\pgfqpoint{4.267222in}{5.609167in}}%
\pgfusepath{clip}%
\pgfsetbuttcap%
\pgfsetmiterjoin%
\definecolor{currentfill}{rgb}{0.795098,0.200980,0.206863}%
\pgfsetfillcolor{currentfill}%
\pgfsetlinewidth{0.000000pt}%
\definecolor{currentstroke}{rgb}{0.000000,0.000000,0.000000}%
\pgfsetstrokecolor{currentstroke}%
\pgfsetstrokeopacity{0.000000}%
\pgfsetdash{}{0pt}%
\pgfpathmoveto{\pgfqpoint{1.249408in}{3.750820in}}%
\pgfpathlineto{\pgfqpoint{1.942676in}{3.750820in}}%
\pgfpathlineto{\pgfqpoint{1.942676in}{3.670689in}}%
\pgfpathlineto{\pgfqpoint{1.249408in}{3.670689in}}%
\pgfpathlineto{\pgfqpoint{1.249408in}{3.750820in}}%
\pgfpathclose%
\pgfusepath{fill}%
\end{pgfscope}%
\begin{pgfscope}%
\pgfpathrectangle{\pgfqpoint{1.249408in}{0.585648in}}{\pgfqpoint{4.267222in}{5.609167in}}%
\pgfusepath{clip}%
\pgfsetbuttcap%
\pgfsetmiterjoin%
\definecolor{currentfill}{rgb}{0.795098,0.200980,0.206863}%
\pgfsetfillcolor{currentfill}%
\pgfsetlinewidth{0.000000pt}%
\definecolor{currentstroke}{rgb}{0.000000,0.000000,0.000000}%
\pgfsetstrokecolor{currentstroke}%
\pgfsetstrokeopacity{0.000000}%
\pgfsetdash{}{0pt}%
\pgfpathmoveto{\pgfqpoint{1.249408in}{3.350165in}}%
\pgfpathlineto{\pgfqpoint{2.813863in}{3.350165in}}%
\pgfpathlineto{\pgfqpoint{2.813863in}{3.270034in}}%
\pgfpathlineto{\pgfqpoint{1.249408in}{3.270034in}}%
\pgfpathlineto{\pgfqpoint{1.249408in}{3.350165in}}%
\pgfpathclose%
\pgfusepath{fill}%
\end{pgfscope}%
\begin{pgfscope}%
\pgfpathrectangle{\pgfqpoint{1.249408in}{0.585648in}}{\pgfqpoint{4.267222in}{5.609167in}}%
\pgfusepath{clip}%
\pgfsetbuttcap%
\pgfsetmiterjoin%
\definecolor{currentfill}{rgb}{0.795098,0.200980,0.206863}%
\pgfsetfillcolor{currentfill}%
\pgfsetlinewidth{0.000000pt}%
\definecolor{currentstroke}{rgb}{0.000000,0.000000,0.000000}%
\pgfsetstrokecolor{currentstroke}%
\pgfsetstrokeopacity{0.000000}%
\pgfsetdash{}{0pt}%
\pgfpathmoveto{\pgfqpoint{1.249408in}{2.949511in}}%
\pgfpathlineto{\pgfqpoint{0.646112in}{2.949511in}}%
\pgfpathlineto{\pgfqpoint{0.646112in}{2.869380in}}%
\pgfpathlineto{\pgfqpoint{1.249408in}{2.869380in}}%
\pgfpathlineto{\pgfqpoint{1.249408in}{2.949511in}}%
\pgfpathclose%
\pgfusepath{fill}%
\end{pgfscope}%
\begin{pgfscope}%
\pgfpathrectangle{\pgfqpoint{1.249408in}{0.585648in}}{\pgfqpoint{4.267222in}{5.609167in}}%
\pgfusepath{clip}%
\pgfsetbuttcap%
\pgfsetmiterjoin%
\definecolor{currentfill}{rgb}{0.795098,0.200980,0.206863}%
\pgfsetfillcolor{currentfill}%
\pgfsetlinewidth{0.000000pt}%
\definecolor{currentstroke}{rgb}{0.000000,0.000000,0.000000}%
\pgfsetstrokecolor{currentstroke}%
\pgfsetstrokeopacity{0.000000}%
\pgfsetdash{}{0pt}%
\pgfpathmoveto{\pgfqpoint{1.249408in}{2.548856in}}%
\pgfpathlineto{\pgfqpoint{2.529473in}{2.548856in}}%
\pgfpathlineto{\pgfqpoint{2.529473in}{2.468725in}}%
\pgfpathlineto{\pgfqpoint{1.249408in}{2.468725in}}%
\pgfpathlineto{\pgfqpoint{1.249408in}{2.548856in}}%
\pgfpathclose%
\pgfusepath{fill}%
\end{pgfscope}%
\begin{pgfscope}%
\pgfpathrectangle{\pgfqpoint{1.249408in}{0.585648in}}{\pgfqpoint{4.267222in}{5.609167in}}%
\pgfusepath{clip}%
\pgfsetbuttcap%
\pgfsetmiterjoin%
\definecolor{currentfill}{rgb}{0.795098,0.200980,0.206863}%
\pgfsetfillcolor{currentfill}%
\pgfsetlinewidth{0.000000pt}%
\definecolor{currentstroke}{rgb}{0.000000,0.000000,0.000000}%
\pgfsetstrokecolor{currentstroke}%
\pgfsetstrokeopacity{0.000000}%
\pgfsetdash{}{0pt}%
\pgfpathmoveto{\pgfqpoint{1.249408in}{2.148201in}}%
\pgfpathlineto{\pgfqpoint{-0.838921in}{2.148201in}}%
\pgfpathlineto{\pgfqpoint{-0.838921in}{2.068070in}}%
\pgfpathlineto{\pgfqpoint{1.249408in}{2.068070in}}%
\pgfpathlineto{\pgfqpoint{1.249408in}{2.148201in}}%
\pgfpathclose%
\pgfusepath{fill}%
\end{pgfscope}%
\begin{pgfscope}%
\pgfpathrectangle{\pgfqpoint{1.249408in}{0.585648in}}{\pgfqpoint{4.267222in}{5.609167in}}%
\pgfusepath{clip}%
\pgfsetbuttcap%
\pgfsetmiterjoin%
\definecolor{currentfill}{rgb}{0.795098,0.200980,0.206863}%
\pgfsetfillcolor{currentfill}%
\pgfsetlinewidth{0.000000pt}%
\definecolor{currentstroke}{rgb}{0.000000,0.000000,0.000000}%
\pgfsetstrokecolor{currentstroke}%
\pgfsetstrokeopacity{0.000000}%
\pgfsetdash{}{0pt}%
\pgfpathmoveto{\pgfqpoint{1.249408in}{1.747546in}}%
\pgfpathlineto{\pgfqpoint{1.672386in}{1.747546in}}%
\pgfpathlineto{\pgfqpoint{1.672386in}{1.667415in}}%
\pgfpathlineto{\pgfqpoint{1.249408in}{1.667415in}}%
\pgfpathlineto{\pgfqpoint{1.249408in}{1.747546in}}%
\pgfpathclose%
\pgfusepath{fill}%
\end{pgfscope}%
\begin{pgfscope}%
\pgfpathrectangle{\pgfqpoint{1.249408in}{0.585648in}}{\pgfqpoint{4.267222in}{5.609167in}}%
\pgfusepath{clip}%
\pgfsetbuttcap%
\pgfsetmiterjoin%
\definecolor{currentfill}{rgb}{0.795098,0.200980,0.206863}%
\pgfsetfillcolor{currentfill}%
\pgfsetlinewidth{0.000000pt}%
\definecolor{currentstroke}{rgb}{0.000000,0.000000,0.000000}%
\pgfsetstrokecolor{currentstroke}%
\pgfsetstrokeopacity{0.000000}%
\pgfsetdash{}{0pt}%
\pgfpathmoveto{\pgfqpoint{1.249408in}{1.346892in}}%
\pgfpathlineto{\pgfqpoint{3.187004in}{1.346892in}}%
\pgfpathlineto{\pgfqpoint{3.187004in}{1.266761in}}%
\pgfpathlineto{\pgfqpoint{1.249408in}{1.266761in}}%
\pgfpathlineto{\pgfqpoint{1.249408in}{1.346892in}}%
\pgfpathclose%
\pgfusepath{fill}%
\end{pgfscope}%
\begin{pgfscope}%
\pgfpathrectangle{\pgfqpoint{1.249408in}{0.585648in}}{\pgfqpoint{4.267222in}{5.609167in}}%
\pgfusepath{clip}%
\pgfsetbuttcap%
\pgfsetmiterjoin%
\definecolor{currentfill}{rgb}{0.795098,0.200980,0.206863}%
\pgfsetfillcolor{currentfill}%
\pgfsetlinewidth{0.000000pt}%
\definecolor{currentstroke}{rgb}{0.000000,0.000000,0.000000}%
\pgfsetstrokecolor{currentstroke}%
\pgfsetstrokeopacity{0.000000}%
\pgfsetdash{}{0pt}%
\pgfpathmoveto{\pgfqpoint{1.249408in}{0.946237in}}%
\pgfpathlineto{\pgfqpoint{2.443344in}{0.946237in}}%
\pgfpathlineto{\pgfqpoint{2.443344in}{0.866106in}}%
\pgfpathlineto{\pgfqpoint{1.249408in}{0.866106in}}%
\pgfpathlineto{\pgfqpoint{1.249408in}{0.946237in}}%
\pgfpathclose%
\pgfusepath{fill}%
\end{pgfscope}%
\begin{pgfscope}%
\pgfpathrectangle{\pgfqpoint{1.249408in}{0.585648in}}{\pgfqpoint{4.267222in}{5.609167in}}%
\pgfusepath{clip}%
\pgfsetrectcap%
\pgfsetroundjoin%
\pgfsetlinewidth{2.168100pt}%
\definecolor{currentstroke}{rgb}{0.260000,0.260000,0.260000}%
\pgfsetstrokecolor{currentstroke}%
\pgfsetdash{}{0pt}%
\pgfusepath{stroke}%
\end{pgfscope}%
\begin{pgfscope}%
\pgfpathrectangle{\pgfqpoint{1.249408in}{0.585648in}}{\pgfqpoint{4.267222in}{5.609167in}}%
\pgfusepath{clip}%
\pgfsetrectcap%
\pgfsetroundjoin%
\pgfsetlinewidth{2.168100pt}%
\definecolor{currentstroke}{rgb}{0.260000,0.260000,0.260000}%
\pgfsetstrokecolor{currentstroke}%
\pgfsetdash{}{0pt}%
\pgfusepath{stroke}%
\end{pgfscope}%
\begin{pgfscope}%
\pgfpathrectangle{\pgfqpoint{1.249408in}{0.585648in}}{\pgfqpoint{4.267222in}{5.609167in}}%
\pgfusepath{clip}%
\pgfsetrectcap%
\pgfsetroundjoin%
\pgfsetlinewidth{2.168100pt}%
\definecolor{currentstroke}{rgb}{0.260000,0.260000,0.260000}%
\pgfsetstrokecolor{currentstroke}%
\pgfsetdash{}{0pt}%
\pgfusepath{stroke}%
\end{pgfscope}%
\begin{pgfscope}%
\pgfpathrectangle{\pgfqpoint{1.249408in}{0.585648in}}{\pgfqpoint{4.267222in}{5.609167in}}%
\pgfusepath{clip}%
\pgfsetrectcap%
\pgfsetroundjoin%
\pgfsetlinewidth{2.168100pt}%
\definecolor{currentstroke}{rgb}{0.260000,0.260000,0.260000}%
\pgfsetstrokecolor{currentstroke}%
\pgfsetdash{}{0pt}%
\pgfusepath{stroke}%
\end{pgfscope}%
\begin{pgfscope}%
\pgfpathrectangle{\pgfqpoint{1.249408in}{0.585648in}}{\pgfqpoint{4.267222in}{5.609167in}}%
\pgfusepath{clip}%
\pgfsetrectcap%
\pgfsetroundjoin%
\pgfsetlinewidth{2.168100pt}%
\definecolor{currentstroke}{rgb}{0.260000,0.260000,0.260000}%
\pgfsetstrokecolor{currentstroke}%
\pgfsetdash{}{0pt}%
\pgfusepath{stroke}%
\end{pgfscope}%
\begin{pgfscope}%
\pgfpathrectangle{\pgfqpoint{1.249408in}{0.585648in}}{\pgfqpoint{4.267222in}{5.609167in}}%
\pgfusepath{clip}%
\pgfsetrectcap%
\pgfsetroundjoin%
\pgfsetlinewidth{2.168100pt}%
\definecolor{currentstroke}{rgb}{0.260000,0.260000,0.260000}%
\pgfsetstrokecolor{currentstroke}%
\pgfsetdash{}{0pt}%
\pgfusepath{stroke}%
\end{pgfscope}%
\begin{pgfscope}%
\pgfpathrectangle{\pgfqpoint{1.249408in}{0.585648in}}{\pgfqpoint{4.267222in}{5.609167in}}%
\pgfusepath{clip}%
\pgfsetrectcap%
\pgfsetroundjoin%
\pgfsetlinewidth{2.168100pt}%
\definecolor{currentstroke}{rgb}{0.260000,0.260000,0.260000}%
\pgfsetstrokecolor{currentstroke}%
\pgfsetdash{}{0pt}%
\pgfusepath{stroke}%
\end{pgfscope}%
\begin{pgfscope}%
\pgfpathrectangle{\pgfqpoint{1.249408in}{0.585648in}}{\pgfqpoint{4.267222in}{5.609167in}}%
\pgfusepath{clip}%
\pgfsetrectcap%
\pgfsetroundjoin%
\pgfsetlinewidth{2.168100pt}%
\definecolor{currentstroke}{rgb}{0.260000,0.260000,0.260000}%
\pgfsetstrokecolor{currentstroke}%
\pgfsetdash{}{0pt}%
\pgfusepath{stroke}%
\end{pgfscope}%
\begin{pgfscope}%
\pgfpathrectangle{\pgfqpoint{1.249408in}{0.585648in}}{\pgfqpoint{4.267222in}{5.609167in}}%
\pgfusepath{clip}%
\pgfsetrectcap%
\pgfsetroundjoin%
\pgfsetlinewidth{2.168100pt}%
\definecolor{currentstroke}{rgb}{0.260000,0.260000,0.260000}%
\pgfsetstrokecolor{currentstroke}%
\pgfsetdash{}{0pt}%
\pgfusepath{stroke}%
\end{pgfscope}%
\begin{pgfscope}%
\pgfpathrectangle{\pgfqpoint{1.249408in}{0.585648in}}{\pgfqpoint{4.267222in}{5.609167in}}%
\pgfusepath{clip}%
\pgfsetrectcap%
\pgfsetroundjoin%
\pgfsetlinewidth{2.168100pt}%
\definecolor{currentstroke}{rgb}{0.260000,0.260000,0.260000}%
\pgfsetstrokecolor{currentstroke}%
\pgfsetdash{}{0pt}%
\pgfusepath{stroke}%
\end{pgfscope}%
\begin{pgfscope}%
\pgfpathrectangle{\pgfqpoint{1.249408in}{0.585648in}}{\pgfqpoint{4.267222in}{5.609167in}}%
\pgfusepath{clip}%
\pgfsetrectcap%
\pgfsetroundjoin%
\pgfsetlinewidth{2.168100pt}%
\definecolor{currentstroke}{rgb}{0.260000,0.260000,0.260000}%
\pgfsetstrokecolor{currentstroke}%
\pgfsetdash{}{0pt}%
\pgfusepath{stroke}%
\end{pgfscope}%
\begin{pgfscope}%
\pgfpathrectangle{\pgfqpoint{1.249408in}{0.585648in}}{\pgfqpoint{4.267222in}{5.609167in}}%
\pgfusepath{clip}%
\pgfsetrectcap%
\pgfsetroundjoin%
\pgfsetlinewidth{2.168100pt}%
\definecolor{currentstroke}{rgb}{0.260000,0.260000,0.260000}%
\pgfsetstrokecolor{currentstroke}%
\pgfsetdash{}{0pt}%
\pgfusepath{stroke}%
\end{pgfscope}%
\begin{pgfscope}%
\pgfpathrectangle{\pgfqpoint{1.249408in}{0.585648in}}{\pgfqpoint{4.267222in}{5.609167in}}%
\pgfusepath{clip}%
\pgfsetrectcap%
\pgfsetroundjoin%
\pgfsetlinewidth{2.168100pt}%
\definecolor{currentstroke}{rgb}{0.260000,0.260000,0.260000}%
\pgfsetstrokecolor{currentstroke}%
\pgfsetdash{}{0pt}%
\pgfusepath{stroke}%
\end{pgfscope}%
\begin{pgfscope}%
\pgfpathrectangle{\pgfqpoint{1.249408in}{0.585648in}}{\pgfqpoint{4.267222in}{5.609167in}}%
\pgfusepath{clip}%
\pgfsetrectcap%
\pgfsetroundjoin%
\pgfsetlinewidth{2.168100pt}%
\definecolor{currentstroke}{rgb}{0.260000,0.260000,0.260000}%
\pgfsetstrokecolor{currentstroke}%
\pgfsetdash{}{0pt}%
\pgfusepath{stroke}%
\end{pgfscope}%
\begin{pgfscope}%
\pgfpathrectangle{\pgfqpoint{1.249408in}{0.585648in}}{\pgfqpoint{4.267222in}{5.609167in}}%
\pgfusepath{clip}%
\pgfsetbuttcap%
\pgfsetmiterjoin%
\definecolor{currentfill}{rgb}{0.278922,0.487745,0.658333}%
\pgfsetfillcolor{currentfill}%
\pgfsetlinewidth{0.000000pt}%
\definecolor{currentstroke}{rgb}{0.000000,0.000000,0.000000}%
\pgfsetstrokecolor{currentstroke}%
\pgfsetstrokeopacity{0.000000}%
\pgfsetdash{}{0pt}%
\pgfpathmoveto{\pgfqpoint{1.249408in}{6.074618in}}%
\pgfpathlineto{\pgfqpoint{4.152123in}{6.074618in}}%
\pgfpathlineto{\pgfqpoint{4.152123in}{5.994487in}}%
\pgfpathlineto{\pgfqpoint{1.249408in}{5.994487in}}%
\pgfpathlineto{\pgfqpoint{1.249408in}{6.074618in}}%
\pgfpathclose%
\pgfusepath{fill}%
\end{pgfscope}%
\begin{pgfscope}%
\pgfpathrectangle{\pgfqpoint{1.249408in}{0.585648in}}{\pgfqpoint{4.267222in}{5.609167in}}%
\pgfusepath{clip}%
\pgfsetbuttcap%
\pgfsetmiterjoin%
\definecolor{currentfill}{rgb}{0.278922,0.487745,0.658333}%
\pgfsetfillcolor{currentfill}%
\pgfsetlinewidth{0.000000pt}%
\definecolor{currentstroke}{rgb}{0.000000,0.000000,0.000000}%
\pgfsetstrokecolor{currentstroke}%
\pgfsetstrokeopacity{0.000000}%
\pgfsetdash{}{0pt}%
\pgfpathmoveto{\pgfqpoint{1.249408in}{5.673963in}}%
\pgfpathlineto{\pgfqpoint{3.206022in}{5.673963in}}%
\pgfpathlineto{\pgfqpoint{3.206022in}{5.593832in}}%
\pgfpathlineto{\pgfqpoint{1.249408in}{5.593832in}}%
\pgfpathlineto{\pgfqpoint{1.249408in}{5.673963in}}%
\pgfpathclose%
\pgfusepath{fill}%
\end{pgfscope}%
\begin{pgfscope}%
\pgfpathrectangle{\pgfqpoint{1.249408in}{0.585648in}}{\pgfqpoint{4.267222in}{5.609167in}}%
\pgfusepath{clip}%
\pgfsetbuttcap%
\pgfsetmiterjoin%
\definecolor{currentfill}{rgb}{0.278922,0.487745,0.658333}%
\pgfsetfillcolor{currentfill}%
\pgfsetlinewidth{0.000000pt}%
\definecolor{currentstroke}{rgb}{0.000000,0.000000,0.000000}%
\pgfsetstrokecolor{currentstroke}%
\pgfsetstrokeopacity{0.000000}%
\pgfsetdash{}{0pt}%
\pgfpathmoveto{\pgfqpoint{1.249408in}{5.273308in}}%
\pgfpathlineto{\pgfqpoint{2.957946in}{5.273308in}}%
\pgfpathlineto{\pgfqpoint{2.957946in}{5.193177in}}%
\pgfpathlineto{\pgfqpoint{1.249408in}{5.193177in}}%
\pgfpathlineto{\pgfqpoint{1.249408in}{5.273308in}}%
\pgfpathclose%
\pgfusepath{fill}%
\end{pgfscope}%
\begin{pgfscope}%
\pgfpathrectangle{\pgfqpoint{1.249408in}{0.585648in}}{\pgfqpoint{4.267222in}{5.609167in}}%
\pgfusepath{clip}%
\pgfsetbuttcap%
\pgfsetmiterjoin%
\definecolor{currentfill}{rgb}{0.278922,0.487745,0.658333}%
\pgfsetfillcolor{currentfill}%
\pgfsetlinewidth{0.000000pt}%
\definecolor{currentstroke}{rgb}{0.000000,0.000000,0.000000}%
\pgfsetstrokecolor{currentstroke}%
\pgfsetstrokeopacity{0.000000}%
\pgfsetdash{}{0pt}%
\pgfpathmoveto{\pgfqpoint{1.249408in}{4.872653in}}%
\pgfpathlineto{\pgfqpoint{1.288825in}{4.872653in}}%
\pgfpathlineto{\pgfqpoint{1.288825in}{4.792522in}}%
\pgfpathlineto{\pgfqpoint{1.249408in}{4.792522in}}%
\pgfpathlineto{\pgfqpoint{1.249408in}{4.872653in}}%
\pgfpathclose%
\pgfusepath{fill}%
\end{pgfscope}%
\begin{pgfscope}%
\pgfpathrectangle{\pgfqpoint{1.249408in}{0.585648in}}{\pgfqpoint{4.267222in}{5.609167in}}%
\pgfusepath{clip}%
\pgfsetbuttcap%
\pgfsetmiterjoin%
\definecolor{currentfill}{rgb}{0.278922,0.487745,0.658333}%
\pgfsetfillcolor{currentfill}%
\pgfsetlinewidth{0.000000pt}%
\definecolor{currentstroke}{rgb}{0.000000,0.000000,0.000000}%
\pgfsetstrokecolor{currentstroke}%
\pgfsetstrokeopacity{0.000000}%
\pgfsetdash{}{0pt}%
\pgfpathmoveto{\pgfqpoint{1.249408in}{4.471999in}}%
\pgfpathlineto{\pgfqpoint{3.946214in}{4.471999in}}%
\pgfpathlineto{\pgfqpoint{3.946214in}{4.391868in}}%
\pgfpathlineto{\pgfqpoint{1.249408in}{4.391868in}}%
\pgfpathlineto{\pgfqpoint{1.249408in}{4.471999in}}%
\pgfpathclose%
\pgfusepath{fill}%
\end{pgfscope}%
\begin{pgfscope}%
\pgfpathrectangle{\pgfqpoint{1.249408in}{0.585648in}}{\pgfqpoint{4.267222in}{5.609167in}}%
\pgfusepath{clip}%
\pgfsetbuttcap%
\pgfsetmiterjoin%
\definecolor{currentfill}{rgb}{0.278922,0.487745,0.658333}%
\pgfsetfillcolor{currentfill}%
\pgfsetlinewidth{0.000000pt}%
\definecolor{currentstroke}{rgb}{0.000000,0.000000,0.000000}%
\pgfsetstrokecolor{currentstroke}%
\pgfsetstrokeopacity{0.000000}%
\pgfsetdash{}{0pt}%
\pgfpathmoveto{\pgfqpoint{1.249408in}{4.071344in}}%
\pgfpathlineto{\pgfqpoint{4.130198in}{4.071344in}}%
\pgfpathlineto{\pgfqpoint{4.130198in}{3.991213in}}%
\pgfpathlineto{\pgfqpoint{1.249408in}{3.991213in}}%
\pgfpathlineto{\pgfqpoint{1.249408in}{4.071344in}}%
\pgfpathclose%
\pgfusepath{fill}%
\end{pgfscope}%
\begin{pgfscope}%
\pgfpathrectangle{\pgfqpoint{1.249408in}{0.585648in}}{\pgfqpoint{4.267222in}{5.609167in}}%
\pgfusepath{clip}%
\pgfsetbuttcap%
\pgfsetmiterjoin%
\definecolor{currentfill}{rgb}{0.278922,0.487745,0.658333}%
\pgfsetfillcolor{currentfill}%
\pgfsetlinewidth{0.000000pt}%
\definecolor{currentstroke}{rgb}{0.000000,0.000000,0.000000}%
\pgfsetstrokecolor{currentstroke}%
\pgfsetstrokeopacity{0.000000}%
\pgfsetdash{}{0pt}%
\pgfpathmoveto{\pgfqpoint{1.249408in}{3.670689in}}%
\pgfpathlineto{\pgfqpoint{2.441662in}{3.670689in}}%
\pgfpathlineto{\pgfqpoint{2.441662in}{3.590558in}}%
\pgfpathlineto{\pgfqpoint{1.249408in}{3.590558in}}%
\pgfpathlineto{\pgfqpoint{1.249408in}{3.670689in}}%
\pgfpathclose%
\pgfusepath{fill}%
\end{pgfscope}%
\begin{pgfscope}%
\pgfpathrectangle{\pgfqpoint{1.249408in}{0.585648in}}{\pgfqpoint{4.267222in}{5.609167in}}%
\pgfusepath{clip}%
\pgfsetbuttcap%
\pgfsetmiterjoin%
\definecolor{currentfill}{rgb}{0.278922,0.487745,0.658333}%
\pgfsetfillcolor{currentfill}%
\pgfsetlinewidth{0.000000pt}%
\definecolor{currentstroke}{rgb}{0.000000,0.000000,0.000000}%
\pgfsetstrokecolor{currentstroke}%
\pgfsetstrokeopacity{0.000000}%
\pgfsetdash{}{0pt}%
\pgfpathmoveto{\pgfqpoint{1.249408in}{3.270034in}}%
\pgfpathlineto{\pgfqpoint{2.340860in}{3.270034in}}%
\pgfpathlineto{\pgfqpoint{2.340860in}{3.189903in}}%
\pgfpathlineto{\pgfqpoint{1.249408in}{3.189903in}}%
\pgfpathlineto{\pgfqpoint{1.249408in}{3.270034in}}%
\pgfpathclose%
\pgfusepath{fill}%
\end{pgfscope}%
\begin{pgfscope}%
\pgfpathrectangle{\pgfqpoint{1.249408in}{0.585648in}}{\pgfqpoint{4.267222in}{5.609167in}}%
\pgfusepath{clip}%
\pgfsetbuttcap%
\pgfsetmiterjoin%
\definecolor{currentfill}{rgb}{0.278922,0.487745,0.658333}%
\pgfsetfillcolor{currentfill}%
\pgfsetlinewidth{0.000000pt}%
\definecolor{currentstroke}{rgb}{0.000000,0.000000,0.000000}%
\pgfsetstrokecolor{currentstroke}%
\pgfsetstrokeopacity{0.000000}%
\pgfsetdash{}{0pt}%
\pgfpathmoveto{\pgfqpoint{1.249408in}{2.869380in}}%
\pgfpathlineto{\pgfqpoint{1.306068in}{2.869380in}}%
\pgfpathlineto{\pgfqpoint{1.306068in}{2.789249in}}%
\pgfpathlineto{\pgfqpoint{1.249408in}{2.789249in}}%
\pgfpathlineto{\pgfqpoint{1.249408in}{2.869380in}}%
\pgfpathclose%
\pgfusepath{fill}%
\end{pgfscope}%
\begin{pgfscope}%
\pgfpathrectangle{\pgfqpoint{1.249408in}{0.585648in}}{\pgfqpoint{4.267222in}{5.609167in}}%
\pgfusepath{clip}%
\pgfsetbuttcap%
\pgfsetmiterjoin%
\definecolor{currentfill}{rgb}{0.278922,0.487745,0.658333}%
\pgfsetfillcolor{currentfill}%
\pgfsetlinewidth{0.000000pt}%
\definecolor{currentstroke}{rgb}{0.000000,0.000000,0.000000}%
\pgfsetstrokecolor{currentstroke}%
\pgfsetstrokeopacity{0.000000}%
\pgfsetdash{}{0pt}%
\pgfpathmoveto{\pgfqpoint{1.249408in}{2.468725in}}%
\pgfpathlineto{\pgfqpoint{2.662544in}{2.468725in}}%
\pgfpathlineto{\pgfqpoint{2.662544in}{2.388594in}}%
\pgfpathlineto{\pgfqpoint{1.249408in}{2.388594in}}%
\pgfpathlineto{\pgfqpoint{1.249408in}{2.468725in}}%
\pgfpathclose%
\pgfusepath{fill}%
\end{pgfscope}%
\begin{pgfscope}%
\pgfpathrectangle{\pgfqpoint{1.249408in}{0.585648in}}{\pgfqpoint{4.267222in}{5.609167in}}%
\pgfusepath{clip}%
\pgfsetbuttcap%
\pgfsetmiterjoin%
\definecolor{currentfill}{rgb}{0.278922,0.487745,0.658333}%
\pgfsetfillcolor{currentfill}%
\pgfsetlinewidth{0.000000pt}%
\definecolor{currentstroke}{rgb}{0.000000,0.000000,0.000000}%
\pgfsetstrokecolor{currentstroke}%
\pgfsetstrokeopacity{0.000000}%
\pgfsetdash{}{0pt}%
\pgfpathmoveto{\pgfqpoint{1.249408in}{2.068070in}}%
\pgfpathlineto{\pgfqpoint{1.141214in}{2.068070in}}%
\pgfpathlineto{\pgfqpoint{1.141214in}{1.987939in}}%
\pgfpathlineto{\pgfqpoint{1.249408in}{1.987939in}}%
\pgfpathlineto{\pgfqpoint{1.249408in}{2.068070in}}%
\pgfpathclose%
\pgfusepath{fill}%
\end{pgfscope}%
\begin{pgfscope}%
\pgfpathrectangle{\pgfqpoint{1.249408in}{0.585648in}}{\pgfqpoint{4.267222in}{5.609167in}}%
\pgfusepath{clip}%
\pgfsetbuttcap%
\pgfsetmiterjoin%
\definecolor{currentfill}{rgb}{0.278922,0.487745,0.658333}%
\pgfsetfillcolor{currentfill}%
\pgfsetlinewidth{0.000000pt}%
\definecolor{currentstroke}{rgb}{0.000000,0.000000,0.000000}%
\pgfsetstrokecolor{currentstroke}%
\pgfsetstrokeopacity{0.000000}%
\pgfsetdash{}{0pt}%
\pgfpathmoveto{\pgfqpoint{1.249408in}{1.667415in}}%
\pgfpathlineto{\pgfqpoint{1.183729in}{1.667415in}}%
\pgfpathlineto{\pgfqpoint{1.183729in}{1.587284in}}%
\pgfpathlineto{\pgfqpoint{1.249408in}{1.587284in}}%
\pgfpathlineto{\pgfqpoint{1.249408in}{1.667415in}}%
\pgfpathclose%
\pgfusepath{fill}%
\end{pgfscope}%
\begin{pgfscope}%
\pgfpathrectangle{\pgfqpoint{1.249408in}{0.585648in}}{\pgfqpoint{4.267222in}{5.609167in}}%
\pgfusepath{clip}%
\pgfsetbuttcap%
\pgfsetmiterjoin%
\definecolor{currentfill}{rgb}{0.278922,0.487745,0.658333}%
\pgfsetfillcolor{currentfill}%
\pgfsetlinewidth{0.000000pt}%
\definecolor{currentstroke}{rgb}{0.000000,0.000000,0.000000}%
\pgfsetstrokecolor{currentstroke}%
\pgfsetstrokeopacity{0.000000}%
\pgfsetdash{}{0pt}%
\pgfpathmoveto{\pgfqpoint{1.249408in}{1.266761in}}%
\pgfpathlineto{\pgfqpoint{2.979604in}{1.266761in}}%
\pgfpathlineto{\pgfqpoint{2.979604in}{1.186630in}}%
\pgfpathlineto{\pgfqpoint{1.249408in}{1.186630in}}%
\pgfpathlineto{\pgfqpoint{1.249408in}{1.266761in}}%
\pgfpathclose%
\pgfusepath{fill}%
\end{pgfscope}%
\begin{pgfscope}%
\pgfpathrectangle{\pgfqpoint{1.249408in}{0.585648in}}{\pgfqpoint{4.267222in}{5.609167in}}%
\pgfusepath{clip}%
\pgfsetbuttcap%
\pgfsetmiterjoin%
\definecolor{currentfill}{rgb}{0.278922,0.487745,0.658333}%
\pgfsetfillcolor{currentfill}%
\pgfsetlinewidth{0.000000pt}%
\definecolor{currentstroke}{rgb}{0.000000,0.000000,0.000000}%
\pgfsetstrokecolor{currentstroke}%
\pgfsetstrokeopacity{0.000000}%
\pgfsetdash{}{0pt}%
\pgfpathmoveto{\pgfqpoint{1.249408in}{0.866106in}}%
\pgfpathlineto{\pgfqpoint{2.530365in}{0.866106in}}%
\pgfpathlineto{\pgfqpoint{2.530365in}{0.785975in}}%
\pgfpathlineto{\pgfqpoint{1.249408in}{0.785975in}}%
\pgfpathlineto{\pgfqpoint{1.249408in}{0.866106in}}%
\pgfpathclose%
\pgfusepath{fill}%
\end{pgfscope}%
\begin{pgfscope}%
\pgfpathrectangle{\pgfqpoint{1.249408in}{0.585648in}}{\pgfqpoint{4.267222in}{5.609167in}}%
\pgfusepath{clip}%
\pgfsetrectcap%
\pgfsetroundjoin%
\pgfsetlinewidth{2.168100pt}%
\definecolor{currentstroke}{rgb}{0.260000,0.260000,0.260000}%
\pgfsetstrokecolor{currentstroke}%
\pgfsetdash{}{0pt}%
\pgfusepath{stroke}%
\end{pgfscope}%
\begin{pgfscope}%
\pgfpathrectangle{\pgfqpoint{1.249408in}{0.585648in}}{\pgfqpoint{4.267222in}{5.609167in}}%
\pgfusepath{clip}%
\pgfsetrectcap%
\pgfsetroundjoin%
\pgfsetlinewidth{2.168100pt}%
\definecolor{currentstroke}{rgb}{0.260000,0.260000,0.260000}%
\pgfsetstrokecolor{currentstroke}%
\pgfsetdash{}{0pt}%
\pgfusepath{stroke}%
\end{pgfscope}%
\begin{pgfscope}%
\pgfpathrectangle{\pgfqpoint{1.249408in}{0.585648in}}{\pgfqpoint{4.267222in}{5.609167in}}%
\pgfusepath{clip}%
\pgfsetrectcap%
\pgfsetroundjoin%
\pgfsetlinewidth{2.168100pt}%
\definecolor{currentstroke}{rgb}{0.260000,0.260000,0.260000}%
\pgfsetstrokecolor{currentstroke}%
\pgfsetdash{}{0pt}%
\pgfusepath{stroke}%
\end{pgfscope}%
\begin{pgfscope}%
\pgfpathrectangle{\pgfqpoint{1.249408in}{0.585648in}}{\pgfqpoint{4.267222in}{5.609167in}}%
\pgfusepath{clip}%
\pgfsetrectcap%
\pgfsetroundjoin%
\pgfsetlinewidth{2.168100pt}%
\definecolor{currentstroke}{rgb}{0.260000,0.260000,0.260000}%
\pgfsetstrokecolor{currentstroke}%
\pgfsetdash{}{0pt}%
\pgfusepath{stroke}%
\end{pgfscope}%
\begin{pgfscope}%
\pgfpathrectangle{\pgfqpoint{1.249408in}{0.585648in}}{\pgfqpoint{4.267222in}{5.609167in}}%
\pgfusepath{clip}%
\pgfsetrectcap%
\pgfsetroundjoin%
\pgfsetlinewidth{2.168100pt}%
\definecolor{currentstroke}{rgb}{0.260000,0.260000,0.260000}%
\pgfsetstrokecolor{currentstroke}%
\pgfsetdash{}{0pt}%
\pgfusepath{stroke}%
\end{pgfscope}%
\begin{pgfscope}%
\pgfpathrectangle{\pgfqpoint{1.249408in}{0.585648in}}{\pgfqpoint{4.267222in}{5.609167in}}%
\pgfusepath{clip}%
\pgfsetrectcap%
\pgfsetroundjoin%
\pgfsetlinewidth{2.168100pt}%
\definecolor{currentstroke}{rgb}{0.260000,0.260000,0.260000}%
\pgfsetstrokecolor{currentstroke}%
\pgfsetdash{}{0pt}%
\pgfusepath{stroke}%
\end{pgfscope}%
\begin{pgfscope}%
\pgfpathrectangle{\pgfqpoint{1.249408in}{0.585648in}}{\pgfqpoint{4.267222in}{5.609167in}}%
\pgfusepath{clip}%
\pgfsetrectcap%
\pgfsetroundjoin%
\pgfsetlinewidth{2.168100pt}%
\definecolor{currentstroke}{rgb}{0.260000,0.260000,0.260000}%
\pgfsetstrokecolor{currentstroke}%
\pgfsetdash{}{0pt}%
\pgfusepath{stroke}%
\end{pgfscope}%
\begin{pgfscope}%
\pgfpathrectangle{\pgfqpoint{1.249408in}{0.585648in}}{\pgfqpoint{4.267222in}{5.609167in}}%
\pgfusepath{clip}%
\pgfsetrectcap%
\pgfsetroundjoin%
\pgfsetlinewidth{2.168100pt}%
\definecolor{currentstroke}{rgb}{0.260000,0.260000,0.260000}%
\pgfsetstrokecolor{currentstroke}%
\pgfsetdash{}{0pt}%
\pgfusepath{stroke}%
\end{pgfscope}%
\begin{pgfscope}%
\pgfpathrectangle{\pgfqpoint{1.249408in}{0.585648in}}{\pgfqpoint{4.267222in}{5.609167in}}%
\pgfusepath{clip}%
\pgfsetrectcap%
\pgfsetroundjoin%
\pgfsetlinewidth{2.168100pt}%
\definecolor{currentstroke}{rgb}{0.260000,0.260000,0.260000}%
\pgfsetstrokecolor{currentstroke}%
\pgfsetdash{}{0pt}%
\pgfusepath{stroke}%
\end{pgfscope}%
\begin{pgfscope}%
\pgfpathrectangle{\pgfqpoint{1.249408in}{0.585648in}}{\pgfqpoint{4.267222in}{5.609167in}}%
\pgfusepath{clip}%
\pgfsetrectcap%
\pgfsetroundjoin%
\pgfsetlinewidth{2.168100pt}%
\definecolor{currentstroke}{rgb}{0.260000,0.260000,0.260000}%
\pgfsetstrokecolor{currentstroke}%
\pgfsetdash{}{0pt}%
\pgfusepath{stroke}%
\end{pgfscope}%
\begin{pgfscope}%
\pgfpathrectangle{\pgfqpoint{1.249408in}{0.585648in}}{\pgfqpoint{4.267222in}{5.609167in}}%
\pgfusepath{clip}%
\pgfsetrectcap%
\pgfsetroundjoin%
\pgfsetlinewidth{2.168100pt}%
\definecolor{currentstroke}{rgb}{0.260000,0.260000,0.260000}%
\pgfsetstrokecolor{currentstroke}%
\pgfsetdash{}{0pt}%
\pgfusepath{stroke}%
\end{pgfscope}%
\begin{pgfscope}%
\pgfpathrectangle{\pgfqpoint{1.249408in}{0.585648in}}{\pgfqpoint{4.267222in}{5.609167in}}%
\pgfusepath{clip}%
\pgfsetrectcap%
\pgfsetroundjoin%
\pgfsetlinewidth{2.168100pt}%
\definecolor{currentstroke}{rgb}{0.260000,0.260000,0.260000}%
\pgfsetstrokecolor{currentstroke}%
\pgfsetdash{}{0pt}%
\pgfusepath{stroke}%
\end{pgfscope}%
\begin{pgfscope}%
\pgfpathrectangle{\pgfqpoint{1.249408in}{0.585648in}}{\pgfqpoint{4.267222in}{5.609167in}}%
\pgfusepath{clip}%
\pgfsetrectcap%
\pgfsetroundjoin%
\pgfsetlinewidth{2.168100pt}%
\definecolor{currentstroke}{rgb}{0.260000,0.260000,0.260000}%
\pgfsetstrokecolor{currentstroke}%
\pgfsetdash{}{0pt}%
\pgfusepath{stroke}%
\end{pgfscope}%
\begin{pgfscope}%
\pgfpathrectangle{\pgfqpoint{1.249408in}{0.585648in}}{\pgfqpoint{4.267222in}{5.609167in}}%
\pgfusepath{clip}%
\pgfsetrectcap%
\pgfsetroundjoin%
\pgfsetlinewidth{2.168100pt}%
\definecolor{currentstroke}{rgb}{0.260000,0.260000,0.260000}%
\pgfsetstrokecolor{currentstroke}%
\pgfsetdash{}{0pt}%
\pgfusepath{stroke}%
\end{pgfscope}%
\begin{pgfscope}%
\pgfpathrectangle{\pgfqpoint{1.249408in}{0.585648in}}{\pgfqpoint{4.267222in}{5.609167in}}%
\pgfusepath{clip}%
\pgfsetbuttcap%
\pgfsetmiterjoin%
\definecolor{currentfill}{rgb}{0.348529,0.636765,0.339706}%
\pgfsetfillcolor{currentfill}%
\pgfsetlinewidth{0.000000pt}%
\definecolor{currentstroke}{rgb}{0.000000,0.000000,0.000000}%
\pgfsetstrokecolor{currentstroke}%
\pgfsetstrokeopacity{0.000000}%
\pgfsetdash{}{0pt}%
\pgfpathmoveto{\pgfqpoint{1.249408in}{5.994487in}}%
\pgfpathlineto{\pgfqpoint{4.160245in}{5.994487in}}%
\pgfpathlineto{\pgfqpoint{4.160245in}{5.914356in}}%
\pgfpathlineto{\pgfqpoint{1.249408in}{5.914356in}}%
\pgfpathlineto{\pgfqpoint{1.249408in}{5.994487in}}%
\pgfpathclose%
\pgfusepath{fill}%
\end{pgfscope}%
\begin{pgfscope}%
\pgfpathrectangle{\pgfqpoint{1.249408in}{0.585648in}}{\pgfqpoint{4.267222in}{5.609167in}}%
\pgfusepath{clip}%
\pgfsetbuttcap%
\pgfsetmiterjoin%
\definecolor{currentfill}{rgb}{0.348529,0.636765,0.339706}%
\pgfsetfillcolor{currentfill}%
\pgfsetlinewidth{0.000000pt}%
\definecolor{currentstroke}{rgb}{0.000000,0.000000,0.000000}%
\pgfsetstrokecolor{currentstroke}%
\pgfsetstrokeopacity{0.000000}%
\pgfsetdash{}{0pt}%
\pgfpathmoveto{\pgfqpoint{1.249408in}{5.593832in}}%
\pgfpathlineto{\pgfqpoint{3.268964in}{5.593832in}}%
\pgfpathlineto{\pgfqpoint{3.268964in}{5.513701in}}%
\pgfpathlineto{\pgfqpoint{1.249408in}{5.513701in}}%
\pgfpathlineto{\pgfqpoint{1.249408in}{5.593832in}}%
\pgfpathclose%
\pgfusepath{fill}%
\end{pgfscope}%
\begin{pgfscope}%
\pgfpathrectangle{\pgfqpoint{1.249408in}{0.585648in}}{\pgfqpoint{4.267222in}{5.609167in}}%
\pgfusepath{clip}%
\pgfsetbuttcap%
\pgfsetmiterjoin%
\definecolor{currentfill}{rgb}{0.348529,0.636765,0.339706}%
\pgfsetfillcolor{currentfill}%
\pgfsetlinewidth{0.000000pt}%
\definecolor{currentstroke}{rgb}{0.000000,0.000000,0.000000}%
\pgfsetstrokecolor{currentstroke}%
\pgfsetstrokeopacity{0.000000}%
\pgfsetdash{}{0pt}%
\pgfpathmoveto{\pgfqpoint{1.249408in}{5.193177in}}%
\pgfpathlineto{\pgfqpoint{3.036602in}{5.193177in}}%
\pgfpathlineto{\pgfqpoint{3.036602in}{5.113046in}}%
\pgfpathlineto{\pgfqpoint{1.249408in}{5.113046in}}%
\pgfpathlineto{\pgfqpoint{1.249408in}{5.193177in}}%
\pgfpathclose%
\pgfusepath{fill}%
\end{pgfscope}%
\begin{pgfscope}%
\pgfpathrectangle{\pgfqpoint{1.249408in}{0.585648in}}{\pgfqpoint{4.267222in}{5.609167in}}%
\pgfusepath{clip}%
\pgfsetbuttcap%
\pgfsetmiterjoin%
\definecolor{currentfill}{rgb}{0.348529,0.636765,0.339706}%
\pgfsetfillcolor{currentfill}%
\pgfsetlinewidth{0.000000pt}%
\definecolor{currentstroke}{rgb}{0.000000,0.000000,0.000000}%
\pgfsetstrokecolor{currentstroke}%
\pgfsetstrokeopacity{0.000000}%
\pgfsetdash{}{0pt}%
\pgfpathmoveto{\pgfqpoint{1.249408in}{4.792522in}}%
\pgfpathlineto{\pgfqpoint{1.278462in}{4.792522in}}%
\pgfpathlineto{\pgfqpoint{1.278462in}{4.712392in}}%
\pgfpathlineto{\pgfqpoint{1.249408in}{4.712392in}}%
\pgfpathlineto{\pgfqpoint{1.249408in}{4.792522in}}%
\pgfpathclose%
\pgfusepath{fill}%
\end{pgfscope}%
\begin{pgfscope}%
\pgfpathrectangle{\pgfqpoint{1.249408in}{0.585648in}}{\pgfqpoint{4.267222in}{5.609167in}}%
\pgfusepath{clip}%
\pgfsetbuttcap%
\pgfsetmiterjoin%
\definecolor{currentfill}{rgb}{0.348529,0.636765,0.339706}%
\pgfsetfillcolor{currentfill}%
\pgfsetlinewidth{0.000000pt}%
\definecolor{currentstroke}{rgb}{0.000000,0.000000,0.000000}%
\pgfsetstrokecolor{currentstroke}%
\pgfsetstrokeopacity{0.000000}%
\pgfsetdash{}{0pt}%
\pgfpathmoveto{\pgfqpoint{1.249408in}{4.391868in}}%
\pgfpathlineto{\pgfqpoint{4.089151in}{4.391868in}}%
\pgfpathlineto{\pgfqpoint{4.089151in}{4.311737in}}%
\pgfpathlineto{\pgfqpoint{1.249408in}{4.311737in}}%
\pgfpathlineto{\pgfqpoint{1.249408in}{4.391868in}}%
\pgfpathclose%
\pgfusepath{fill}%
\end{pgfscope}%
\begin{pgfscope}%
\pgfpathrectangle{\pgfqpoint{1.249408in}{0.585648in}}{\pgfqpoint{4.267222in}{5.609167in}}%
\pgfusepath{clip}%
\pgfsetbuttcap%
\pgfsetmiterjoin%
\definecolor{currentfill}{rgb}{0.348529,0.636765,0.339706}%
\pgfsetfillcolor{currentfill}%
\pgfsetlinewidth{0.000000pt}%
\definecolor{currentstroke}{rgb}{0.000000,0.000000,0.000000}%
\pgfsetstrokecolor{currentstroke}%
\pgfsetstrokeopacity{0.000000}%
\pgfsetdash{}{0pt}%
\pgfpathmoveto{\pgfqpoint{1.249408in}{3.991213in}}%
\pgfpathlineto{\pgfqpoint{4.139279in}{3.991213in}}%
\pgfpathlineto{\pgfqpoint{4.139279in}{3.911082in}}%
\pgfpathlineto{\pgfqpoint{1.249408in}{3.911082in}}%
\pgfpathlineto{\pgfqpoint{1.249408in}{3.991213in}}%
\pgfpathclose%
\pgfusepath{fill}%
\end{pgfscope}%
\begin{pgfscope}%
\pgfpathrectangle{\pgfqpoint{1.249408in}{0.585648in}}{\pgfqpoint{4.267222in}{5.609167in}}%
\pgfusepath{clip}%
\pgfsetbuttcap%
\pgfsetmiterjoin%
\definecolor{currentfill}{rgb}{0.348529,0.636765,0.339706}%
\pgfsetfillcolor{currentfill}%
\pgfsetlinewidth{0.000000pt}%
\definecolor{currentstroke}{rgb}{0.000000,0.000000,0.000000}%
\pgfsetstrokecolor{currentstroke}%
\pgfsetstrokeopacity{0.000000}%
\pgfsetdash{}{0pt}%
\pgfpathmoveto{\pgfqpoint{1.249408in}{3.590558in}}%
\pgfpathlineto{\pgfqpoint{2.489353in}{3.590558in}}%
\pgfpathlineto{\pgfqpoint{2.489353in}{3.510427in}}%
\pgfpathlineto{\pgfqpoint{1.249408in}{3.510427in}}%
\pgfpathlineto{\pgfqpoint{1.249408in}{3.590558in}}%
\pgfpathclose%
\pgfusepath{fill}%
\end{pgfscope}%
\begin{pgfscope}%
\pgfpathrectangle{\pgfqpoint{1.249408in}{0.585648in}}{\pgfqpoint{4.267222in}{5.609167in}}%
\pgfusepath{clip}%
\pgfsetbuttcap%
\pgfsetmiterjoin%
\definecolor{currentfill}{rgb}{0.348529,0.636765,0.339706}%
\pgfsetfillcolor{currentfill}%
\pgfsetlinewidth{0.000000pt}%
\definecolor{currentstroke}{rgb}{0.000000,0.000000,0.000000}%
\pgfsetstrokecolor{currentstroke}%
\pgfsetstrokeopacity{0.000000}%
\pgfsetdash{}{0pt}%
\pgfpathmoveto{\pgfqpoint{1.249408in}{3.189903in}}%
\pgfpathlineto{\pgfqpoint{2.520318in}{3.189903in}}%
\pgfpathlineto{\pgfqpoint{2.520318in}{3.109772in}}%
\pgfpathlineto{\pgfqpoint{1.249408in}{3.109772in}}%
\pgfpathlineto{\pgfqpoint{1.249408in}{3.189903in}}%
\pgfpathclose%
\pgfusepath{fill}%
\end{pgfscope}%
\begin{pgfscope}%
\pgfpathrectangle{\pgfqpoint{1.249408in}{0.585648in}}{\pgfqpoint{4.267222in}{5.609167in}}%
\pgfusepath{clip}%
\pgfsetbuttcap%
\pgfsetmiterjoin%
\definecolor{currentfill}{rgb}{0.348529,0.636765,0.339706}%
\pgfsetfillcolor{currentfill}%
\pgfsetlinewidth{0.000000pt}%
\definecolor{currentstroke}{rgb}{0.000000,0.000000,0.000000}%
\pgfsetstrokecolor{currentstroke}%
\pgfsetstrokeopacity{0.000000}%
\pgfsetdash{}{0pt}%
\pgfpathmoveto{\pgfqpoint{1.249408in}{2.789249in}}%
\pgfpathlineto{\pgfqpoint{1.312496in}{2.789249in}}%
\pgfpathlineto{\pgfqpoint{1.312496in}{2.709118in}}%
\pgfpathlineto{\pgfqpoint{1.249408in}{2.709118in}}%
\pgfpathlineto{\pgfqpoint{1.249408in}{2.789249in}}%
\pgfpathclose%
\pgfusepath{fill}%
\end{pgfscope}%
\begin{pgfscope}%
\pgfpathrectangle{\pgfqpoint{1.249408in}{0.585648in}}{\pgfqpoint{4.267222in}{5.609167in}}%
\pgfusepath{clip}%
\pgfsetbuttcap%
\pgfsetmiterjoin%
\definecolor{currentfill}{rgb}{0.348529,0.636765,0.339706}%
\pgfsetfillcolor{currentfill}%
\pgfsetlinewidth{0.000000pt}%
\definecolor{currentstroke}{rgb}{0.000000,0.000000,0.000000}%
\pgfsetstrokecolor{currentstroke}%
\pgfsetstrokeopacity{0.000000}%
\pgfsetdash{}{0pt}%
\pgfpathmoveto{\pgfqpoint{1.249408in}{2.388594in}}%
\pgfpathlineto{\pgfqpoint{2.538331in}{2.388594in}}%
\pgfpathlineto{\pgfqpoint{2.538331in}{2.308463in}}%
\pgfpathlineto{\pgfqpoint{1.249408in}{2.308463in}}%
\pgfpathlineto{\pgfqpoint{1.249408in}{2.388594in}}%
\pgfpathclose%
\pgfusepath{fill}%
\end{pgfscope}%
\begin{pgfscope}%
\pgfpathrectangle{\pgfqpoint{1.249408in}{0.585648in}}{\pgfqpoint{4.267222in}{5.609167in}}%
\pgfusepath{clip}%
\pgfsetbuttcap%
\pgfsetmiterjoin%
\definecolor{currentfill}{rgb}{0.348529,0.636765,0.339706}%
\pgfsetfillcolor{currentfill}%
\pgfsetlinewidth{0.000000pt}%
\definecolor{currentstroke}{rgb}{0.000000,0.000000,0.000000}%
\pgfsetstrokecolor{currentstroke}%
\pgfsetstrokeopacity{0.000000}%
\pgfsetdash{}{0pt}%
\pgfpathmoveto{\pgfqpoint{1.249408in}{1.987939in}}%
\pgfpathlineto{\pgfqpoint{-0.753096in}{1.987939in}}%
\pgfpathlineto{\pgfqpoint{-0.753096in}{1.907808in}}%
\pgfpathlineto{\pgfqpoint{1.249408in}{1.907808in}}%
\pgfpathlineto{\pgfqpoint{1.249408in}{1.987939in}}%
\pgfpathclose%
\pgfusepath{fill}%
\end{pgfscope}%
\begin{pgfscope}%
\pgfpathrectangle{\pgfqpoint{1.249408in}{0.585648in}}{\pgfqpoint{4.267222in}{5.609167in}}%
\pgfusepath{clip}%
\pgfsetbuttcap%
\pgfsetmiterjoin%
\definecolor{currentfill}{rgb}{0.348529,0.636765,0.339706}%
\pgfsetfillcolor{currentfill}%
\pgfsetlinewidth{0.000000pt}%
\definecolor{currentstroke}{rgb}{0.000000,0.000000,0.000000}%
\pgfsetstrokecolor{currentstroke}%
\pgfsetstrokeopacity{0.000000}%
\pgfsetdash{}{0pt}%
\pgfpathmoveto{\pgfqpoint{1.249408in}{1.587284in}}%
\pgfpathlineto{\pgfqpoint{1.346610in}{1.587284in}}%
\pgfpathlineto{\pgfqpoint{1.346610in}{1.507153in}}%
\pgfpathlineto{\pgfqpoint{1.249408in}{1.507153in}}%
\pgfpathlineto{\pgfqpoint{1.249408in}{1.587284in}}%
\pgfpathclose%
\pgfusepath{fill}%
\end{pgfscope}%
\begin{pgfscope}%
\pgfpathrectangle{\pgfqpoint{1.249408in}{0.585648in}}{\pgfqpoint{4.267222in}{5.609167in}}%
\pgfusepath{clip}%
\pgfsetbuttcap%
\pgfsetmiterjoin%
\definecolor{currentfill}{rgb}{0.348529,0.636765,0.339706}%
\pgfsetfillcolor{currentfill}%
\pgfsetlinewidth{0.000000pt}%
\definecolor{currentstroke}{rgb}{0.000000,0.000000,0.000000}%
\pgfsetstrokecolor{currentstroke}%
\pgfsetstrokeopacity{0.000000}%
\pgfsetdash{}{0pt}%
\pgfpathmoveto{\pgfqpoint{1.249408in}{1.186630in}}%
\pgfpathlineto{\pgfqpoint{3.043990in}{1.186630in}}%
\pgfpathlineto{\pgfqpoint{3.043990in}{1.106499in}}%
\pgfpathlineto{\pgfqpoint{1.249408in}{1.106499in}}%
\pgfpathlineto{\pgfqpoint{1.249408in}{1.186630in}}%
\pgfpathclose%
\pgfusepath{fill}%
\end{pgfscope}%
\begin{pgfscope}%
\pgfpathrectangle{\pgfqpoint{1.249408in}{0.585648in}}{\pgfqpoint{4.267222in}{5.609167in}}%
\pgfusepath{clip}%
\pgfsetbuttcap%
\pgfsetmiterjoin%
\definecolor{currentfill}{rgb}{0.348529,0.636765,0.339706}%
\pgfsetfillcolor{currentfill}%
\pgfsetlinewidth{0.000000pt}%
\definecolor{currentstroke}{rgb}{0.000000,0.000000,0.000000}%
\pgfsetstrokecolor{currentstroke}%
\pgfsetstrokeopacity{0.000000}%
\pgfsetdash{}{0pt}%
\pgfpathmoveto{\pgfqpoint{1.249408in}{0.785975in}}%
\pgfpathlineto{\pgfqpoint{2.396831in}{0.785975in}}%
\pgfpathlineto{\pgfqpoint{2.396831in}{0.705844in}}%
\pgfpathlineto{\pgfqpoint{1.249408in}{0.705844in}}%
\pgfpathlineto{\pgfqpoint{1.249408in}{0.785975in}}%
\pgfpathclose%
\pgfusepath{fill}%
\end{pgfscope}%
\begin{pgfscope}%
\pgfpathrectangle{\pgfqpoint{1.249408in}{0.585648in}}{\pgfqpoint{4.267222in}{5.609167in}}%
\pgfusepath{clip}%
\pgfsetrectcap%
\pgfsetroundjoin%
\pgfsetlinewidth{2.168100pt}%
\definecolor{currentstroke}{rgb}{0.260000,0.260000,0.260000}%
\pgfsetstrokecolor{currentstroke}%
\pgfsetdash{}{0pt}%
\pgfusepath{stroke}%
\end{pgfscope}%
\begin{pgfscope}%
\pgfpathrectangle{\pgfqpoint{1.249408in}{0.585648in}}{\pgfqpoint{4.267222in}{5.609167in}}%
\pgfusepath{clip}%
\pgfsetrectcap%
\pgfsetroundjoin%
\pgfsetlinewidth{2.168100pt}%
\definecolor{currentstroke}{rgb}{0.260000,0.260000,0.260000}%
\pgfsetstrokecolor{currentstroke}%
\pgfsetdash{}{0pt}%
\pgfusepath{stroke}%
\end{pgfscope}%
\begin{pgfscope}%
\pgfpathrectangle{\pgfqpoint{1.249408in}{0.585648in}}{\pgfqpoint{4.267222in}{5.609167in}}%
\pgfusepath{clip}%
\pgfsetrectcap%
\pgfsetroundjoin%
\pgfsetlinewidth{2.168100pt}%
\definecolor{currentstroke}{rgb}{0.260000,0.260000,0.260000}%
\pgfsetstrokecolor{currentstroke}%
\pgfsetdash{}{0pt}%
\pgfusepath{stroke}%
\end{pgfscope}%
\begin{pgfscope}%
\pgfpathrectangle{\pgfqpoint{1.249408in}{0.585648in}}{\pgfqpoint{4.267222in}{5.609167in}}%
\pgfusepath{clip}%
\pgfsetrectcap%
\pgfsetroundjoin%
\pgfsetlinewidth{2.168100pt}%
\definecolor{currentstroke}{rgb}{0.260000,0.260000,0.260000}%
\pgfsetstrokecolor{currentstroke}%
\pgfsetdash{}{0pt}%
\pgfusepath{stroke}%
\end{pgfscope}%
\begin{pgfscope}%
\pgfpathrectangle{\pgfqpoint{1.249408in}{0.585648in}}{\pgfqpoint{4.267222in}{5.609167in}}%
\pgfusepath{clip}%
\pgfsetrectcap%
\pgfsetroundjoin%
\pgfsetlinewidth{2.168100pt}%
\definecolor{currentstroke}{rgb}{0.260000,0.260000,0.260000}%
\pgfsetstrokecolor{currentstroke}%
\pgfsetdash{}{0pt}%
\pgfusepath{stroke}%
\end{pgfscope}%
\begin{pgfscope}%
\pgfpathrectangle{\pgfqpoint{1.249408in}{0.585648in}}{\pgfqpoint{4.267222in}{5.609167in}}%
\pgfusepath{clip}%
\pgfsetrectcap%
\pgfsetroundjoin%
\pgfsetlinewidth{2.168100pt}%
\definecolor{currentstroke}{rgb}{0.260000,0.260000,0.260000}%
\pgfsetstrokecolor{currentstroke}%
\pgfsetdash{}{0pt}%
\pgfusepath{stroke}%
\end{pgfscope}%
\begin{pgfscope}%
\pgfpathrectangle{\pgfqpoint{1.249408in}{0.585648in}}{\pgfqpoint{4.267222in}{5.609167in}}%
\pgfusepath{clip}%
\pgfsetrectcap%
\pgfsetroundjoin%
\pgfsetlinewidth{2.168100pt}%
\definecolor{currentstroke}{rgb}{0.260000,0.260000,0.260000}%
\pgfsetstrokecolor{currentstroke}%
\pgfsetdash{}{0pt}%
\pgfusepath{stroke}%
\end{pgfscope}%
\begin{pgfscope}%
\pgfpathrectangle{\pgfqpoint{1.249408in}{0.585648in}}{\pgfqpoint{4.267222in}{5.609167in}}%
\pgfusepath{clip}%
\pgfsetrectcap%
\pgfsetroundjoin%
\pgfsetlinewidth{2.168100pt}%
\definecolor{currentstroke}{rgb}{0.260000,0.260000,0.260000}%
\pgfsetstrokecolor{currentstroke}%
\pgfsetdash{}{0pt}%
\pgfusepath{stroke}%
\end{pgfscope}%
\begin{pgfscope}%
\pgfpathrectangle{\pgfqpoint{1.249408in}{0.585648in}}{\pgfqpoint{4.267222in}{5.609167in}}%
\pgfusepath{clip}%
\pgfsetrectcap%
\pgfsetroundjoin%
\pgfsetlinewidth{2.168100pt}%
\definecolor{currentstroke}{rgb}{0.260000,0.260000,0.260000}%
\pgfsetstrokecolor{currentstroke}%
\pgfsetdash{}{0pt}%
\pgfusepath{stroke}%
\end{pgfscope}%
\begin{pgfscope}%
\pgfpathrectangle{\pgfqpoint{1.249408in}{0.585648in}}{\pgfqpoint{4.267222in}{5.609167in}}%
\pgfusepath{clip}%
\pgfsetrectcap%
\pgfsetroundjoin%
\pgfsetlinewidth{2.168100pt}%
\definecolor{currentstroke}{rgb}{0.260000,0.260000,0.260000}%
\pgfsetstrokecolor{currentstroke}%
\pgfsetdash{}{0pt}%
\pgfusepath{stroke}%
\end{pgfscope}%
\begin{pgfscope}%
\pgfpathrectangle{\pgfqpoint{1.249408in}{0.585648in}}{\pgfqpoint{4.267222in}{5.609167in}}%
\pgfusepath{clip}%
\pgfsetrectcap%
\pgfsetroundjoin%
\pgfsetlinewidth{2.168100pt}%
\definecolor{currentstroke}{rgb}{0.260000,0.260000,0.260000}%
\pgfsetstrokecolor{currentstroke}%
\pgfsetdash{}{0pt}%
\pgfusepath{stroke}%
\end{pgfscope}%
\begin{pgfscope}%
\pgfpathrectangle{\pgfqpoint{1.249408in}{0.585648in}}{\pgfqpoint{4.267222in}{5.609167in}}%
\pgfusepath{clip}%
\pgfsetrectcap%
\pgfsetroundjoin%
\pgfsetlinewidth{2.168100pt}%
\definecolor{currentstroke}{rgb}{0.260000,0.260000,0.260000}%
\pgfsetstrokecolor{currentstroke}%
\pgfsetdash{}{0pt}%
\pgfusepath{stroke}%
\end{pgfscope}%
\begin{pgfscope}%
\pgfpathrectangle{\pgfqpoint{1.249408in}{0.585648in}}{\pgfqpoint{4.267222in}{5.609167in}}%
\pgfusepath{clip}%
\pgfsetrectcap%
\pgfsetroundjoin%
\pgfsetlinewidth{2.168100pt}%
\definecolor{currentstroke}{rgb}{0.260000,0.260000,0.260000}%
\pgfsetstrokecolor{currentstroke}%
\pgfsetdash{}{0pt}%
\pgfusepath{stroke}%
\end{pgfscope}%
\begin{pgfscope}%
\pgfpathrectangle{\pgfqpoint{1.249408in}{0.585648in}}{\pgfqpoint{4.267222in}{5.609167in}}%
\pgfusepath{clip}%
\pgfsetrectcap%
\pgfsetroundjoin%
\pgfsetlinewidth{2.168100pt}%
\definecolor{currentstroke}{rgb}{0.260000,0.260000,0.260000}%
\pgfsetstrokecolor{currentstroke}%
\pgfsetdash{}{0pt}%
\pgfusepath{stroke}%
\end{pgfscope}%
\begin{pgfscope}%
\pgfpathrectangle{\pgfqpoint{1.249408in}{0.585648in}}{\pgfqpoint{4.267222in}{5.609167in}}%
\pgfusepath{clip}%
\pgfsetbuttcap%
\pgfsetmiterjoin%
\definecolor{currentfill}{rgb}{0.565196,0.347549,0.597549}%
\pgfsetfillcolor{currentfill}%
\pgfsetlinewidth{0.000000pt}%
\definecolor{currentstroke}{rgb}{0.000000,0.000000,0.000000}%
\pgfsetstrokecolor{currentstroke}%
\pgfsetstrokeopacity{0.000000}%
\pgfsetdash{}{0pt}%
\pgfpathmoveto{\pgfqpoint{1.249408in}{5.914356in}}%
\pgfpathlineto{\pgfqpoint{4.553355in}{5.914356in}}%
\pgfpathlineto{\pgfqpoint{4.553355in}{5.834225in}}%
\pgfpathlineto{\pgfqpoint{1.249408in}{5.834225in}}%
\pgfpathlineto{\pgfqpoint{1.249408in}{5.914356in}}%
\pgfpathclose%
\pgfusepath{fill}%
\end{pgfscope}%
\begin{pgfscope}%
\pgfpathrectangle{\pgfqpoint{1.249408in}{0.585648in}}{\pgfqpoint{4.267222in}{5.609167in}}%
\pgfusepath{clip}%
\pgfsetbuttcap%
\pgfsetmiterjoin%
\definecolor{currentfill}{rgb}{0.565196,0.347549,0.597549}%
\pgfsetfillcolor{currentfill}%
\pgfsetlinewidth{0.000000pt}%
\definecolor{currentstroke}{rgb}{0.000000,0.000000,0.000000}%
\pgfsetstrokecolor{currentstroke}%
\pgfsetstrokeopacity{0.000000}%
\pgfsetdash{}{0pt}%
\pgfpathmoveto{\pgfqpoint{1.249408in}{5.513701in}}%
\pgfpathlineto{\pgfqpoint{3.466631in}{5.513701in}}%
\pgfpathlineto{\pgfqpoint{3.466631in}{5.433570in}}%
\pgfpathlineto{\pgfqpoint{1.249408in}{5.433570in}}%
\pgfpathlineto{\pgfqpoint{1.249408in}{5.513701in}}%
\pgfpathclose%
\pgfusepath{fill}%
\end{pgfscope}%
\begin{pgfscope}%
\pgfpathrectangle{\pgfqpoint{1.249408in}{0.585648in}}{\pgfqpoint{4.267222in}{5.609167in}}%
\pgfusepath{clip}%
\pgfsetbuttcap%
\pgfsetmiterjoin%
\definecolor{currentfill}{rgb}{0.565196,0.347549,0.597549}%
\pgfsetfillcolor{currentfill}%
\pgfsetlinewidth{0.000000pt}%
\definecolor{currentstroke}{rgb}{0.000000,0.000000,0.000000}%
\pgfsetstrokecolor{currentstroke}%
\pgfsetstrokeopacity{0.000000}%
\pgfsetdash{}{0pt}%
\pgfpathmoveto{\pgfqpoint{1.249408in}{5.113046in}}%
\pgfpathlineto{\pgfqpoint{3.045024in}{5.113046in}}%
\pgfpathlineto{\pgfqpoint{3.045024in}{5.032915in}}%
\pgfpathlineto{\pgfqpoint{1.249408in}{5.032915in}}%
\pgfpathlineto{\pgfqpoint{1.249408in}{5.113046in}}%
\pgfpathclose%
\pgfusepath{fill}%
\end{pgfscope}%
\begin{pgfscope}%
\pgfpathrectangle{\pgfqpoint{1.249408in}{0.585648in}}{\pgfqpoint{4.267222in}{5.609167in}}%
\pgfusepath{clip}%
\pgfsetbuttcap%
\pgfsetmiterjoin%
\definecolor{currentfill}{rgb}{0.565196,0.347549,0.597549}%
\pgfsetfillcolor{currentfill}%
\pgfsetlinewidth{0.000000pt}%
\definecolor{currentstroke}{rgb}{0.000000,0.000000,0.000000}%
\pgfsetstrokecolor{currentstroke}%
\pgfsetstrokeopacity{0.000000}%
\pgfsetdash{}{0pt}%
\pgfpathmoveto{\pgfqpoint{1.249408in}{4.712392in}}%
\pgfpathlineto{\pgfqpoint{1.367048in}{4.712392in}}%
\pgfpathlineto{\pgfqpoint{1.367048in}{4.632261in}}%
\pgfpathlineto{\pgfqpoint{1.249408in}{4.632261in}}%
\pgfpathlineto{\pgfqpoint{1.249408in}{4.712392in}}%
\pgfpathclose%
\pgfusepath{fill}%
\end{pgfscope}%
\begin{pgfscope}%
\pgfpathrectangle{\pgfqpoint{1.249408in}{0.585648in}}{\pgfqpoint{4.267222in}{5.609167in}}%
\pgfusepath{clip}%
\pgfsetbuttcap%
\pgfsetmiterjoin%
\definecolor{currentfill}{rgb}{0.565196,0.347549,0.597549}%
\pgfsetfillcolor{currentfill}%
\pgfsetlinewidth{0.000000pt}%
\definecolor{currentstroke}{rgb}{0.000000,0.000000,0.000000}%
\pgfsetstrokecolor{currentstroke}%
\pgfsetstrokeopacity{0.000000}%
\pgfsetdash{}{0pt}%
\pgfpathmoveto{\pgfqpoint{1.249408in}{4.311737in}}%
\pgfpathlineto{\pgfqpoint{4.903641in}{4.311737in}}%
\pgfpathlineto{\pgfqpoint{4.903641in}{4.231606in}}%
\pgfpathlineto{\pgfqpoint{1.249408in}{4.231606in}}%
\pgfpathlineto{\pgfqpoint{1.249408in}{4.311737in}}%
\pgfpathclose%
\pgfusepath{fill}%
\end{pgfscope}%
\begin{pgfscope}%
\pgfpathrectangle{\pgfqpoint{1.249408in}{0.585648in}}{\pgfqpoint{4.267222in}{5.609167in}}%
\pgfusepath{clip}%
\pgfsetbuttcap%
\pgfsetmiterjoin%
\definecolor{currentfill}{rgb}{0.565196,0.347549,0.597549}%
\pgfsetfillcolor{currentfill}%
\pgfsetlinewidth{0.000000pt}%
\definecolor{currentstroke}{rgb}{0.000000,0.000000,0.000000}%
\pgfsetstrokecolor{currentstroke}%
\pgfsetstrokeopacity{0.000000}%
\pgfsetdash{}{0pt}%
\pgfpathmoveto{\pgfqpoint{1.249408in}{3.911082in}}%
\pgfpathlineto{\pgfqpoint{3.613973in}{3.911082in}}%
\pgfpathlineto{\pgfqpoint{3.613973in}{3.830951in}}%
\pgfpathlineto{\pgfqpoint{1.249408in}{3.830951in}}%
\pgfpathlineto{\pgfqpoint{1.249408in}{3.911082in}}%
\pgfpathclose%
\pgfusepath{fill}%
\end{pgfscope}%
\begin{pgfscope}%
\pgfpathrectangle{\pgfqpoint{1.249408in}{0.585648in}}{\pgfqpoint{4.267222in}{5.609167in}}%
\pgfusepath{clip}%
\pgfsetbuttcap%
\pgfsetmiterjoin%
\definecolor{currentfill}{rgb}{0.565196,0.347549,0.597549}%
\pgfsetfillcolor{currentfill}%
\pgfsetlinewidth{0.000000pt}%
\definecolor{currentstroke}{rgb}{0.000000,0.000000,0.000000}%
\pgfsetstrokecolor{currentstroke}%
\pgfsetstrokeopacity{0.000000}%
\pgfsetdash{}{0pt}%
\pgfpathmoveto{\pgfqpoint{1.249408in}{3.510427in}}%
\pgfpathlineto{\pgfqpoint{2.461981in}{3.510427in}}%
\pgfpathlineto{\pgfqpoint{2.461981in}{3.430296in}}%
\pgfpathlineto{\pgfqpoint{1.249408in}{3.430296in}}%
\pgfpathlineto{\pgfqpoint{1.249408in}{3.510427in}}%
\pgfpathclose%
\pgfusepath{fill}%
\end{pgfscope}%
\begin{pgfscope}%
\pgfpathrectangle{\pgfqpoint{1.249408in}{0.585648in}}{\pgfqpoint{4.267222in}{5.609167in}}%
\pgfusepath{clip}%
\pgfsetbuttcap%
\pgfsetmiterjoin%
\definecolor{currentfill}{rgb}{0.565196,0.347549,0.597549}%
\pgfsetfillcolor{currentfill}%
\pgfsetlinewidth{0.000000pt}%
\definecolor{currentstroke}{rgb}{0.000000,0.000000,0.000000}%
\pgfsetstrokecolor{currentstroke}%
\pgfsetstrokeopacity{0.000000}%
\pgfsetdash{}{0pt}%
\pgfpathmoveto{\pgfqpoint{1.249408in}{3.109772in}}%
\pgfpathlineto{\pgfqpoint{3.044675in}{3.109772in}}%
\pgfpathlineto{\pgfqpoint{3.044675in}{3.029642in}}%
\pgfpathlineto{\pgfqpoint{1.249408in}{3.029642in}}%
\pgfpathlineto{\pgfqpoint{1.249408in}{3.109772in}}%
\pgfpathclose%
\pgfusepath{fill}%
\end{pgfscope}%
\begin{pgfscope}%
\pgfpathrectangle{\pgfqpoint{1.249408in}{0.585648in}}{\pgfqpoint{4.267222in}{5.609167in}}%
\pgfusepath{clip}%
\pgfsetbuttcap%
\pgfsetmiterjoin%
\definecolor{currentfill}{rgb}{0.565196,0.347549,0.597549}%
\pgfsetfillcolor{currentfill}%
\pgfsetlinewidth{0.000000pt}%
\definecolor{currentstroke}{rgb}{0.000000,0.000000,0.000000}%
\pgfsetstrokecolor{currentstroke}%
\pgfsetstrokeopacity{0.000000}%
\pgfsetdash{}{0pt}%
\pgfpathmoveto{\pgfqpoint{1.249408in}{2.709118in}}%
\pgfpathlineto{\pgfqpoint{1.306438in}{2.709118in}}%
\pgfpathlineto{\pgfqpoint{1.306438in}{2.628987in}}%
\pgfpathlineto{\pgfqpoint{1.249408in}{2.628987in}}%
\pgfpathlineto{\pgfqpoint{1.249408in}{2.709118in}}%
\pgfpathclose%
\pgfusepath{fill}%
\end{pgfscope}%
\begin{pgfscope}%
\pgfpathrectangle{\pgfqpoint{1.249408in}{0.585648in}}{\pgfqpoint{4.267222in}{5.609167in}}%
\pgfusepath{clip}%
\pgfsetbuttcap%
\pgfsetmiterjoin%
\definecolor{currentfill}{rgb}{0.565196,0.347549,0.597549}%
\pgfsetfillcolor{currentfill}%
\pgfsetlinewidth{0.000000pt}%
\definecolor{currentstroke}{rgb}{0.000000,0.000000,0.000000}%
\pgfsetstrokecolor{currentstroke}%
\pgfsetstrokeopacity{0.000000}%
\pgfsetdash{}{0pt}%
\pgfpathmoveto{\pgfqpoint{1.249408in}{2.308463in}}%
\pgfpathlineto{\pgfqpoint{2.608527in}{2.308463in}}%
\pgfpathlineto{\pgfqpoint{2.608527in}{2.228332in}}%
\pgfpathlineto{\pgfqpoint{1.249408in}{2.228332in}}%
\pgfpathlineto{\pgfqpoint{1.249408in}{2.308463in}}%
\pgfpathclose%
\pgfusepath{fill}%
\end{pgfscope}%
\begin{pgfscope}%
\pgfpathrectangle{\pgfqpoint{1.249408in}{0.585648in}}{\pgfqpoint{4.267222in}{5.609167in}}%
\pgfusepath{clip}%
\pgfsetbuttcap%
\pgfsetmiterjoin%
\definecolor{currentfill}{rgb}{0.565196,0.347549,0.597549}%
\pgfsetfillcolor{currentfill}%
\pgfsetlinewidth{0.000000pt}%
\definecolor{currentstroke}{rgb}{0.000000,0.000000,0.000000}%
\pgfsetstrokecolor{currentstroke}%
\pgfsetstrokeopacity{0.000000}%
\pgfsetdash{}{0pt}%
\pgfpathmoveto{\pgfqpoint{1.249408in}{1.907808in}}%
\pgfpathlineto{\pgfqpoint{1.147790in}{1.907808in}}%
\pgfpathlineto{\pgfqpoint{1.147790in}{1.827677in}}%
\pgfpathlineto{\pgfqpoint{1.249408in}{1.827677in}}%
\pgfpathlineto{\pgfqpoint{1.249408in}{1.907808in}}%
\pgfpathclose%
\pgfusepath{fill}%
\end{pgfscope}%
\begin{pgfscope}%
\pgfpathrectangle{\pgfqpoint{1.249408in}{0.585648in}}{\pgfqpoint{4.267222in}{5.609167in}}%
\pgfusepath{clip}%
\pgfsetbuttcap%
\pgfsetmiterjoin%
\definecolor{currentfill}{rgb}{0.565196,0.347549,0.597549}%
\pgfsetfillcolor{currentfill}%
\pgfsetlinewidth{0.000000pt}%
\definecolor{currentstroke}{rgb}{0.000000,0.000000,0.000000}%
\pgfsetstrokecolor{currentstroke}%
\pgfsetstrokeopacity{0.000000}%
\pgfsetdash{}{0pt}%
\pgfpathmoveto{\pgfqpoint{1.249408in}{1.507153in}}%
\pgfpathlineto{\pgfqpoint{1.503409in}{1.507153in}}%
\pgfpathlineto{\pgfqpoint{1.503409in}{1.427022in}}%
\pgfpathlineto{\pgfqpoint{1.249408in}{1.427022in}}%
\pgfpathlineto{\pgfqpoint{1.249408in}{1.507153in}}%
\pgfpathclose%
\pgfusepath{fill}%
\end{pgfscope}%
\begin{pgfscope}%
\pgfpathrectangle{\pgfqpoint{1.249408in}{0.585648in}}{\pgfqpoint{4.267222in}{5.609167in}}%
\pgfusepath{clip}%
\pgfsetbuttcap%
\pgfsetmiterjoin%
\definecolor{currentfill}{rgb}{0.565196,0.347549,0.597549}%
\pgfsetfillcolor{currentfill}%
\pgfsetlinewidth{0.000000pt}%
\definecolor{currentstroke}{rgb}{0.000000,0.000000,0.000000}%
\pgfsetstrokecolor{currentstroke}%
\pgfsetstrokeopacity{0.000000}%
\pgfsetdash{}{0pt}%
\pgfpathmoveto{\pgfqpoint{1.249408in}{1.106499in}}%
\pgfpathlineto{\pgfqpoint{3.219352in}{1.106499in}}%
\pgfpathlineto{\pgfqpoint{3.219352in}{1.026368in}}%
\pgfpathlineto{\pgfqpoint{1.249408in}{1.026368in}}%
\pgfpathlineto{\pgfqpoint{1.249408in}{1.106499in}}%
\pgfpathclose%
\pgfusepath{fill}%
\end{pgfscope}%
\begin{pgfscope}%
\pgfpathrectangle{\pgfqpoint{1.249408in}{0.585648in}}{\pgfqpoint{4.267222in}{5.609167in}}%
\pgfusepath{clip}%
\pgfsetbuttcap%
\pgfsetmiterjoin%
\definecolor{currentfill}{rgb}{0.565196,0.347549,0.597549}%
\pgfsetfillcolor{currentfill}%
\pgfsetlinewidth{0.000000pt}%
\definecolor{currentstroke}{rgb}{0.000000,0.000000,0.000000}%
\pgfsetstrokecolor{currentstroke}%
\pgfsetstrokeopacity{0.000000}%
\pgfsetdash{}{0pt}%
\pgfpathmoveto{\pgfqpoint{1.249408in}{0.705844in}}%
\pgfpathlineto{\pgfqpoint{2.713329in}{0.705844in}}%
\pgfpathlineto{\pgfqpoint{2.713329in}{0.625713in}}%
\pgfpathlineto{\pgfqpoint{1.249408in}{0.625713in}}%
\pgfpathlineto{\pgfqpoint{1.249408in}{0.705844in}}%
\pgfpathclose%
\pgfusepath{fill}%
\end{pgfscope}%
\begin{pgfscope}%
\pgfpathrectangle{\pgfqpoint{1.249408in}{0.585648in}}{\pgfqpoint{4.267222in}{5.609167in}}%
\pgfusepath{clip}%
\pgfsetrectcap%
\pgfsetroundjoin%
\pgfsetlinewidth{2.168100pt}%
\definecolor{currentstroke}{rgb}{0.260000,0.260000,0.260000}%
\pgfsetstrokecolor{currentstroke}%
\pgfsetdash{}{0pt}%
\pgfusepath{stroke}%
\end{pgfscope}%
\begin{pgfscope}%
\pgfpathrectangle{\pgfqpoint{1.249408in}{0.585648in}}{\pgfqpoint{4.267222in}{5.609167in}}%
\pgfusepath{clip}%
\pgfsetrectcap%
\pgfsetroundjoin%
\pgfsetlinewidth{2.168100pt}%
\definecolor{currentstroke}{rgb}{0.260000,0.260000,0.260000}%
\pgfsetstrokecolor{currentstroke}%
\pgfsetdash{}{0pt}%
\pgfusepath{stroke}%
\end{pgfscope}%
\begin{pgfscope}%
\pgfpathrectangle{\pgfqpoint{1.249408in}{0.585648in}}{\pgfqpoint{4.267222in}{5.609167in}}%
\pgfusepath{clip}%
\pgfsetrectcap%
\pgfsetroundjoin%
\pgfsetlinewidth{2.168100pt}%
\definecolor{currentstroke}{rgb}{0.260000,0.260000,0.260000}%
\pgfsetstrokecolor{currentstroke}%
\pgfsetdash{}{0pt}%
\pgfusepath{stroke}%
\end{pgfscope}%
\begin{pgfscope}%
\pgfpathrectangle{\pgfqpoint{1.249408in}{0.585648in}}{\pgfqpoint{4.267222in}{5.609167in}}%
\pgfusepath{clip}%
\pgfsetrectcap%
\pgfsetroundjoin%
\pgfsetlinewidth{2.168100pt}%
\definecolor{currentstroke}{rgb}{0.260000,0.260000,0.260000}%
\pgfsetstrokecolor{currentstroke}%
\pgfsetdash{}{0pt}%
\pgfusepath{stroke}%
\end{pgfscope}%
\begin{pgfscope}%
\pgfpathrectangle{\pgfqpoint{1.249408in}{0.585648in}}{\pgfqpoint{4.267222in}{5.609167in}}%
\pgfusepath{clip}%
\pgfsetrectcap%
\pgfsetroundjoin%
\pgfsetlinewidth{2.168100pt}%
\definecolor{currentstroke}{rgb}{0.260000,0.260000,0.260000}%
\pgfsetstrokecolor{currentstroke}%
\pgfsetdash{}{0pt}%
\pgfusepath{stroke}%
\end{pgfscope}%
\begin{pgfscope}%
\pgfpathrectangle{\pgfqpoint{1.249408in}{0.585648in}}{\pgfqpoint{4.267222in}{5.609167in}}%
\pgfusepath{clip}%
\pgfsetrectcap%
\pgfsetroundjoin%
\pgfsetlinewidth{2.168100pt}%
\definecolor{currentstroke}{rgb}{0.260000,0.260000,0.260000}%
\pgfsetstrokecolor{currentstroke}%
\pgfsetdash{}{0pt}%
\pgfusepath{stroke}%
\end{pgfscope}%
\begin{pgfscope}%
\pgfpathrectangle{\pgfqpoint{1.249408in}{0.585648in}}{\pgfqpoint{4.267222in}{5.609167in}}%
\pgfusepath{clip}%
\pgfsetrectcap%
\pgfsetroundjoin%
\pgfsetlinewidth{2.168100pt}%
\definecolor{currentstroke}{rgb}{0.260000,0.260000,0.260000}%
\pgfsetstrokecolor{currentstroke}%
\pgfsetdash{}{0pt}%
\pgfusepath{stroke}%
\end{pgfscope}%
\begin{pgfscope}%
\pgfpathrectangle{\pgfqpoint{1.249408in}{0.585648in}}{\pgfqpoint{4.267222in}{5.609167in}}%
\pgfusepath{clip}%
\pgfsetrectcap%
\pgfsetroundjoin%
\pgfsetlinewidth{2.168100pt}%
\definecolor{currentstroke}{rgb}{0.260000,0.260000,0.260000}%
\pgfsetstrokecolor{currentstroke}%
\pgfsetdash{}{0pt}%
\pgfusepath{stroke}%
\end{pgfscope}%
\begin{pgfscope}%
\pgfpathrectangle{\pgfqpoint{1.249408in}{0.585648in}}{\pgfqpoint{4.267222in}{5.609167in}}%
\pgfusepath{clip}%
\pgfsetrectcap%
\pgfsetroundjoin%
\pgfsetlinewidth{2.168100pt}%
\definecolor{currentstroke}{rgb}{0.260000,0.260000,0.260000}%
\pgfsetstrokecolor{currentstroke}%
\pgfsetdash{}{0pt}%
\pgfusepath{stroke}%
\end{pgfscope}%
\begin{pgfscope}%
\pgfpathrectangle{\pgfqpoint{1.249408in}{0.585648in}}{\pgfqpoint{4.267222in}{5.609167in}}%
\pgfusepath{clip}%
\pgfsetrectcap%
\pgfsetroundjoin%
\pgfsetlinewidth{2.168100pt}%
\definecolor{currentstroke}{rgb}{0.260000,0.260000,0.260000}%
\pgfsetstrokecolor{currentstroke}%
\pgfsetdash{}{0pt}%
\pgfusepath{stroke}%
\end{pgfscope}%
\begin{pgfscope}%
\pgfpathrectangle{\pgfqpoint{1.249408in}{0.585648in}}{\pgfqpoint{4.267222in}{5.609167in}}%
\pgfusepath{clip}%
\pgfsetrectcap%
\pgfsetroundjoin%
\pgfsetlinewidth{2.168100pt}%
\definecolor{currentstroke}{rgb}{0.260000,0.260000,0.260000}%
\pgfsetstrokecolor{currentstroke}%
\pgfsetdash{}{0pt}%
\pgfusepath{stroke}%
\end{pgfscope}%
\begin{pgfscope}%
\pgfpathrectangle{\pgfqpoint{1.249408in}{0.585648in}}{\pgfqpoint{4.267222in}{5.609167in}}%
\pgfusepath{clip}%
\pgfsetrectcap%
\pgfsetroundjoin%
\pgfsetlinewidth{2.168100pt}%
\definecolor{currentstroke}{rgb}{0.260000,0.260000,0.260000}%
\pgfsetstrokecolor{currentstroke}%
\pgfsetdash{}{0pt}%
\pgfusepath{stroke}%
\end{pgfscope}%
\begin{pgfscope}%
\pgfpathrectangle{\pgfqpoint{1.249408in}{0.585648in}}{\pgfqpoint{4.267222in}{5.609167in}}%
\pgfusepath{clip}%
\pgfsetrectcap%
\pgfsetroundjoin%
\pgfsetlinewidth{2.168100pt}%
\definecolor{currentstroke}{rgb}{0.260000,0.260000,0.260000}%
\pgfsetstrokecolor{currentstroke}%
\pgfsetdash{}{0pt}%
\pgfusepath{stroke}%
\end{pgfscope}%
\begin{pgfscope}%
\pgfpathrectangle{\pgfqpoint{1.249408in}{0.585648in}}{\pgfqpoint{4.267222in}{5.609167in}}%
\pgfusepath{clip}%
\pgfsetrectcap%
\pgfsetroundjoin%
\pgfsetlinewidth{2.168100pt}%
\definecolor{currentstroke}{rgb}{0.260000,0.260000,0.260000}%
\pgfsetstrokecolor{currentstroke}%
\pgfsetdash{}{0pt}%
\pgfusepath{stroke}%
\end{pgfscope}%
\begin{pgfscope}%
\pgfsetbuttcap%
\pgfsetmiterjoin%
\definecolor{currentfill}{rgb}{1.000000,1.000000,1.000000}%
\pgfsetfillcolor{currentfill}%
\pgfsetfillopacity{0.800000}%
\pgfsetlinewidth{0.803000pt}%
\definecolor{currentstroke}{rgb}{0.800000,0.800000,0.800000}%
\pgfsetstrokecolor{currentstroke}%
\pgfsetstrokeopacity{0.800000}%
\pgfsetdash{}{0pt}%
\pgfpathmoveto{\pgfqpoint{3.838865in}{0.682870in}}%
\pgfpathlineto{\pgfqpoint{5.380519in}{0.682870in}}%
\pgfpathquadraticcurveto{\pgfqpoint{5.419408in}{0.682870in}}{\pgfqpoint{5.419408in}{0.721759in}}%
\pgfpathlineto{\pgfqpoint{5.419408in}{2.077312in}}%
\pgfpathquadraticcurveto{\pgfqpoint{5.419408in}{2.116201in}}{\pgfqpoint{5.380519in}{2.116201in}}%
\pgfpathlineto{\pgfqpoint{3.838865in}{2.116201in}}%
\pgfpathquadraticcurveto{\pgfqpoint{3.799976in}{2.116201in}}{\pgfqpoint{3.799976in}{2.077312in}}%
\pgfpathlineto{\pgfqpoint{3.799976in}{0.721759in}}%
\pgfpathquadraticcurveto{\pgfqpoint{3.799976in}{0.682870in}}{\pgfqpoint{3.838865in}{0.682870in}}%
\pgfpathlineto{\pgfqpoint{3.838865in}{0.682870in}}%
\pgfpathclose%
\pgfusepath{stroke,fill}%
\end{pgfscope}%
\begin{pgfscope}%
\definecolor{textcolor}{rgb}{0.000000,0.000000,0.000000}%
\pgfsetstrokecolor{textcolor}%
\pgfsetfillcolor{textcolor}%
\pgftext[x=4.058715in,y=1.899535in,left,base]{\color{textcolor}\sffamily\fontsize{14.000000}{16.800000}\selectfont Configuration}%
\end{pgfscope}%
\begin{pgfscope}%
\pgfsetbuttcap%
\pgfsetmiterjoin%
\definecolor{currentfill}{rgb}{0.795098,0.200980,0.206863}%
\pgfsetfillcolor{currentfill}%
\pgfsetlinewidth{0.000000pt}%
\definecolor{currentstroke}{rgb}{0.000000,0.000000,0.000000}%
\pgfsetstrokecolor{currentstroke}%
\pgfsetstrokeopacity{0.000000}%
\pgfsetdash{}{0pt}%
\pgfpathmoveto{\pgfqpoint{3.877754in}{1.624535in}}%
\pgfpathlineto{\pgfqpoint{4.266643in}{1.624535in}}%
\pgfpathlineto{\pgfqpoint{4.266643in}{1.760646in}}%
\pgfpathlineto{\pgfqpoint{3.877754in}{1.760646in}}%
\pgfpathlineto{\pgfqpoint{3.877754in}{1.624535in}}%
\pgfpathclose%
\pgfusepath{fill}%
\end{pgfscope}%
\begin{pgfscope}%
\definecolor{textcolor}{rgb}{0.000000,0.000000,0.000000}%
\pgfsetstrokecolor{textcolor}%
\pgfsetfillcolor{textcolor}%
\pgftext[x=4.422198in,y=1.624535in,left,base]{\color{textcolor}\sffamily\fontsize{14.000000}{16.800000}\selectfont i9-9900K}%
\end{pgfscope}%
\begin{pgfscope}%
\pgfsetbuttcap%
\pgfsetmiterjoin%
\definecolor{currentfill}{rgb}{0.278922,0.487745,0.658333}%
\pgfsetfillcolor{currentfill}%
\pgfsetlinewidth{0.000000pt}%
\definecolor{currentstroke}{rgb}{0.000000,0.000000,0.000000}%
\pgfsetstrokecolor{currentstroke}%
\pgfsetstrokeopacity{0.000000}%
\pgfsetdash{}{0pt}%
\pgfpathmoveto{\pgfqpoint{3.877754in}{1.349535in}}%
\pgfpathlineto{\pgfqpoint{4.266643in}{1.349535in}}%
\pgfpathlineto{\pgfqpoint{4.266643in}{1.485647in}}%
\pgfpathlineto{\pgfqpoint{3.877754in}{1.485647in}}%
\pgfpathlineto{\pgfqpoint{3.877754in}{1.349535in}}%
\pgfpathclose%
\pgfusepath{fill}%
\end{pgfscope}%
\begin{pgfscope}%
\definecolor{textcolor}{rgb}{0.000000,0.000000,0.000000}%
\pgfsetstrokecolor{textcolor}%
\pgfsetfillcolor{textcolor}%
\pgftext[x=4.422198in,y=1.349535in,left,base]{\color{textcolor}\sffamily\fontsize{14.000000}{16.800000}\selectfont Epyc Rome}%
\end{pgfscope}%
\begin{pgfscope}%
\pgfsetbuttcap%
\pgfsetmiterjoin%
\definecolor{currentfill}{rgb}{0.348529,0.636765,0.339706}%
\pgfsetfillcolor{currentfill}%
\pgfsetlinewidth{0.000000pt}%
\definecolor{currentstroke}{rgb}{0.000000,0.000000,0.000000}%
\pgfsetstrokecolor{currentstroke}%
\pgfsetstrokeopacity{0.000000}%
\pgfsetdash{}{0pt}%
\pgfpathmoveto{\pgfqpoint{3.877754in}{1.074536in}}%
\pgfpathlineto{\pgfqpoint{4.266643in}{1.074536in}}%
\pgfpathlineto{\pgfqpoint{4.266643in}{1.210647in}}%
\pgfpathlineto{\pgfqpoint{3.877754in}{1.210647in}}%
\pgfpathlineto{\pgfqpoint{3.877754in}{1.074536in}}%
\pgfpathclose%
\pgfusepath{fill}%
\end{pgfscope}%
\begin{pgfscope}%
\definecolor{textcolor}{rgb}{0.000000,0.000000,0.000000}%
\pgfsetstrokecolor{textcolor}%
\pgfsetfillcolor{textcolor}%
\pgftext[x=4.422198in,y=1.074536in,left,base]{\color{textcolor}\sffamily\fontsize{14.000000}{16.800000}\selectfont TR 3990X}%
\end{pgfscope}%
\begin{pgfscope}%
\pgfsetbuttcap%
\pgfsetmiterjoin%
\definecolor{currentfill}{rgb}{0.565196,0.347549,0.597549}%
\pgfsetfillcolor{currentfill}%
\pgfsetlinewidth{0.000000pt}%
\definecolor{currentstroke}{rgb}{0.000000,0.000000,0.000000}%
\pgfsetstrokecolor{currentstroke}%
\pgfsetstrokeopacity{0.000000}%
\pgfsetdash{}{0pt}%
\pgfpathmoveto{\pgfqpoint{3.877754in}{0.799536in}}%
\pgfpathlineto{\pgfqpoint{4.266643in}{0.799536in}}%
\pgfpathlineto{\pgfqpoint{4.266643in}{0.935647in}}%
\pgfpathlineto{\pgfqpoint{3.877754in}{0.935647in}}%
\pgfpathlineto{\pgfqpoint{3.877754in}{0.799536in}}%
\pgfpathclose%
\pgfusepath{fill}%
\end{pgfscope}%
\begin{pgfscope}%
\definecolor{textcolor}{rgb}{0.000000,0.000000,0.000000}%
\pgfsetstrokecolor{textcolor}%
\pgfsetfillcolor{textcolor}%
\pgftext[x=4.422198in,y=0.799536in,left,base]{\color{textcolor}\sffamily\fontsize{14.000000}{16.800000}\selectfont Xeon}%
\end{pgfscope}%
\end{pgfpicture}%
\makeatother%
\endgroup%
%
    \endgroup    \caption{The performance impact of full protection with \rtwoc on four different machines (see \cref{sss:full-overhead}).}
    % \propername{Geomean int} shows the geometric mean overhead of the SPEC CPU 2017 integer benchmarks only and \propername{geomean all} of the entire benchmark suite.}
    \label{fig:perf-full}
\end{figure}

\subsection{System Configuration}\label{ss:eval-cfg} % about quarter of a page
% We performed the evaluation of the total overhead on four different machines.
We evaluate \rtwoc{} on four different machines.
Machine \propername{EPYC Rome} is equipped with an AMD EPYC Rome 7H12 CPU running at 3.2 GHz, 1TB DDR4 RAM running at 3200 MHz, and Debian 11.
Machine \propername{i9-9900K} is equipped with an Intel Core i9-9900K CPU running at 3.6 GHz, 64GB DDR4 RAM running at 2667 MHz, and Debian 11.
% Machine \propername{TR 3970X A} is equipped with an AMD Ryzen Threadripper 3970X CPU running at 3.7 GHz, 128GB DDR4 RAM running at 2400 MHz, and Debian 10.
Machine \propername{TR 3970X} is equipped with an AMD Ryzen Threadripper 3970X CPU running at 3.7 GHz, 128GB DDR4 RAM running at 2400 MHz, and Debian 10.
Machine \propername{Xeon} is equipped with an Intel Xeon Platinum 8358 CPU running at 2.60GHz, 256GB DDR4 RAM running at 3200 MHz, and Debian 11.

On each machine we used the bundled GCC and gold linker version to compile LLVM.
Our LLVM modifications are based on LLVM 11 and we compiled the benchmarks against the bundled \propername{glibc} and \propername{libstdc\plusplus} versions.
%    For \propername{musl} we used version 1.2.2 and the \propername{libc\plusplus} included with LLVM 11.

\subsection{Performance}\label{ss:perf}

To evaluate the performance impact of \rtwoc, we built and ran the SPEC CPU 2017 benchmark suite using our \rtwoc{} compiler.
The SPEC CPU 2017 suite is a collection of CPU-intensive C and \cpp benchmark programs.
%% We included the floating point benchmark suite to enable direct comparison with prior work \cite{vanderKouwe2019}.
To allow for direct comparison with prior work, we included the floating point benchmarks~\cite{vanderKouwe2019}.

We compiled all benchmarks with the \code{-O3} optimization level and link-time optimization---LLVM's ThinLTO model in our case---enabled.
As LLVM does not yet support the compilation of \propername{glibc}, we compiled the benchmarks against the unprotected system version of \propername{glibc} and \propername{libstdc\plusplus}.
To measure the worst-case overhead, we also enabled \glspl{BTRA} for call sites to unprotected code (see \cref{ss:limitations-coverage}).
For the evaluation of full \rtwoc{} we took the median execution time of 20 runs.
For the analysis of \rtwoc{}'s components we used \propername{EPYC Rome} and took the median execution time of 12 runs.
%\rtwoc{} inserts between one and five traps into each function prolog and each pair of \btra AVX setup instructions contains one to three NOPs (see \cref{ss:strengthening}).
Since the location of return addresses and the distribution of \glspl{heapbt} is random, we recompiled the benchmarks with a different seed for each of the executions.
To guarantee a fair comparison, we compiled the baseline with the same compiler version and flags but with \rtwoc disabled.

For the webserver benchmarks, we used \propername{wrk} as client and \propername{nginx} version 1.14.2 and \propername{Apache} version 2.4.54, serving 64-byte pages.
We split the CPU cores between \propername{wrk} and the webserver and gradually increased the number of concurrent connections until the CPU was fully saturated.
We compared the median throughput of five runs at the previously determined saturation connection count.
%Even when testing on the same host, however, we could not fully saturate the CPU, regardless of the number of connections\footnote{We tested up to 65536 connections and although the throughput started decreasing, the CPU saturation never exceeded 80\%}.

%To determine the incurred memory overhead for the SPEC benchmarks we recorded that maximum resident-set size of the benchmark processes.
%For the webservers we recorded the resident-set size of the parent webserver process in regular intervals and calculated the median.


%    To asses the performance impact of protecting the standard library we also built the benchmarks against a fully protected \propername{musl} libc and LLVM's \propername{libc\plusplus}.
%    Clang/LLVM currently does not compile the \propername{glibc} library, which is why we chose \propername{musl} and \propername{libc\plusplus} instead.
%    Since 2 of the 9 benchmarks in the benchmark suite are incompatible with \propername{musl}\footnote{The benchmarks \propername{perlbench} and \propername{gcc} are not compatible.}
%    and to ensure comparability with prior work, we report only the results of the \propername{glibc} based benchmarks here.
%    The performance results of the benchmarks compiled with the \musllib toolchain can be found in~\cref{apx:additional-benchmarks}.

\subsubsection{\glspl{BTRA}}\label{sss:eval-btra}
We evaluated the overhead of \glspl{BTRA} with the \code{push} setup and with our optimized AVX2 setup sequence respectively.
To isolate the overhead of \glspl{BTRA}, we disabled other diversification measures.
We configured \rtwoc{} to instrument each call site with a total of 10 \glspl{BTRA} and between 1 and 9 NOPs (see \cref{ss:strengthening}).

For the \code{push} setup, 10 \glspl{BTRA} mean that \rtwoc{} inserts up to 12 push instructions per call site: 10 for the \glspl{BTRA}, one for the return address, and one to keep the stack aligned (see \cref{ss:impl-rads}).
\cref{tab:perf-components} shows that the geometric mean overhead of this configuration is 6\%, but the outlier \propername{omnetpp} has an overhead of 21\%.

In contrast to the \code{push} instructions, setting up 10 \glspl{BTRA} with AVX2 instructions requires only 7 instructions (see \cref{ss:impl-rads}).
\fbetodo{adjust to new numbers}
\cref{tab:perf-components} shows that the optimization improves the overall performance by 2\%.
Most importantly, the optimization decreases the overhead of the outlier \propername{omnetpp} by about 13 absolute percent points, down to 8\%.
In this configuration, the maximum overhead of 10\% is caused by \propername{xalancbmk}.

To analyze the overhead of \cfs (see \cref{sss:stack-arguments}), we built a configuration without applying any diversification measure, but with \cfs \emph{enabled}.
Enabling only \cfs allows us to measure its performance impact, and the missed opportunities of the frame-pointer omission optimization.
We found that the resulting geometric mean performance overhead is \geomeancfs with a maximum impact of \maxoverheadcfs.
These numbers suggest that the majority of the overhead is caused by writing the \glspl{BTRA} to the stack.

%\subsubsection{\glspl{BTRA} with vector instructions}
%We built a configuration that uses our optimized AVX2 instruction setup sequence (see \propername{AVX2} in \cref{fig:perf-components}).
%In contrast to the \code{push} instructions, setting up ten \glspl{BTRA} with AVX2 instructions requires only 7 instructions (see \cref{ss:impl-rads}).
%\cref{fig:perf-components} shows that the performance increase afforded by the optimization varies per benchmark.
%\fbetodo{adjust to new numbers}
%Most importantly, the optimization decreases the overhead of the outlier \propername{omnetpp} by about 16\% absolute percent points, to 6.82\%.

%    \begin{table}[t!]
%      \renewcommand{\arraystretch}{1.0}
%      \centering
%      {\small
%        \begin{tabular}{lrrr}
%            % (median < '(2558   20996  371     301     6856   6548   1899   541     1564   1079   727     424    ))
%            % (median < '(121 397 10  8   46 38  254 17  14  90  8   161 ))
%            % (median < '(95 98 97 97 59 99 88 97 99 92 99 72))
%            \toprule
%            & \multicolumn{2}{c}{Code Pointers} & \\
%            \cmidrule{2-3}
%            {Benchmark}               & RA     & FP    & Pct. RA \\
%            \midrule
%            \propername{perlbench} & 2,558  & 120   & 95\%    \\
%            \propername{gcc}       & 20,996 & 396   & 98\%    \\
%            \propername{mcf}       & 371    & 9     & 97\%    \\
%%            \propername{lbm}       & 301    & 7     & 97\%    \\
%            \propername{omnetpp}   & 6,856  & 4,696 & 59\%    \\
%            \propername{xalancbmk} & 6,548  & 37    & 99\%    \\
%            \propername{x264}      & 1,899  & 253   & 88\%    \\
%            \propername{deepsjeng} & 541    & 16    & 97\%    \\
%%            \propername{imagick}   & 1,564  & 13    & 99\%    \\
%            \propername{leela}     & 1,079  & 89    & 92\%    \\
%%            \propername{nab}       & 727    & 7     & 99\%    \\
%            \propername{xz}        & 424    & 160   & 72\%    \\
%            \midrule
%            \propername{Apache}        & 6443   & 402   & 94\%    \\
%            \propername{nginx}         & 2976   & 46    & 98\%    \\
%            \midrule
%            Median                    & 1731   & 67    & 97\%    \\
%            \bottomrule
%        \end{tabular}}
%        \caption{Code pointer origin analysis: Distinct return addresses (RA) vs distinct function pointers (FP) on the stack in SPEC CPU 2017, Apache and nginx.}
%        \label{tab:quant-returns-vs-function-ptrs}
%        \vspace*{-2em}
%    \end{table}

\subsubsection{\glspl{heapbt}}\label{sss:eval-heapbt}
We configured \rtwoc{} to insert between zero and five \glspl{heapbt} per function, but disabled other diversification measures.
\fbetodo{adjust to new numbers}
\cref{tab:perf-components} shows that the geometric mean overhead of \glspl{heapbt} is 2\% with \propername{xalancbmk} causing the maximum overhead of 5\%.
The optimization to insert \glspl{heapbt} only in functions that write to their stack frame improves performance by 1\%.

\subsubsection{Prolog \& Layout Randomization}\label{sss:eval-layout}
We also isolated the performance impact of prolog trap insertion, and code- and data-layout randomization techniques---i.e., stack slot shuffling, global variable shuffling, and register-allocation randomization.
The prolog insertion randomly inserts between one and five traps into each function prolog, causing a geometric mean overhead of 2\%, with \propername{xalancbmk} being most affected at 6\%.
The combination of layout randomization techniques generally caused negligible overhead.

% \vspace*{-1em}

\subsubsection{Full \rtwoc}\label{sss:full-overhead}
We built a configuration with all \rtwoc{} protections enabled (see \cref{fig:perf-full}).
% would be nice to reiterate, but space is precious
This configuration
\begin{enumerate*}[label={(\roman*)}]
    \item protects return addresses with \glspl{BTRA} (see \cref{ss:impl-rads});
    \item injects \glspl{heapbt} (see \cref{ss:impl-heap-boobytraps});
    \item performs stack slot shuffling, global variable shuffling and register-allocation randomization;
    \item and inserts traps into function prologs and NOPs at call sites (see \cref{ss:strengthening}).
\end{enumerate*}
The geometric mean overhead is similar on all systems, with the \propername{Xeon} machine showing the highest overhead at 8.5\% for the full benchmark suite.
Some benchmarks show diverging results on different machines.
On \propername{i9-9900K}, \propername{perlbench} has a significantly higher overhead than on the other machines.
For \propername{omnetpp}, the \propername{Xeon} machine has the highest overhead at 21\%.
Conversely, \propername{xalancbmk} shows better results on \propername{i9-9900K} and \propername{Xeon} than on the AMD machines.

On \propername{i9-9900K}, we found the webserver throughput \emph{decrease} to be 13\% for \propername{nginx} and 12\% for \propername{Apache}.
On the AMD machines, the throughput decrease was between 3 and 4 percent for both \propername{nginx} and \propername{Apache}.

%    \subsubsection{Adaptive parameter selection}\label{sss:adaptive}
%    To demonstrate the advantages of adaptive security, we built two configurations with adaptive parameter selection based on performance profiles.
%    Note that the performance profiles also influence LLVM's optimization decisions.
%    For that reason we compared the configurations with adaptive parameter selection to a (faster) profile-enabled baseline.
%    The configuration \propername{Adaptive-PR1} uses the parameter range $[0,2]$ for the lower bound, and the range $[2,5]$ for the upper bound of the subsequent random trial.
%    As a result, the random trial for the coolest call site samples from $[2,5]$ and for the hottest call sites from $[0,2]$ (see \cref{ss:impl-pgo} for the details).
%    The configuration \propername{Adaptive-PR2} provides better probabilistic security by using larger parameter ranges.
%    Specifically, \propername{Adaptive-PR2} uses a range of $[1,5]$ for the lower bound, and a range of $[2,10]$ for the upper bound.
%    The resulting random trial range for the coolest call sites is $[5,10]$ and for the hottest call sites $[1,2]$.
%    \propername{Adaptive-PR2} differs from \propername{Adaptive-PR1} by allowing twice as many BTRAs in the cold call sites, and protecting even hot call sites with at least \emph{two} BTRAs (also see~\Cref{fig:average-decoy-counts} for BTRA distribution).
%
%    \rtwoc's use of performance profiles positively influences performance in two ways.
%    First, performance profiles allow for more aggressive inlining:
%    Inlined call sites have no return address and, therefore, also do not need BTRAs.
%    Second, the profiles allow \rtwoc to decrease the number of BTRAs for hot call sites.
%    To evaluate these two effects separately, we built a configuration that uses a fixed number of ten BTRAs in total \emph{despite} the availability of performance profiles, thus cancelling out the effect of adaptive parameter selection.
%    From the comparison we can see that adaptive parameter selection still increases performance over the \propername{Fixed} configuration already optimized with inlining.

\begin{table}[t]
    \begin{tabular}{lrr}
        \toprule
        {}                        & max  & geomean \\
        \midrule
        \propername{Push}         & 1.21 & 1.06    \\
        \propername{AVX}          & 1.10 & 1.04    \\
        \propername{\gls{heapbt}} & 1.05 & 1.02    \\
        \propername{Prolog}       & 1.06 & 1.02    \\
        \propername{Layout}       & 1.02 & 1.00    \\
        \bottomrule
    \end{tabular}
    \caption{The maximum and geometric mean overhead of \rtwoc{}'s components.
    See \cref{sss:eval-btra}, \cref{sss:eval-heapbt} and \cref{sss:eval-layout} for details.
    The overhead is relative to the baseline without \rtwoc{}.
    }
    \label{tab:perf-components}
\end{table}

\subsubsection{Memory overhead}
To evaluate \rtwoc{}'s memory overhead we linked the benchmark programs from the SPEC CPU 2017 suite to a static library that prints the \propername{maxrss rusage} metric once the program ends.
The \propername{maxrss} metric is the maximum resident-set size of a process during its lifetime.
We chose this methodology because it allows measuring the memory overhead without impacting the benchmark performance.
With this methodology, we found the memory overhead of the SPEC benchmarks to be 1-3\%.

For the webserver benchmarks we had to choose a different methodology because the webservers spawn child processes.
With child processes, \propername{maxrss} reflects the maximum usage among all child processes instead of the combined maximum usage.
Instead, we started a separate monitoring process that records the RSS usage of each webserver process every second and calculated the median.
With this methodology, we found the memory overhead of the webserver benchmarks to be about 100\%.
We verified experimentally that about 55\% of the memory overhead is caused by the page allocations for \glspl{heapbt}.
The rest is caused by \glspl{BTRA} and the increased binary size.

\subsection{Scalability}\label{ss:scalability}
Although the SPEC benchmark suite covers a wide variety of test programs, we also compiled real-world software with \rtwoc.
Apart from Apache and nginx, we also compiled the GTK version of WebKit~\cite{Webkit} and Chromium~\cite{Chromium}.
We built both browsers with a fixed total number of 10 \glspl{BTRA} per call site.
WebKit and Chromium are massive \cpp projects with more than 4.5 million lines and almost 32 million lines of C/C\plusplus code, respectively.

To verify that \rtwoc does not introduce errors into the browser, we ran the included tests as well as the Speedometer browser benchmark.
To pass the tests, we had to modify a single source file in Chromium, and three source files in Webkit to deactivate \rtwoc for a few functions.
In both cases, unprotected code called an \rtwoc compiled function with stack arguments.
We discuss this implementation limitation in \cref{ss:abi-change}.
We did not include the Speedometer performance results in the performance evaluation since Speedometer's results showed a variation of more than 20\% even in the baseline.
In daily browsing we did not notice any difference.

%%% Local Variables:
%%% mode: latex
%%% TeX-master: "../eurosys22"
%%% End:
