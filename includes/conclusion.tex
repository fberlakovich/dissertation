% Conclusion Chapter
\fbetodo{Unfinished}

This thesis explored the compiler's role \emph{beyond} optimization: as a tool for improving both the security and the reliability of software.
The unifying insight is that compilers possess semantic knowledge---about program structure, data layout, optimization decisions, and control-flow dependencies---that is difficult or impossible to recover after compilation.
By leveraging this knowledge, we can achieve hardening and testing capabilities that post-hoc approaches cannot match.

In Part~\ref{part:defense}, we presented \rtwoc, a compiler-driven defense against code-reuse attacks that exploits the compiler's knowledge of stack frame geometry.
Unlike prior randomization defenses that focused primarily on code layout, \rtwoc additionally randomizes the datalayout, particularly on the stack.
By inserting booby-trapped pointers that are indistinguishable from real ones, \rtwoc forces attackers to guess among decoys, increasing the difficulty of reconnaissance-based attacks.
\fbetodo{better summarize evaluation}

In Part~\ref{part:fuzzing}, we presented \lool, a compiler fuzzing framework that uses the compiler's optimization log as a domain-specific coverage signal.
Rather than treating the compiler as a black box and relying on generic code coverage, \lool directly observes which optimizations the compiler performs on each input.
A genetic algorithm uses this feedback to breed inputs that trigger rare optimization interactions.
Our evaluation on the GraalVM compiler showed that \lool outperforms both static parameter configurations and AFL-based parametric fuzzing in triggering rare optimizations.

Finally, we introduced \gls{PSC}, a technique that leverages compiler analyses to expose hidden program state to coverage-guided fuzzers.
The key insight is that compilers already perform sophisticated reasoning about data dependencies---for devirtualization, value-range analysis, and other optimizations.
By repurposing these analyses to transform data dependencies into control dependencies, PSC provides stepping stones that help fuzzers explore deeper into the state space.
Preliminary results do not show a clear advantage, calling for further research in that direction.
