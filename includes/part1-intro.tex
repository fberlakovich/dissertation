% Part I Introduction - Defending Against Code-Reuse with Compiler Assistance

Memory-unsafe languages like C and \cpp remain the foundation of critical infrastructure, and with them, the threat of memory corruption vulnerabilities persists.
Among the most dangerous exploitation techniques are \glspl{CRA}, which hijack a program's control flow to execute attacker-chosen code sequences.

This part of the thesis explores how compiler-driven software diversity can defend against such attacks.
In the context of software diversity, the compiler can help to \emph{maximize the information asymmetry} between the defender and the attacker in the defender's favor.
Its intimate knowledge of program structure---stack layouts, function boundaries, and data dependencies---enables it to introduce entropy and mimicry where it matters most.

\Cref{ch:code-reuse-coevolution} traces the co-evolution of \glspl{CRA} and their defenses, establishing the context for our contribution.
The subsequent chapter then presents \rtwoc, a defense that leverages compiler knowledge to randomize not only code but also the observable data layouts that recent attacks exploit for reconnaissance.
