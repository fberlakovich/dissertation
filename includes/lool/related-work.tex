\section{Related Work}

\paragraph{Swarm testing}
\lool{} follows the idea of swarm testing~\cite{Groce2012}.
The insight behind swarm testing is that inputs including more features are not necessarily beneficial and in some cases even detrimental to the testing effort~\cite{Groce2013}.
Features, in our case Java language constructs, can either trigger, suppress or be irrelevant for a certain bug.
Thus, \lool{} tries to achieve a large diversity of input features, which includes also the \emph{omission} of features in some inputs.
\citeauthor{Alipour2016} extended swarm testing to be \emph{directed}~\cite{Alipour2016}.
Like \lool{}, directed swarm testing incorporates statistical information about triggers and suppressors from previous runs into the generation of inputs.
However, \lool{} chooses the targets for its directed swarm testing approach automatically based on the optimization log coverage.
