\section{Motivation}

\begin{figure}[t]
    \centering
    \includeDrawioFigure{figures/r2c/r2c-overview}{
        \draw (PointA) node {\circledtikz{\code{\footnotesize A}}};
        \draw (PointB) node {\circledtikz{\code{\footnotesize B}}};
        \draw (PointC) node {\circledtikz{\code{\footnotesize C}}};
        \draw (Compare1) node {\LARGE$\not\equiv$};
        \draw (Compare5) node {\LARGE$\not\equiv$};
        \draw (Compare2) node {\LARGE$\equiv$};
        \draw (Compare6) node {\LARGE$\not\equiv$};
        \draw (Compare3) node {\LARGE$\equiv$};
        \draw (Compare7) node {\LARGE$\threeapprox$};
        \draw (Compare4) node {\LARGE$\equiv$};
        \draw (Compare8) node {\LARGE$\not\equiv$};
    }

    \captionsetup{margin={0pt,0.3cm},oneside}
    \caption{Prior systems primarily diversify code, leaving the layout of observable data predictable (left vs middle, see \cref{ss:aocr}).
    \rtwoc (right) diversifies code and observable data.}
    \vspace*{-1em}
    \label{fig:overview-unprotected}
\end{figure}


\fbetodo{Short r2c specific introduction}