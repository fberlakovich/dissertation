\section{Motivation}

The previous chapter traced the arms race between code-reuse attacks and defenses.
A recurring theme in this history is that each new defense closes one avenue of attack, only for attackers to find another.
Leakage-resilient code randomization, for example, effectively prevents direct code disclosure, but as \gls{AOCR} demonstrated, predictable \emph{data} layouts---particularly on the stack---still leak enough information to mount successful attacks.

This observation motivates \rtwoc: a defense that extends the principle of randomization from code to observable data.
The key insight is that the compiler possesses precise knowledge of stack frame geometry, including the positions of return addresses, spilled registers, and local variables.
By leveraging this knowledge, the compiler can transform stack layouts to disguise sensitive pointers among indistinguishable decoys---a technique we call \emph{mimicry}.

\begin{figure}[t]
    \centering
    \includeDrawioFigure{figures/r2c/overview}{
        \draw (PointA) node {\circledtikz{\code{\footnotesize A}}};
        \draw (PointB) node {\circledtikz{\code{\footnotesize B}}};
        \draw (PointC) node {\circledtikz{\code{\footnotesize C}}};
        \draw (Compare1) node {\(\LARGE\not\equiv\)};
        \draw (Compare5) node {\(\LARGE\not\equiv\)};
        \draw (Compare2) node {\(\LARGE\equiv\)};
        \draw (Compare6) node {\(\LARGE\not\equiv\)};
        \draw (Compare3) node {\(\LARGE\equiv\)};
        \draw (Compare7) node {\(\LARGE\threeapprox\)};
        \draw (Compare4) node {\(\LARGE\equiv\)};
        \draw (Compare8) node {\(\LARGE\not\equiv\)};
    }

    \captionsetup{margin={0pt,0.3cm},oneside}
    \caption{Prior systems primarily diversify code, leaving the layout of observable data predictable (left vs middle, see \cref{s:aocr}).
    \rtwoc (right) diversifies code and observable data.}
    \vspace*{-1em}
    \label{r2c:fig:overview-unprotected-vs-protected}
\end{figure}