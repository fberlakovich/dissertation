\section{Limitations}\label{r2c:ss:limitations}

\subsection{Coverage}\label{r2c:ss:limitations-coverage}
\rtwoc{} is able to protect only the call-sites and functions it actually compiles.
\glspl{BTRA} for calls to unprotected functions are disabled by default as these functions would overwrite all the \glspl{BTRA} after the return address.
Without \glspl{BTRA} after the return address, the return address would always be the last address in the list of addresses.
Overwritten \glspl{BTRA} are only an issue, however, if the attacker knows which parts of the program have not been compiled by \rtwoc.
Lacking this information, an attacker does not know where the return address is the last address.

\subsection{Change of calling convention}\label{r2c:ss:abi-change}
At present, our prototype implementation does not support calling functions with stack arguments from code not compiled by \rtwoc.
This incompatibility is due to such functions expecting the caller to prepare a frame pointer to account for the changed calling convention.
During our evaluation of \rtwoc{}, we encountered just three such cases (one in the unit tests of WebKit, one in the XML parser callbacks of WebKit, and one in the regular expression implementation of Chromium).
With three cases in 35 million lines of C/\cpp code, we conclude that this combination is rare in practice and, thus, opted for disabling the emission of \glspl{BTRA} for the affected functions.
Note that these cases could also be supported by automatically inserting a trampoline for externally visible functions with stack parameters.
% For ease of implementation we simply deactivated DRAs for the affected functions.

\subsection{Probability of Successful Guesses}
\glspl{BTRA} are mainly effective in deterring \emph{multiple} return address leaks, \eg if an attacker wants to infer multiple gadget locations from return addresses.
The reason is that guessing a single address location among the \glspl{BTRA} still leaves the attacker with a reasonably high chance of success.
Specifically, guessing a single return address location among 12 addresses (\ie 11 \glspl{BTRA}) has a success probability of ~9\%.
Increasing the number of \glspl{BTRA} within a single stack frame reduces this probability only marginally.
For example, doubling the number of \glspl{BTRA}, \eg with AVX512 would lead to 23 \glspl{BTRA}, but still a success probability of ~4\%.
Similarly, \glspl{heapbt} would need to outnumber real heap pointers on the stack with a ratio of 1 to 100 to reach a success probability of less than 1\%.
