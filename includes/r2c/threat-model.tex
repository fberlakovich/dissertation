\section{Threat Model and Assumptions}\label{r2c:s:threat-model}

Our threat model assumes that the attacker has access to a memory corruption vulnerability that enables control-flow hijacking.
In particular, we assume that the program contains gadgets for a ROP attack as well as suitable functions for a whole-function reuse attack.
In addition, we assume that the attacker can deterministically leak stack frames (e.g., with the help of Malicious Thread Blocking)~\cite{Rudd2017}.

% \subsection{Assumptions}
Our defense integrates with existing defenses.
We assume, specifically, that the data section is protected against code injection (\eg \wox or DEP)~\cite{DEP} and the text section with some form of execute-only memory.
Due to compatibility problems with \propername{xom-switch}, we could not evaluate the impact of execute-only memory in combination with \rtwoc{}.
However, the overhead introduced by execute-only hardware solutions is generally negligible~\cite{Zhang2018}.
% Other defenses are optional to, but compatible with, Readactor.

% We discuss the mutually beneficial aspects of  \rtwoc with other defenses in (see \cref{s:discussion}).
Our implementation focuses on protecting against information disclosure through the stack or the data section.
We do not consider other types of information leaks such as side-channels~\cite{Kim2014,Lipp2018,Kocher2019}.
Note that using side channels to uncloak EPT-based execute-only memory does not work~\cite{Goktas2020}.
Side channels could, however, be used to infer heap information.

%%% Local Variables:
%%% mode: latex
%%% TeX-master: "../eurosys22"
%%% End:
