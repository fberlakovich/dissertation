\section{Discussion}\label{psp:s:discussion}
% -----------------------------------------------------------------------------
% ANSWERS TO RESEARCH QUESTIONS
% -----------------------------------------------------------------------------
% RQ1 (VRP coverage): NO - No statistically significant improvement.
%     Avg +0.15% across LTO/non-LTO, A12 = 0.51 (negligible)
%
% RQ2 (Unfold coverage): NO - No statistically significant improvement.
%     Avg +0.46%, one benchmark (openssl_x509) shows +8.2% but not significant
%
% RQ3 (Throughput): NO PENALTY - VRP shows +1.8% faster execution
%     PSP transformations do not slow down fuzzing
%
% RQ4 (Bug finding): MIXED - Changes which bugs are found, not total count
%     TIF008: +10-50× with PSP; TIF014: -30-60% with PSP
%
% RQ5 (Significance): NO - Only 1/54 comparisons significant (a regression)
%     Effect sizes uniformly negligible (A12 ≈ 0.5)

% -----------------------------------------------------------------------------
% INTERPRETATION
% -----------------------------------------------------------------------------
% The data suggests PSP transformations have negligible impact on fuzzing
% effectiveness in 24-hour campaigns on FuzzBench targets.
%
% Possible explanations:
% 1. Campaign duration: 24h may be too short to see benefits in deeper state space
% 2. Benchmark selection: FuzzBench targets may not exercise transformed code paths
% 3. Transformation coverage: Compiler may not apply VRP/unfolding in hot paths
% 4. Already saturated: AFL++ may already cover relevant branches without help
% 5. Bug sensitivity: Different bugs favor different exploration strategies
%
% The time-series data shows benchmark-specific patterns:
% - libtiff: VRP+LTO maintains consistent ~3% lead (suggests VRP helps here)
% - openssl_x509: Unfold diverges after cycle 20 (late-stage benefit)
% - sqlite3: VRP reaches 90% coverage faster but final coverage similar
% These patterns suggest PSP may provide marginal benefits in specific contexts.

% -----------------------------------------------------------------------------
% LIMITATIONS
% -----------------------------------------------------------------------------
% - 24-hour campaign duration (longer campaigns may show different results)
% - FuzzBench benchmark selection (may not be representative of all software)
% - Single LLVM version (transformation applicability varies by compiler)
% - Coverage-based evaluation (PSP may help with properties not captured by coverage)

% =============================================================================
% FIGURES AND TABLES TO CREATE
% =============================================================================
% FIGURE 1: Final coverage comparison (bar chart with error bars)
%           Show 3-4 representative benchmarks, all 5 fuzzers
%
% FIGURE 2: Coverage over time (line plots)
%           (a) libtiff_magma - VRP+LTO advantage
%           (b) openssl_x509_magma - Unfold late divergence
%           (c) sqlite3_magma - VRP early advantage
%
% TABLE 1: Configuration overview (5 rows)
% TABLE 2: Final coverage statistical summary (3 comparisons)
% TABLE 3: Per-benchmark results (full 18 benchmarks, can go in appendix)
% TABLE 4: Coverage-over-time metrics (AUC, time-to-90%)
% TABLE 5: Magma bug triggering results (focus on TIF008/TIF014/PNG003)
% =============================================================================
