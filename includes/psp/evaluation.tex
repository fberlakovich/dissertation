\section{Evaluation}\label{psp:s:evaluation}

This section describes the evaluation of our prototype implementation described in \cref{psp:s:implementation}.
With our evaluation we aim to test the following hypotheses:
\begin{hypotheses}
    \hypothesis[psp:hpt:vrp-coverage]{\acrlong{VRP} Coverage} \acrlong{VRP} improves the fuzzer's ability to discover new edges;
    \hypothesis[psp:hpt:unfold-coverage]{Call Unfolding Coverage} Indirect call unfolding improves the fuzzer's ability to discover new edges;
    \hypothesis[psp:hpt:bugs]{Bug Finding with \acrlong{PSP}} \acrlong{PSP} transformations improve bug-finding capability.
\end{hypotheses}

\subsection{System Configuration}\label{psp:ss:system-configuration}

Experiments were conducted on a private Kubernetes cluster consisting of 10 identical worker nodes, each equipped with dual AMD EPYC 7H12 64-core processors (256 threads) and 1 TB of RAM, running Debian 12.
The cluster is orchestrated by K3s with the Volcano batch scheduler.
Fuzzing trials run in parallel across the cluster, with each fuzzer instance limited to a single dedicated CPU core and distributed evenly across machines.
Each fuzzer-benchmark combination ran for 24 hours, with the initial seed corpus included with the benchmark.
During fuzzing, the fuzzing infrastructure takes corpus snapshots at regular intervals.
Coverage is measured offline after the experiment completes by replaying each snapshot's corpus through a separate binary built with Clang's source-based coverage instrumentation.
Shared storage for corpora and results is provided via NFS, with results aggregated into a PostgreSQL database.
The evaluation infrastructure is based on a customized fork of FuzzBench.

\subsection{Evaluated Configurations and Benchmarks}\label{psp:ss:configurations}


\begin{table}[t]
    \centering
    \caption{Evaluated fuzzer configurations}
    \label{tab:psp-configurations}
    \begin{tabular}{>{\nextrow}E l l}
        \toprule
        {Configuration}     & {Description}                        & {Baseline}       \\
        \midrule
        \startdatarows
        aflplusplus         & Standard \aflpp (non-LTO)            & ---              \\
        aflplusplus\_lto    & \aflpp with LTO instrumentation      & ---              \\
        aflplusplus\_vrp    & \aflpp + \gls{VRP} transformations   & \propername{aflplusplus} \\
        aflplusplus\_unfold & \aflpp LTO + indirect call unfolding & \propername{aflplusplus\_lto} \\
        \bottomrule
    \end{tabular}
\end{table}

To test the effectiveness of \gls{PSP} and in particular dynamic call unfolding and \gls{VRP}, we build a separate configuration for each component.
We use \aflpp version \propername{4.32c} as the fuzzer.
\gls{PSP} configurations differ from the baseline configurations only in the instrumented binary because \gls{PSP} is compatible with any code-coverage-guided fuzzer.
As dynamic call unfolding relies on \gls{LTO} for a whole-program view, we test \propername{aflplusplus\_unfold} against an \aflpp configuration with \gls{LTO} enabled.
\cref{tab:psp-configurations} summarizes the evaluated configurations and their baselines.
We let each pair of fuzzer and benchmark run for 24h with a dedicated CPU core and repeated the experiment 30 times.

The 23 FuzzBench and 22 Magma benchmarks used in the evaluation are listed in \cref{app:tab:fuzzbench-benchmarks,app:tab:magma-benchmarks} in the appendix.
We had to exclude the \propername{poppler} and \propername{lua} benchmarks because of unresolved build issues with our benchmarking infrastructure.
For \propername{php} \propername{libjpeg} we only tested \propername{aflplusplus} and \propername{aflplusplus\_vrp} because the benchmark builds are not compatible with \gls{LTO}.

\subsection{Statistical Methodology}\label{psp:ss:statistical-methodology}

Following the recommendations of \citeauthor{klees2018} for fuzzer evaluation, we run 30 independent trials per configuration and apply statistical tests to determine whether observed differences are significant.
For all tests, we use a significance threshold $\alpha = 0.05$, \ie only observations with $p < 0.05$ are considered statistically significant.

\subsubsection{Coverage Comparison}\label{psp:sss:coverage-comparison-method}
For edge coverage, we use the \mwu test, since it does not assume a normal distribution.
The \mwu test allows us to estimate how likely an observed difference in edge coverage is assuming that there is no systematic difference between \gls{PSP} configurations and the baseline.
Intuitively, $p<0.001$ means that if there was no difference between \gls{VRP} and the baseline, the chance of observing such an outcome is less than 0.1\%.
Even a statistically significant difference can be negligible, however, which is why we also report the Vargha--Delaney effect size~$\hat{A}_{12}$.
The Vargha--Delaney effect size estimates the probability that a randomly chosen instance from one group outperforms a randomly chosen instance from the comparison group.
For example, the $\hat{A}_{12}=0.14$ of \propername{sqlite3\_ossfuzz} means that the chance of a randomly chosen \gls{VRP} instance achieving more edge coverage on \propername{sqlite3\_ossfuzz} than the baseline is 14\%.
We consider deviations of $|0.5 - \hat{A}_{12}| > 0.06$ as non-negligible, with further thresholds at $0.14$ (medium) and $0.21$ (large)~\cite{vargha2000}.

\subsubsection{Bug Triggering Comparison}\label{psp:sss:bug-triggering-method}
For per-bug triggering rates, we use Fisher's exact test on $2 \times 2$ contingency tables consisting of two rows (\gls{PSP} config and baseline) and two columns (\enquote{triggered} and \enquote{not triggered}).
The cells contain the number of instances that triggered (or did not trigger) the bug in the respective configuration.
Fisher's exact test tells us how likely such an observation (or a more extreme one) is, assuming that there is no difference between \gls{PSP} and the baseline.
One issue with this calculation is that with 33 bugs and, thus, 33 tests, the accepted false positive rate of $\alpha = 0.05$ accumulates.
Concretely, with $33 \times 0.05 = 1.65$ we expect between 1 and 2 bugs to show a significant difference ($p<0.05$) even if there was actually no real difference between \gls{PSP} and the baseline.
We thus apply the Benjamini-Hochberg correction to bound the false discovery rate at 5\%.
\cref{psp:tab:bugs-magma} shades only those differences whose $p$ values is below the adjusted threshold given by the Benjamini-Hochberg correction.

\subsection{Edge Coverage}\label{psp:ss:coverage}

We measure the achieved edge coverage by replaying the corpus found by each configuration on a binary instrumented with Clang's coverage sanitizer.
Measuring coverage with a common binary is particularly important since \gls{PSP} introduces new edge ids not present in the baseline binary.
\cref{psp:tab:coverage-fuzzbench,psp:tab:coverage-magma} show the final edge coverage comparison for FuzzBench and Magma benchmarks, respectively.

\begin{table}[t]
    \centering
    \caption{Final edge coverage on FuzzBench benchmarks: difference in median edge count~($\Delta$), relative difference~($\Delta\%$), Vargha--Delaney effect size~($\hat{A}_{12}$), and \mwu $p$-value.
    Shading indicates significant results with non-negligible effect: \colorbox{cellbetter1}{green} for improvement, \colorbox{cellworse1}{red} for degradation; intensity reflects effect magnitude (small/medium/large).
    See \cref{psp:sss:coverage-comparison-method} for details.}
    \label{psp:tab:coverage-fuzzbench}
    \footnotesize
    \setlength{\extrarowheight}{1pt}
    \resizebox{\textwidth}{!} & {$\hat{A}_{12}$} & {$p$} & {$\Delta$} & {$\Delta$\%} & {$\hat{A}_{12}$} & {$p$} \\
        \midrule
        \startdatarows
        \fileInput{generated/psp/coverage-fuzzbench}
        \bottomrule
    \end{tabular}}
\end{table}

The FuzzBench table shows that \gls{VRP} had a statistically significant ($p<0.05$) positive effect on 4 benchmarks, but also a statistically significant negative effect on 4.
There are positive and negative effects on other benchmarks as well, but their $p$ value is too large to rule out random measurement noise.
Of the statistically significant effects, \propername{libpng\_libpng\_read\_fuzzer} has a non-minimal positive effect size and \propername{sqlite3\_ossfuzz} a non-minimal negative effect size.
The relative negative delta ($-2.5$) of \propername{sqlite3\_ossfuzz} is significantly higher than the positive delta ($0.3$) of \propername{libpng\_libpng\_read\_fuzzer}.
For Call Unfolding, only two benchmarks show a statistically significant difference, but both with a minimal effect size.

\begin{table}[t]
    \centering
    \caption{Final edge coverage on Magma benchmarks (columns and shading as in \cref{psp:tab:coverage-fuzzbench}).}
    \label{psp:tab:coverage-magma}
    \footnotesize
    \setlength{\extrarowheight}{1pt}
    \begin{tabular}{>{\nextrow}E DRAP @{\hspace{1.5em}} DRAP}
        \toprule
        & \multicolumn{4}{c}{VRP} & \multicolumn{4}{c}{Call Unfolding} \\
        \cmidrule(lr){2-5} \cmidrule(lr){6-9}
        {Benchmark} & {$\Delta$} & {$\Delta$\%} & {$\hat{A}_{12}$} & {$p$} & {$\Delta$} & {$\Delta$\%} & {$\hat{A}_{12}$} & {$p$} \\
        \midrule
        \startdatarows
        \fileInput{generated/psp/coverage-magma}
        \bottomrule
    \end{tabular}
\end{table}

For the Magma benchmark suite, the edge coverage results are more clearly negative.
Specifically, 7 our of 20 benchmarks show a statistically significant negative effect on the final edge coverage, 6 of which have a moderate effect size.
The Call Unfolding results are all within the range of statistical uncertainty.

\subsection{Bug Detection (Magma)}\label{psp:ss:bugs}

\begin{table}[t]
    \centering
    \caption{Magma bug triggering statistics.
    Columns $n$ and $n_b$ show how many instances of a \gls{PSP} configuration ($n$) and how many instances of the baseline configuration ($n_b$) triggered the bug.
    TTF and TTF$_b$ show the median time to first trigger in minutes for the \gls{PSP} configuration and its baseline, respectively.
    Table cells which show a significant difference according to an exact Fisher test with Benjamini-Hochberg correction are shaded in \colorbox{cellworse3}{red}.
    See \cref{psp:sss:bug-triggering-method} for details.}
    \label{psp:tab:bugs-magma}
    \footnotesize
    \begin{tabular}{>{\nextrow}E E NN S[table-format=4.0] S[table-format=4.0] NN S[table-format=4.0] S[table-format=4.0]}
        \toprule
        & & \multicolumn{4}{c}{VRP} & \multicolumn{4}{c}{Call Unfolding} \\
        \cmidrule(lr){3-6} \cmidrule(lr){7-10}
        {Benchmark} & {Bug} & {$n$} & {$n_b$} & {TTF} & {TTF$_b$} & {$n$} & {$n_b$} & {TTF} & {TTF$_b$} \\
        \midrule
        \startdatarows
        \fileInput{generated/psp/bugs-magma}
        \bottomrule
    \end{tabular}
\end{table}

\cref{psp:tab:bugs-magma} shows the per-bug triggering results on Magma benchmarks.
For each seeded bug, we report how many of the 30 trial instances triggered it and the median time to first trigger~(TTF) in minutes among instances that did trigger.
As pure number differences can be misleading, we apply Fisher's exact test within the two groups (\gls{VRP} vs.\ baseline and Call Unfolding vs.\ baseline) for each bug and apply the Benjamini-Hochberg correction.
The shaded cells show that \gls{VRP} has a significantly weaker bug-finding performance for two of 33 bugs.
Triggering rates for other bugs and for Call Unfolding do not show statistically significant differences compared to the respective baseline.
